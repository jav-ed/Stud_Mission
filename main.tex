\documentclass[a4paper, 11pt]{scrreprt}

%Region
\usepackage[utf8]{inputenc}
\usepackage{amsmath}
\usepackage{amsfonts}
\usepackage{amssymb}

% To get some features - page styling
\usepackage{fancyhdr}
\usepackage[left= 3.5cm,right = 2cm, bottom = 3 cm]{geometry}

% Clickable Ref and fast navigation through table of
% recognition of a table of content
% see: https://tex.stackexchange.com/questions/50747/options-for-appearance-of-links-in-hyperref
\usepackage{hyperref}
 \hypersetup{
    %  deactivate the border
     colorlinks=true,
    %  color of the refs
     linkcolor=black,
     filecolor=blue,
     citecolor = black,
     urlcolor=cyan,
     }


% graphics library
\usepackage{graphicx}

% both packages required for Inkscape pic imports
\usepackage{transparent}
\usepackage{color}

%  So the graphics can be found
\graphicspath{{2_Figures/},{2_Figures/1_Task/}, {2_Figures/2_Task/},{2_Figures/3_Task/},
               {2_Figures/0_Deco/}, {2_Figures/0_Deco/}, {2_Figures/3_Task/1_Steps/},
               {2_Figures/3_Task/1_Steps}, {2_Figures/3_Task/1_Steps/Version_2_Correct_Steps/*}, 
               {2_Figures/3_Task/1_Steps/Version_3_Same_Tol/*}  }
               

% \graphicspath{{2_Figures/*} }


% Um den Namen des Autos aus Bibtex anzuzeigen
%\usepackage[round, authoryear]{natbib}
\usepackage{natbib}

%Zitationsstil
%\bibliographystyle{unsrt}
%\bibliographystyle{unsrtnat}
% \bibliographystyle{plainnat}
% \bibliographystyle{apacite}
\bibliographystyle{plain} %[plain]

% % for tables which needs more than one page
% \usepackage{longtable}

\usepackage{multirow}

% % --------------- Table and Color ---------------
% % Table coloring
% \usepackage[table]{xcolor}
% \definecolor{lightgray}{gray}{0.9}
% \definecolor{lightblue}{rgb}{0.93,0.95,1.0}
% \definecolor{orange}{rgb}{0.93,0.95,1.0}

% %\usepackage{tabu}
% \usepackage[table]{xcolor}

% \definecolor{tableHeader}{RGB}{211, 47, 47}
% \definecolor{tableLineOne}{RGB}{245, 245, 245}
% \definecolor{tableLineTwo}{RGB}{249, 199, 29}
% % --------------- Table and Color- END ---------------

% For splited lists
\usepackage{multicol}

%If you want to generate a caption without being inside of a float object
\usepackage{caption}
% \usepackage{subcaption}

% For optimization problems definition's
\usepackage{optidef}

\usepackage{printlen}


% % Code-Implementation matlab
% \usepackage[numbered,framed]{matlab-prettifier}

% % Load Tikz
% \usepackage{pgfplots}

% % Abbreviation-Table
% \usepackage{acronym}

% % ---------------------- XDSM --------------------------

\usepackage{tikz}

% % Optional packages such as sfmath set through python interface
\usepackage{sfmath}

\usepackage{blindtext, rotating}


% % Define the set of TikZ packages to be included in the architecture diagram document
\usetikzlibrary{arrows,chains,positioning,scopes,shapes.geometric,shapes.misc,shadows}

% % Set the border around all of the architecture diagrams to be tight to the diagrams themselves
% % (i.e. no longer need to tinker with page size parameters)
% \usepackage[active,tightpage]{preview}
% \PreviewEnvironment{tikzpicture}
% \setlength{\PreviewBorder}{5pt}


% % ---------------------- XDSM  END --------------------------
% to automatically ensure floats do not go into the next section.
\usepackage[section]{placeins}



% --------------- page header ---------------
\pagestyle{fancy}
\fancyhf{}

% on the left side - section name
\fancyhead[L]{\rightmark}

% pagenumber
\fancyhead[R]{\thepage}

\renewcommand{\headrulewidth}{0pt}
%--------------- Page Header END ---------------


% Shortcuts - see:
% https://github.com/James-Yu/LaTeX-Workshop/wiki/Snippets#Font-commands

%Endregion


\begin{document}

% Just required to get all the symbol numbers in
% order to define the symbol numbers for the arabic language
% \input{1_Latex_Files/0_Arabic_Letters.tex}

% % ---------------- Deco Stuff  ------------------------------


    % \includegraphics[width=0.42\textwidth]{./2_Figures/TUBraunschweig_4C.pdf} & 

    \begin{center}
      \begin{tabular}{p{\textwidth}}
      
      \begin{minipage}{0.5\textwidth}
      \centering
      \includegraphics[width=0.89\textwidth]{./2_Figures/TUBraunschweig_4C.pdf}
      \end{minipage}
      \begin{minipage}{0.5\textwidth}
      \centering
      \includegraphics[width=0.5\textwidth]{./2_Figures/0_Deco/dlr_Logo.jpeg}
      \end{minipage}
      
      
      
      \\
      
      \begin{center}
      \LARGE{\textsc{
        Development of a module for mission analysis for a gradient-based aerodynamic shape optimization process\\
      }}
      \end{center}
      
      \\
      
      
      \begin{center}
      \large{Technische Universität Braunschweig \\
      Department of Mechanical Engineering \\
      Chair for Overall Aircraft Design
      }
      \end{center}
      
      \\
      
      \begin{center}
      \textbf{\Large{Studienarbeit}}
      \end{center}
      
      
      \begin{center}
        Student research project
      \end{center}
      
      
      \begin{center}
      written by
      \end{center}
      
      \begin{center}
      \large{\textbf{Javed Butt}} \\
      
      \large{5027847} \\
      \end{center}
      
      \begin{center}
      \large{born on 20.05.1996 in Gujrat}
      \end{center}
      \\
      
      \begin{center}
      \begin{tabular}{lll}
      \textbf{Submission date:} & & 19.08.2021\\
      \textbf{Examiner :} & & Prof. Dr.-Ing. Stefan Görtz\\
      \textbf{Supervisor :} & & M.Sc. Andrei Merle (M.Sc.)\\
      
      
      \end{tabular}
      \end{center}
      
      \end{tabular}
      \end{center}
      %Damit die erste Seite = Deckblatt nicht nummeriert wird.
      \thispagestyle{empty}




      

\chapter*{Acknowledgments}

All praise and thanks to the 
\textbf{ONE} to Whom all praise, 
thanks and gratitude belongs. 

\vspace{1cm}
My special thanks goes to Andrei Merle, who was 
the supervisor of this work. He showed  his 
knowledge in a modest way, supported 
me actively in getting this work completed by 
helping me solving equations, reviewing my 
work and providing beneficial tricks and tips. 
Especially, when it came to solving equations,
I really appreciated his help. 
Thank you for your time and 
effort you spend on this work Andrei Merle. \newline

Also, Professor Stefan Görtz - At
each end of our conversations, he would 
ask me to call him, if any problems or 
questions would arise - thank you for 
your support Professor Görtz.


% % % ---------------- Selbstständigkeitserklärung ------------------------


\chapter*{Declaration of independent authorship}

I hereby declare that the present work, the Studienarbeit,
is solely and independently done by myself in all
aspects, such as developments, code implementations, and writing of report. In
addition, I confirm that I did not use any tools, materials or sources other than
those explicitly specified.\newline \break

\vspace{1cm}
\noindent Full name: Javed Butt \newline \break
\noindent Date and place: 19.08.2021, Braunschweig\newline \break

\vspace{1cm}
\noindent Signature:

\begin{figure}[!h]
    \centering
    \includegraphics[width =0.2\textwidth]
    % In order to insert an eps file - Only_File_Name (Without file extension)
    {2_Figures/0_Deco/signature_1.jpg}
    % \caption{Adapted coordinates by using cosine function and initial CST modes}
    \label{fig_0_signature}
\end{figure}




% =====================================================================
% ========================= Lists =====================================
% =====================================================================
\tableofcontents
% \listoffigures
% \listoftables

% =====================================================================
% ========================= Main Tasks ================================
% =====================================================================

% % % ---------------- Task 1  ------------------------------


\chapter{Introduction}
\label{chap_1_Intro}
This work is written for the \emph{Technische Universiät Braunschweig}
in a cooperation with the German Aerospace Center (\emph{DLR}).
It is the result of research topic defined by my supervisor 
Andrei Merle and Prof. Stefan Görtz, both working for the DLR. The overall goal, 
in very brief terms, is to calculate the fuel consumption 
and its gradients with respect to later introduced shape parameters 
for one or multiple individual flight missions. The developed tool for that purpose 
is called \emph{missioninformer}.
The workflow of the \emph{missioninformer} can be explained  
with figure \ref{fig_1_Missinfo}. \emph{Missioninformer}
can be treated as a black box, which makes its implementation into 
existing analysis and optimization processes
easier. It receives an input, the so-called 
\emph{mission input}, which contains descriptive information about 
a flight mission, e.g. payload, cruise range, flight Mach number (a 
more detailed explanation will be given in section \ref{fig_1_Missinfo})
and aerodynamical data.
With these inputs \emph{missioninformer} provides the user with 2 outputs.
The first one is a state value, it is the amount of fuel in kg which is 
required for the mission. The second output are the gradients of 
the fuel for the mission with respect to arbitrary aircraft 
shape parameters. \newline


% ===================================================
% ==================== Motivation ===================
% ===================================================
\section{Motivation}
\label{sec_Motivation}
Aviation has become an 
indispensable part of global economy. Its importance is 
evident in the private sector, where it is connecting
destinations worldwide
and in the business sector, where goods are transported and 
business trips are undertaken. Since an aircraft burns fuel and thus 
emits carbon dioxide (CO2), nitrogen oxide (NOx) and 
sulphur dioxide (SO2) its impact on 
the environment is significant 
and cannot be neglected. Emissions are so high that the European commission has 
formulated a document called Flightpath 2050, where it defines two 
of its goals to be reduction of CO2 and NOx emission by 75\% 
per passenger kilometer
and 90\%, respectively. 
Currently, commercial aviation is responsible
 for around 
3.5 \% of the global carbon dioxide and nitrogen 
oxide 
emission \cite{lee_contribution_2021}
and it can be assumed that in an increasingly globalized world the demand 
for fast and convenient mobility will not diminish. 
Industry and governments 
around the world realized that taking active steps towards a cleaner 
flying are essential for saving the planet earth. In 2010, the 
international Civil Aviation Organization (\emph{ICAO}) formulated 
industry-wide goals for reducing carbon emissions. 
These goals can be summarized by 2 main 
objectives, to establish carbon-neutral growth beyond  
year 2020 and to further reduce carbon emissions to half of 
the current level by the year 2050.\newline

One simple way of reducing the overall carbon emissions  
already set to practice is the application of single-engine taxiing. 
With a single operating engine
the fuel emissions are reduced only 
for the time in which the taxiing and waiting are carried out.
Considering a whole flight mission, taxiing and waiting
represents only a small fraction of the mission. Thus, this is 
not enough. Other techniques to reduce fuel consumption 
already in use are flying at an optimal cruise speed or 
using more environmental-friendly handbook trajectories. Methods 
which still acquire much exploration, however, 
can be assumed to play a main part in reducing fuel consumption.
These are overall new 
aircraft configurations, e.g. blended wing body and applying
technologies like laminar flow control, boundary layer 
ingestion, ultra-high bypass ratio engines, more and all electric 
aircraft. The possibility 
to combine multiple technologies is given and 
is expected to lead to best results. The proposed \emph{missioninformer} 
is not restricted to any aircraft configuration nor 
to any new applied technological advancements.\newline

The demand for more environmental-friendly 
aircraft is 
becoming steadily louder. Furthermore, fuel price 
has increased and is 
forecasted to increase in the future. Among others, one 
of the main objectives of
the \emph{DLR}'s institute 
\emph{Aerodynamics and Flow Technology} is to fulfill the described demands. 
Therefore, several 
optimization process were developed here
to find  structural, aerodynamical and 
engine optimized aircraft designs, 
details can be found in \cite{gortz_collaborative_2016, goertz_overview_2020}. 
The fuel consumption 
is targeted as the objective function inside an optimization process. The big 
advantage of such an optimization would result in reducing the CO2 emissions and 
thus pleasing the environmental demands. The \emph{missioninformer} is 
written in a modular manner as mentioned earlier in this chapter.
Therefore, it can be easily implemented into the \emph{DLR}'s 
multi-disciplinary, -point and -objective optimization frameworks. 
Also, the user is enabled to define his own mission or multiple missions. 
Therefore, this feature could be used to enhance the already existing 
multi-point capabilities for the optimization. The importance 
of a multi-mission-optimization becomes visible, when thinking about 
an aircraft, which is used for a wide variety of missions, e.g. 
for different kinds of ranges, flight altitudes, speeds and payloads. 
In case of a single mission optimization, the given 
flight parameter will be suited optimally. However, a small 
change in these conditions could result into unbearable consequences of
the fuel consumption. 
In practice, airlines want to buy aircrafts, which can be used 
for more than just one mission. A multi-mission optimized 
aircraft performs not as good for each condition, however it 
will cover a variety of missions and conditions within reason. The 
missioninformer, however, is not only suited for multi-point 
optimization, rather the enhanced multi-mission optimization.
For instance in the previously mentioned processes masses for which 
the aircraft is trimmed are defined.
Masses are defined through the method of multi-points, i.e.
these are 
only frozen and discontinuous discrete aerodynamic states for which 
an optimization is carried out. In case of the multi-mission
capable \emph{missioninformer}, 
the masses are defined continuously, which comes closer 
to the reality of flying and thus consuming 
fuel. Furthermore, the \emph{missioninformer} takes 
the snowball mass effect, which is described in 
section \ref{sec_Fuel_Mass_iter} into account. Also, 
since the input data is given by the aerodynamic output 
of \emph{DLR}' analysis and 
optimization workflow \emph{FSAerOpt} 
\cite{merle_high-fidelity_2019} involving the 
flow solver \emph{TAU}, each mission fuel mass 
computation gets 
aerodynamical data for a fully trimmed 
state.
\newline


With this introduction the main 8 tasks for 
developing the \emph{missionionformer} from scratch shall be mentioned:
\begin{enumerate}
    \item Literature review on mission analysis and review of Breguet's range formula.
    \item Familiarization with the gradient-based optimization process
    \item Conception of an interpolation module for mission analysis
    \item Implementation of the Python module
    \item Derivation of the module according to the optimization parameters
    \item Verification of the derivation in case the derivation is done by complex-step or analytically
    \item Interface for a gradient-based optimization process
    \item Identification of a best practice
  \end{enumerate}
As far it is possible, each task will be 
elaborated in this article. Since the tasks required much coding, 
it is important to 
mention the used programming language and the dependencies. 
As for the programming language,
\emph{Python 2} and \emph{Python 3} were chosen. For the libraries  
\emph{Matplotlib, Scipy, NumPy} and \emph{DLR}'s own 
Python library \emph{SMARTy} are deployed. For local coding a 
Linux workstation provided 
by the \emph{DLR} with the following hardware configuration 
was used: \emph{CPU:} Intel(R) Xeon(R)
CPU E3-1270 v3 @3.50GHz 
and 32GB of RAM. Calculations were performed on the  following high 
performance computer \emph{CARA}: \emph{CPU:} 2x AMD EPYC 7601 
(32 cores; 2,2 GHz) per node 
and 2168 nodes with 128 GB DDR4 (2666 MHz) RAM. 
All calculations performed in 
this report used the \emph{64} cores option. Some 
parts of \emph{missioninformer} where NumPy is used, can 
be assumed to benefit from this option. However, the 
missioninformer itself is not parallized.

% ===================================================
% ==================== STATE OF THE ART =============
% ===================================================
\section{State of the art}
\label{sec_1_State}
The idea to reduce the entire fuel consumption is not
new and a summary
of the available methods is given in \cite{betts_survey_1998} 
and a more recent one in \cite{kao_modular_2015}. However,
some works still shall be recited. Flight Optimization System 
(\emph{FLOPS}) is a modular approach allowing the user to select 
appropriate modules for a given mission \cite{mccullers_aircraft_1984}.
The aerodynamical data are generated by using the \emph{Delta Method}, 
which is an empirical drag prediction method \cite{feagin_delta_1978}. 
\emph{FLOPS} enables some numerical optimization schemes, e.g. Davidon-Fletcher-Powell,
Broyden-Fletcher-Goldfarb-Shano and a Quadratic Extended Interior Penalty method 
\cite{kao_modular_2015}. The gradients are calculated using finite 
differing, which is easy to apply, however an adequate 
derivation step size is required. The accuracy of gradients may 
not be sufficient and in case of a high number of design variables 
parallelization should be considered in order to get gradients 
in an acceptable time limit. Transport Aircraft System Optimization 
(\emph{TASOPT}) is another well known tool for mission analysis and 
optimization \cite{greitzer_design_2010}. The benefit of the latter 
is that physics-based models, rather than empirical data-tables are 
applied. Using 
data-tables results in interpolation could result into not 
realizable designs. The physics for \emph{TASOPT} is obtained mostly by low-order 
models.\newline 

\emph{pyACDT} is an aircraft conceptual design tool \cite{perez_pyacdt_2008} 
containing a mission analysis component. It uses the Breguet 
range equation and empirically obtained fuel fractions for each 
flight mission segment. Liem et al. \cite{liem_aerostructural_2013} 
take a similar 
approach, however with the advancement, that fuel fractions are only 
going to be used for the startup, taxi and landing segments of the mission.
The fuel burn for climb, cruise and descent segments are calculated 
using a range equation and the flight equilibrium equations. The 
different segments are linked together such that the fuel weights at 
the end points of two neighboring segments must be equal. 
Another well known tool for aircraft conceptual design 
is developed by Lissys Ltd and is called \emph{PIANO}. It 
is freely available, however, only for Windows and thus 
not compatible for HPC-based optimization processes.
\newline 

Probably the two most well known mission analysis tools 
are \emph{pyMission} and \emph{SUAVE}. 
\emph{pyMission} is developed by the authors of 
\cite{kao_modular_2015}
and is now part of the openMDAO framework \cite{gray_openmdao_2019}. It 
uses the flight equilibrium equations and a fuel 
burn rate equation. In order to solve the fuel
burn rate equation, which is an ODE, the amount 
of fuel carried at the end of the mission 
is provided as the initial condition. The explicit Euler 
scheme starting from the end of the mission is applied. 
Thus, marching backwards in distance to the start 
of the mission is performed. For the rest of the mission 
fuel fractions are used. The gradients 
(total derivatives) for 
the optimization performed are obtained 
through the adjoint method \cite{martins_review_2013}. 
\emph{SUAVE} is a conceptual 
aircraft design framework developed at  
Stanford University and can be used for conventional 
and unconventional configurations. Tutorials as well 
as documentation for \emph{SAVE}'s mission analysis tool 
can be found on their 
homepage \cite{noauthor_homepage_2021}. 
In \cite{liem_multimission_2015, liem_aerostructural_2013,
liem_surrogate_2015} the mission analysis 
is applied by using surrogate models and mixture 
of experts. Different kind 
of Kriging methods and RBFs (Radial Basis Function) are used 
to get interpolation models for lift, drag and pitching 
moment coefficients from a four-dimensional input space with 
variables, Mach number, angle of attack, flight 
altitude and tail rotation angle. For startup, taxi, takeoff 
and landing fuel fractions are used. The fuel burn for 
climb, cruise and descent segments are derived through 
numerical integration of a range equation. 
The results of the paper can be summarized as follows: 
the adaptive sampling improves the accuracy of surrogate models, 
the convergence was to be found slow in some cases, particularly when modeling 
complex profiles. Traditional 
surrogate models (RBF and Kriging without mixture of experts)
performed well for the simpler 
lift and pitching moment coefficient ($C_L, C_M$)
profiles. Using a mixture of gradient enhanced Kriging (GEK)
models to approximate drag
coefficients gave approximation errors of less than 
5\% with less than 150 samples, whereas the adaptive sampling
failed to converge when training a global model. Since 
the traditional surrogate models approximate 
less complex behavior well, the authors of 
the papers recommend applying traditional 
surrogate modelling instead of mixture 
of experts. Latter requires much effort 
for its implementation.
Finally, it is worth mentioning Dabas et al., who optimize
the pylon shape by coupling an aerodynamic
optimization with a mission
analysis tool in order to propagate aerodynamic 
shape modifications to mission level \cite{dabas_error-based_2019}. The objective 
function, however is not directly the 
amount of fuel burnt, but rather the Cash Operating Costs (COC),
which contain the burnt fuel as part of its weighted components.


% ===================================================
% ============Introduce Missioninformer =============
% ===================================================
\section{Introducing missioninformer}
\label{sec_introduc_Miss}
In this section the overall processes and the functionalities 
of the proposed tool \emph{missioninformer} shall be presented 
in an abstract way, without intern solved equations at this stage.
It is a code written entirely in \emph{Python} and makes 
use of the following libraries:
\begin{multicols}{2}
    \begin{enumerate}
        \item NumPy \item SciPy  \item matplotlib
        \item SMARTy (DLR's own Python library)
        \item re (regex)
        \item[\vspace{\fill}]
    \end{enumerate}
\end{multicols}

Its main workflow can be described with the figure \ref{fig_1_Missinfo}. It 
obtains two input files and outputs the mass of total fuel which is required 
for the given mission and the gradients of the mission fuel with 
respect to shape parameters. 
The mission input file is completely written in \emph{Python} and uses a regular dictionary, 
which is similar to a json file for storing purposes. 
In order to define one mission in the mission input file, the 
following 3 groups of parameters need to be set. The first 
group, named masses, contains weights:
\begin{multicols}{2}
    \begin{enumerate}
        \item maximal takeoff weight
        \item maximal landing weight  
        \item operating empty weight
        \item manufactures empty weights
        \item maximum zero fuel weight
        \item maximum fuel weight
        \item design payload
        \item maximum payload
    \end{enumerate}
\end{multicols}

\begin{figure}[!h]
    \def\svgwidth{\linewidth}
    \input{2_Figures/1_Task/1_Missioninformer.pdf_tex}
    \caption{Broad overview: Workflow \emph{missioninformer}}
    \label{fig_1_Missinfo}
\end{figure}

\FloatBarrier
The second group, named flying parameters, consists of the 
cruise Mach number (\emph{Ma}), 
cruise altitude (\emph{h}) and cruise range. The third group can be 
adjusted optionally by the user. It allows to change the step size 
for the central differencing scheme, which will be discussed 
in section \ref{sec_Step_Size_Study} and define 
a weighting factor for each mission. The step size steers the accuracy 
and reliability of the calculated gradients. In case, multiple missions 
shall be used, for each mission a weighting factor must be 
given. Using multiple missions or adding missions is done easily by  
copying the section of the parameters and
redefining these with desired values. Each new mission is 
summed to as one overall variable and needs to be passed 
as an argument to the mission collection 
method. \newline

The second input for the \emph{missioninformer} is a database. It is required 
to generate surrogate, i.e. interpolation models for the state variables 
$LoD, AoA \,(\alpha), TSFC$ and their gradients with respect to all 
shape parameters. $LoD$ is called glide ratio, sometimes referred to 
performance and defined as: $LoD = \frac{C_L}{C_D} = \frac{L}{D}$, where 
the prefix \emph{C} stands for coefficient and the letter \emph{L} and \emph{D} 
are denoted as lift and drag, respectively. $AoA$ is the \textbf{a}ngle 
\emph{o}f \textbf{a}ttack 
and is commonly denoted with the greek letter $\alpha$. $TSFC$ is 
called the \textbf{T}hrust \textbf{S}pecific \textbf{F}uel \textbf{C}onsumption.
In order to get an interpolation model for the state and gradient variables, 
the database must contain values for those and their associated inputs
$Ma, h, mass$. The $mass$ is the total mass at the related condition
for which the aircraft is trimmed.
In simpler terms, considering figure \ref{fig_2_Lod_Interpol},
it can be observed that for combinations of $Ma, h, mass$ 
their associated known values for $LoD$ are passed to an interpolation generator. 
Having obtained an interpolation model, for any given set of 
$Ma, h, mass$ the corresponding $\tilde{LoD}$ can be calculated. It should 
clearly stated that, the closer the input set ($Ma, h, mass$) 
values are to those which were used to generate the interpolation model,
the better the prediction for the outcome $\tilde{LoD},
\tilde{AoA}, \tilde{TSFC}$ is. This means, if 
the given input set is highly out of the range of the training data, 
then an absurd extrapolation may occur. Figure \ref{fig_2_Lod_Interpol} 
only depicts the workflow for obtaining an interpolation model for 
$LoD$. However, analogously to the explained proceeding by only 
replacing $LoD$ with the desired output variable ($AoA, TSFC$ and their 
gradients), 3 interpolation models for the state variables and the
remaining for the gradients can be achieved. \newline 

Since for each state variable ($LoD, AoA, TSFC$) the gradient with 
respect to each shape parameter is required, the number of the 
interpolation models for the gradients is directly connected
to the number of 
the shape parameters. For example, having 126 shape parameters, 
results in having 126 gradients for each state variable. Thus,
126 interpolation models for the gradients of each state variable
($LoD, AoA, TSFC$) 
is necessary. In total, $126*3 = 378$ interpolation models only 
for the gradients are required.\newline 
% ============================================
% ================Interpolation =====================
% ============================================

\begin{figure}[!h]
    \def\svgwidth{\linewidth}
    \input{2_Figures/1_Task/2_LoD.pdf_tex}
    \caption{Exemplary interpolation model generation}
    \label{fig_2_Lod_Interpol}
\end{figure}

Even though it might seem to be a heavy computation it 
is much faster compared to the 
CFD calculation performed for each sample point 
to provide the state variables 
and their gradients as training data.
By using the interpolation approach the integration of 
the Breguet's range equation and the iteration 
of the mission fuel consumption 
becomes feasible although using 
accurate but time-expensive CFD-based data. Furthermore,
the tool 
scales nicely with an increasing number of 
missions.
However, 
the disadvantage is that the quality of the calculations 
completely depends on the quality of the interpolation models. The 
interpolation model itself depends on the accuracy of 
the provided input or training data as well as the 
underlying method. Furthermore, 
the final goal is to use the \emph{missioninformer} within a 
gradient based optimization. The optimizing algorithm then 
heavily relies on the correctness of the gradients. 
Additionally, gradient computations itself is a sensitive 
calculation to do. With the importance of the correct chosen 
interpolation model in mind, in section \ref{sec_Surrogates}
two commonly used interpolation methods and several options are investigated.
\newline


The database containing the training samples 
is given by the output of \emph{DLR}'s aerodynamic analysis and 
optimization workflow \emph{FSAerOpt} 
\cite{merle_high-fidelity_2019} involving the 
flow solver \emph{TAU}.
It is calculated at trimmed states, however, trimming is not always
feasible. 
In this case, \emph{missioninformer} automatically identifies the not 
trimmed values and does not include them into the intern \emph{Python} 
database. This feature is part of the modular skills of the 
missioninformer. It thus enables an easy extension or insertion 
to an already existing workflow. This is reinforced by the type of output, 
i.e. the total fuel burn and their gradients are stored as simple ASCII text files 
and as NumPy arrays as well. In case of a \emph{Python} based 
workflow, the NumPy 
arrays can be loaded into the RAM (Random Access Memory)
with one additional line of code. 

\subsection{Workflow}
A more detailed workflow is depicted in figure \ref{fig_3_Workflow}. 
Before discussing the figure, note that, this is still a broad workflow, and 
it is supposed to provide only a general and not a deep understanding. Among 
other, the cruise segment fuel equation, its derivative, its solution, the used 
interpolation models and outer loops for example for the mass recalculations 
are skipped for now. The
detailed elaboration happens in chapter \ref{chap_Methodlogy}. 
The workflow \ref{fig_3_Workflow} can be read as follows: After 
having generated the 
database containing values for $LoD, AoA, TSFC$ and their 
gradients for a given set of inputs ($Ma,h, mass$), the mission input file 
can be adjusted in order to define the flight mission. In case of 
a multi-mission definition one more step regarding weighting is done, which 
is not depicted in the current workflow overview.
% ============================================
% ================ 2nd Workflow ==============
% ============================================
\begin{sidewaysfigure} [!h]
    \resizebox{\textwidth}{!}{
    
%%% Preamble Requirements %%%
% \usepackage{geometry}
% \usepackage{amsfonts}
% \usepackage{amsmath}
% \usepackage{amssymb}
% \usepackage{tikz}

% Optional packages such as sfmath set through python interface
% \usepackage{sfmath}

% \usetikzlibrary{arrows,chains,positioning,scopes,shapes.geometric,shapes.misc,shadows}

%%% End Preamble Requirements %%%

\input{/home/jav/Progs/Virt_Env/writing/lib/python3.12/site-packages/pyxdsm/diagram_styles.tex}
\begin{tikzpicture}

\matrix[MatrixSetup]{
%Row 0
\node [DataIO] (output_mission_Inf) {$\begin{array}{c}mission input, database\end{array}$};&
&
&
&
&
&
\\
%Row 1
\node [Function] (mission_Inf) {$\begin{array}{c}\text{missioninformer}\end{array}$};&
\node [DataInter] (mission_Inf-st_interpol) {$\begin{array}{c}$$LoD, AoA, TSFC$$\end{array}$};&
\node [DataInter] (mission_Inf-grad_interpol) {$\begin{array}{c}\frac{LoD}{dp}, \frac{d\alpha}{dp}, \frac{dTSFC}{dp}\end{array}$};&
&
&
&
\\
%Row 2
&
\node [Function] (st_interpol) {$\begin{array}{c}\text{state intp. model:} = \tilde{S}\end{array}$};&
&
\node [DataInter] (st_interpol-st_fuel_burn) {$\begin{array}{c} \tilde{S}\end{array}$};&
\node [DataInter] (st_interpol-grad_fuel_burn) {$\begin{array}{c} \tilde{S}\end{array}$};&
&
\\
%Row 3
&
&
\node [Function] (grad_interpol) {$\begin{array}{c}\text{grad intp. model:} = \tilde{G}\end{array}$};&
&
\node [DataInter] (grad_interpol-grad_fuel_burn) {$\begin{array}{c}\tilde{G}\end{array}$};&
&
\\
%Row 4
&
&
&
\node [ImplicitFunction] (st_fuel_burn) {$\begin{array}{c}\text{fuel burn state} \\ \text{equation}\end{array}$};&
&
&
\node [DataIO] (right_output_st_fuel_burn) {$\begin{array}{c}\text{state fuel burnt}\end{array}$};\\
%Row 5
&
&
&
&
\node [ImplicitFunction] (grad_fuel_burn) {$\begin{array}{c}\text{fuel burn gradient} \\ \text{equation}\end{array}$};&
&
\node [DataIO] (right_output_grad_fuel_burn) {$\begin{array}{c}\text{gradients fuel burnt}\end{array}$};\\
%Row 6
&
&
&
&
&
&
\\
};

% XDSM process chains


\begin{pgfonlayer}{data}
\path
% Horizontal edges
(mission_Inf) edge [DataLine] (mission_Inf-st_interpol)
(mission_Inf) edge [DataLine] (mission_Inf-grad_interpol)
(st_interpol) edge [DataLine] (st_interpol-st_fuel_burn)
(st_interpol) edge [DataLine] (st_interpol-grad_fuel_burn)
(grad_interpol) edge [DataLine] (grad_interpol-grad_fuel_burn)
(st_fuel_burn) edge [DataLine] (right_output_st_fuel_burn)
(grad_fuel_burn) edge [DataLine] (right_output_grad_fuel_burn)
% Vertical edges
(mission_Inf-st_interpol) edge [DataLine] (st_interpol)
(mission_Inf-grad_interpol) edge [DataLine] (grad_interpol)
(st_interpol-st_fuel_burn) edge [DataLine] (st_fuel_burn)
(st_interpol-grad_fuel_burn) edge [DataLine] (grad_fuel_burn)
(grad_interpol-grad_fuel_burn) edge [DataLine] (grad_fuel_burn)
(mission_Inf) edge [DataLine] (output_mission_Inf);
\end{pgfonlayer}

\end{tikzpicture}

    }
    \caption{Second workflow overview with some more details}
    \label{fig_3_Workflow}
\end{sidewaysfigure}
With these 
inputs \emph{missioninformer} generates interpolation models for states and 
for gradients. In order to get the mass of cruise segment fuel,  
its equation is solved. This equation requires the state 
surrogate models to be called. In fact, the solving of 
state cruise segment fuel equation is numerical and iterative and thus requires 
multiple function calls. In other words,  
solving the cruise segment fuel equation once, invokes 
the surrogate models multiple times. The gradients of 
the cruise segment fuel mass needs the state and the gradient 
interpolation models and also invokes both models 
multiple times for one solution. Contrary to the 
generation of the interpolation models, the invocation of 
the already calculated interpolation model is not 
computationally costly. As mentioned, a deeper insight into 
the whole process is provided in 
chapter \ref{chap_Methodlogy}. Once the state and gradients 
of the cruise segment fuel mass are calculated they are written to 
hard disk as output.


    




% ---------------- Task 2  ------------------------------

\chapter{Methodology}
\label{chap_Methodlogy}
In this chapter the employed equations, iterations and 
surrogate models for 
obtaining the state and gradients of the burnt fuel and their theoretical 
background shall be given. Additionally, the 
way of exploring methods, the reasons for 
having chosen the methods and the proceeding to obtain solutions
shall be explained. Furthermore, some workaround which was 
required will be mentioned, e.g. in the upcoming section, 
where a small tool was developed in order to get 
the atmospherical relationships.

% =====================================================================
% ============= Atmospheric ===========================================
% =====================================================================
\section{Atmospheric conversions}
One term inside the cruise fuel burn equation
for the computation of the cruise segment fuel
is the speed of sound ($a$), which 
is not given by the user. However, what the user provides is 
the altitude. By making use of one of the 3 common atmospheric models,
ICAO-, US- Standard Atmosphere or Norm Atmosphere DIN 5450/LN 9300, 
a relationship between pressure, density and altitude 
($p, \rho, h$), can be obtained. For our purposes, the
\emph{ICAO-Standard Atmosphere(ISA)} and 
\emph{Norm atmosphere DIN 5450/LN 9300} were used within an altitude range 
of 0-20 km (troposphere, lower stratosphere) 
and 20-32 km (upper stratosphere), respectively. With the mentioned 
atmospheric models, altitude can be obtained by providing 
pressure and density and vice versa. The \emph{missioninformer} is
capable of both and by following the equation \eqref{eq_1_Speed_Sound} 
the speed of sound ($a$) can be calculated. The \emph{heat capacity ratio} is chosen 
to be $\kappa = 1.4$. The reviewed process is 
depicted in figure \ref{fig_3_Atmospheric}

\begin{equation}
    a = \sqrt{\kappa \frac{p}{\rho}}
    \label{eq_1_Speed_Sound}
\end{equation}

\begin{figure}[!h]
    \def\svgwidth{\linewidth}
    \input{2_Figures/2_Task/1_atmospheric.pdf_tex}
    \caption{Atmospheric model and calculation of the speed of sound $a$}
    \label{fig_3_Atmospheric}
\end{figure}


%--------------- Von hier aus erneut lesen ---------------------
 
% =====================================================================
% ============= Cruise and Mission Fuel ===============================
% =====================================================================
\section{Cruise and mission fuel mass}
\label{sec_Cruise_Mission_Fuel}
In this section the derivation of the fuel burn equation 
for the cruise segment and the applied solution method shall be 
explained. Since the \emph{missioninformer} considers the whole 
mission it will also be elaborated how the 
fuel masses for the remaining flight segments are 
obtained.\newline 

The derivation of the cruise segment fuel burn equation, which 
takes the form of an Ordinary Differential Equation (ODE) was provided by 
Ilic in an \emph{DLR}-intern report 
\cite{ilic_goal_2013}. For its derivation a constant altitude 
()$h = const.$) is assumed. The cruise fuel burnt 
mass is the mass of fuel, which is required for the airplane in order 
to reach or fly a given range. The ODE is given in equation \eqref{eq_2_ODE_Orig}, 
where $m_{fe}$ is the burnt fuel mass, called fuel expended accordingly 
to the original 
report \cite{ilic_goal_2013}. The derivation is applied with respect to 
the range $ds$, $g,a,Ma,TSFC$ are denoted as the gravitational constant, 
the speed of sound, the Mach number, and $TSFC$ thrust specific fuel 
consumption, respectively. 
\begin{equation}
    \label{eq_2_ODE_Orig}
        \frac{dm_{fe}}{ds} = \frac{g}{a} \: \frac{1}{Ma}\;
        TSFC\, (m_s - m_{fe}) \; \frac{\frac{C_L}{C_D}\,tan(\theta)+1}
        {\frac{C_L}{C_D}\,cos(AoA) + sin(AoA)}
\end{equation}

The fraction $\frac{C_L}{C_D} = \frac{L}{D} = LoD$ is called aerodynamical 
performance or glide ratio. $AoA, \theta, m_s$ are denoted as the 
angle of attack, flight path angle and cruise starting mass. The presented 
equation is only one possible equation formulation, which can be employed to find the 
cruise fuel burnt mass. In order to 
understand why this is so, its derivation shall be explained briefly. In 
his report \cite{ilic_goal_2013}, the author starts with the equilibrium equations 
for steady flight. By inserting them into each other and using 
equation transformations Ilic \cite{ilic_goal_2013} comes to 
an expression for the thrust as: 
$ T = f(m,g,LoD,AoA, \theta)$. At this stage, the change 
of the fuel mass with the distance traveled can be 
expressed in the following two ways: 

\begin{equation}
    \label{eq_3_}
    \frac{dm_{fr}}{ds} = - \frac{TSFC}{Ma\, a \, cos(\theta) }\, T
\end{equation}

\begin{equation}
    \label{eq_4}
    \frac{dm_{fe}}{ds} = \frac{TSFC}{Ma\, a \, cos(\theta) } \,T
\end{equation}

Because the present total airplane mass m 
in the thrust expression $ T = f(m,g,LoD,AoA, \theta)$ can be expressed in
two ways, two equations (ODEs) are possible 
as outcome to describe the cruise fuel mass burnt. One is over the 
remaining fuel mass $m_{fr}$

\begin{equation}
    \label{eq_5}
    m = m_e + m_{fr}  ,
\end{equation}
where $m_e$ is the end mass or the mass right before landing. 
The second ODE is obtained by using the expended fuel mass as: 
\begin{equation}
    \label{eq_6_}
    m = m_s - m_{fe}  ,
\end{equation}
where $m_s$ is the cruise start mass or the mass just 
before cruise, after climb and acceleration. 
According to common field specific knowledge, masses in aerospace 
can be further 
distributed into detailed definitions, e.g. the takeoff mass  
can be further split up in operating empty weight, 
payload and fuel weight. The operating empty weight itself 
can be split up into airframe structure, propulsion 
group, airframe services and equipment, fixes equipment, 
removable equipment, standard items, standard item 
variation and operational items \cite{torenbeek_synthesis_1982}.
Depending on the field of research and the desired 
level on accuracy the 
mass division can be continued. However, the start mass 
$m_s$ from equation \eqref{eq_6_} contains the fuel weight 
for all segments which are followed after climb and
acceleration, cruise till taxi to parking.
During the flight this fuel mass 
is going to be reduced. With this background the two 
upcoming equations can be looked at:

\begin{equation}
    \label{eq_7_}
    \frac{dm_{fr}}{ds} = - \frac{g}{a} \, \frac{1}{Ma}\,TSFC \,(m_e + m_{fr})
    \, \frac{\frac{C_L}{C_D}\,tan(\theta) + 1}{
        \frac{C_L}{C_D} \, cos(AoA) + sin(AoA)}
\end{equation}

\begin{equation}
    \label{eq_8_}
    \frac{dm_{fe}}{ds} = \frac{g}{a} \, \frac{1}{Ma}\,TSFC \,(m_s - m_{fe})
    \, \frac{\frac{C_L}{C_D}\,tan(\theta) + 1}{
        \frac{C_L}{C_D} \, cos(AoA) + sin(AoA)}
\end{equation}

The difference between both comes through their derivation and 
thus the thought process or undertaken assumptions. 
Equation \eqref{eq_7_} works with remaining fuel, 
the end mass $m_{e}$ is known, 
the fuel is calculated and with these 
the start mass $m_{s}$ is obtained, see 
equation \eqref{eq_5}. This version lets 
the start mass be variable and requires 
the end mass as input. For equation 
\eqref{eq_8_} the start mass $m_{s}$ is 
passed as an input argument, the expended or required 
fuel for given cruise range is calculated. \newline 

Comparing 
the solution methods for the ODEs \eqref{eq_7_} 
and \eqref{eq_8_}, the first is integrated backwards
from cruise range end to zero.
In other words, the initial condition is given 
with the assumption that the remaining fuel 
at the end of the range equals zero $m_{fr}(R=R_{cr,max}) =0$, 
where $R$ is denoted as the range. As mentioned in 
section \ref{sec_1_State}, \emph{pyMission} follows 
this approach for solving the ODE.
In contrast, for defining the initial condition 
for equation \eqref{eq_8_} the fuel 
mass at the cruise start or cruise range zero 
is provided. 
It can be understood, that before the 
cruise even starts, no fuel in the cruise 
section is burnt. Note, the presented ODEs 
are only valid for a cruise segment. Therefore, 
just at the start of the cruise segment, neglecting 
other flight segments for this consideration, no 
fuel within the cruise segment is burnt.
Thus, the initial condition is formulated as 
equation \eqref{eq_8_Inital_Condit}. In this way, the ODE is 
integrated forward from 0 to given range.\newline

\begin{equation}
    \label{eq_8_Inital_Condit}
    m_{fe}(R_{cr}=0) = 0
\end{equation}

For this work the viewpoint of equation \ref{eq_8_} 
considering  
fuel expended or burnt is going 
to be applied in order to derive the upcoming 
solutions. Furthermore, the flight path angle is 
set to zero 
$\theta = 0$, see equation \eqref{eq_9}. Additionally, with the introduced 
relationship $LoD = \frac{C_L}{C_D}$ the final 
ODE can be formulated as equation \eqref{eq_10} \newline

\begin{equation}
    \label{eq_9}  
    \frac{dm_{fe}}{ds} = \frac{g}{a} \, \frac{1}{Ma}\,TSFC \,(m_s - m_{fe})
    \, \frac{1}{
        \frac{C_L}{C_D} \, cos(AoA) + sin(AoA)}  
\end{equation}


\begin{equation}
    \label{eq_10}  
    \frac{dm_{fe}}{ds} = \frac{g}{a} \, \frac{1}{Ma}\,TSFC \,(m_s - m_{fe})
    \, \frac{1}{
        LoD \, cos(AoA) + sin(AoA)}  
\end{equation}

Before explaining the methodology for solving the chosen ODE 
\eqref{eq_10}, the inclusion of the remaining flight segments shall 
be explained. In \cite{roskam_airplane_1986} fuel 
fractions, which are the ratio of the aircraft total 
weight at the end of a flight segment to the weight at the start 
of the same segment,
were provided. The employed fuel fractions 
for the associated flight segment are presented in the 
table \ref{tab_1}. The obvious advantage of applying 
empirical based fractions instead of solving physical based equations 
for flight segments other than cruise, is the 
absence of time and thus money consuming modeling
and solving of 
physical based equations and as consequence a 
fast as well as easy implementation.
Also, since only a multiplication is performed, the output 
in \emph{Python} is obtained within more than a feasible time. The 
disadvantage however, is a loss of accuracy. Given the fact that 
most fuel is burnt during the cruise segment, this seems 
a valid approach. For understanding, why the reserve fuel 
fraction is set to $0.05$, consider the 
equation \eqref{eq_56} in the
subsection \ref{subsec_Anly_Attempt}.

\begin{table}[!h]
    \centering
    \begin{tabular}{|l|c|c|}
        \hline
        Flight segment & Fuel fraction\\
        \hline
        engine start & 0.99 \\
        taxi to runway & 0.99 \\
        takeoff & 0.995 \\
        climb and acceleration & 0.98 \\
        descent &  0.99 \\
        landing &  0.992 \\
        taxi to parking &  0.99\\
        reserve &  0.05\\
        \hline
    \end{tabular}
    \caption{Fuel fractions for non cruise flight segments }
	\label{tab_1}
\end{table}





% =====================================================================
% ============= Solve ODE - Anly ======================================
% =====================================================================
\section{Method for solving the ODE}
\label{sec_Solve_ODE}
The chosen ODE \eqref{eq_10} can be solved numerically 
with different kind of solvers. However, in order to 
have a possibility to validate the
quality of the numerical 
solutions, some simplifications shall be introduced. 
With these, it is possible to derive an 
analytical solution. The variables 
$LoD, AoA, TSFC, m_{fe}$ are all functions 
of the present total mass, which is shown in 
equation \eqref{eq_13}. Since the start 
point for the ODE to be valid is at cruise, the 
current total mass is the cruise fuel starting 
mass $m = m_s$. However, itself 
has a dependency on the expended fuel mass,
which will be elaborated in 
section \ref{sec_Fuel_Mass_iter} and makes 
an analytical explicit solution hard to be derived. Additionally, no 
equations for $LoD(m_s), \,AoA(m_s), \,TSFC(m_s)$, which 
would describe a relationship between the variable 
and the total present mass, are 
given and therefore formulating an analytical solution becomes 
impossible.\newline

\begin{equation}
    \label{eq_13}  
    \frac{dm_{fe(m)}}{ds} = \frac{g}{a} \, \frac{1}{Ma}\,TSFC(m) \,(m_s + m_{fe}(m))
    \, \frac{1}{
        LoD(m) \, cos(AoA(m)) + sin(AoA(m))}  
\end{equation}

At this point the mentioned 
simplifications for an analytical solution 
can be mentioned. Assuming $LoD, AoA, TSFC$
to be constant allows to find an analytical
solution, as is presented in the upcoming equations. Consider the 
equation \eqref{eq_11}, where the 
A and B are constants and x is the 
variable for which the solution it is 
pursued.


\begin{equation}
    \label{eq_11}
    \frac{dx}{ds} + A\,x = B
\end{equation}

The solution to this problem \eqref{eq_11} can be 
obtained by using the integrating factor 
given in \eqref{eq_12}, where $C$ is a new 
constant. By replacing $x = m_{fe}$ equation 
\eqref{eq_14} is obtained. 
\begin{equation}
    \label{eq_12}
    x = \frac{B}{A} + C\, e^{-A\, s}
\end{equation}
 
\begin{equation}
    \label{eq_14}
    m_{fe} = \frac{B}{A} + C\, e^{-A\, s}
\end{equation}

It is understood that, the consumed fuel mass 
at the start or at cruise range zero 
has to be zero $m_{fe}(R_{cr} =0) = 0$, which 
is the initial condition to the equation \eqref{eq_14}. 
Inserting this observation leads to equation 
\eqref{eq_15}

\begin{equation}
    \label{eq_15}
    \begin{aligned}
        m_{fe}(s = R_{cr} =0) = 0 =  \frac{B}{A} + C\\
        \Rightarrow  C = - \frac{B}{A}
    \end{aligned}
\end{equation}

Defining $A$ and $B$ as equations \eqref{eq_15_A} and 
\eqref{eq_15_B}, respectively
and incorporating equation \eqref{eq_15}, the 
analytical solution for the cruise fuel burn can be 
written as equation \eqref{eq_16}. 

\begin{equation}
    \label{eq_15_A}
    A = \frac{g}{a} \, \frac{1}{Ma}\,TSFC \, \frac{1}{
        LoD \, cos(AoA) + sin(AoA)}
\end{equation}

\begin{equation}
    \label{eq_15_B}
    B = A \, m_s
\end{equation}



\begin{equation}
    \label{eq_16}
    \begin{aligned}
        m_{fe}(s = R_{cr} =0) = \frac{B}{A} + \frac{-B}{A}\, e^{-A\,s}\\
        \Leftrightarrow  m_{fe} = \frac{B}{A} (1-e^{-A\,s})\\
        \Leftrightarrow  m_{fe} = \frac{B}{A} (1-e^{-A\,s})\\
        \Leftrightarrow  m_{fe} = \frac{A \, m_s}{A} (1-e^{-A\,s})\\
        \Rightarrow m_{fe}(s = R_{cr}) =m_s \left(1-e^{\frac{-g}{a} \, 
            \frac{1}{Ma} \, TSFC \, 
            \frac{1}{LoD \,cos(AoA) + sin(AoA)} \, s} \right)
    \end{aligned}
\end{equation}

This analytical expression in
equation \eqref{eq_16} is compared with the output of 
different numerical solvers of Pythons 
open source library \emph{SciPy}. The 
different numerical ODE solvers are within the  
method \emph{solving initial value problem}  
\emph{(solve\_ivp)} and each solver comes
with several 
options, which are 
documented in \cite{noauthor_scipys_2021}. 
The results and the comparison 
is given in section \ref{sec_Solve_ODE_Methods}. However,
the conclusion of this investigation is that,  
\emph{SciPy}'s Runge Kutta of 
order 5 (\emph{RK45}) can be employed without any 
noticeable loss of accuracy.
\emph{RK45} assumes an accuracy of the fourth-order
method, but steps are taken using the 
fifth-order accurate formula. The occurring deviations to 
the analytically obtained solution \eqref{eq_16} are less 
than milligrams, which exhibits a sufficient 
level of 
accuracy.\newline


\emph{SciPy} allows the user to define the number 
and the location of steps. However, having compared 
its default option with less than $20$ steps
to a ridiculous high number 
of $1460597$ steps, no noticeable deviation 
could be observed. Fewer steps at which calculations 
happen results into a faster execution time. This 
becomes even more clear when recalling that
each step invokes surrogate  evaluations. In case 
of $20$ used steps undertaken, in order to 
solve the cruise fuel consumption ODE, for 
each step in this iterative process, the 
corresponding values 
for $LoD, AoA, TSFC$ with the
input set of $Ma,h, mass$ are necessary.
In simpler terms, each surrogate model
obtain an input set of $Ma,h, mass$ and 
provides the corresponding output for 
$LoD, AoA, TSFC$, which are predictions, thus 
$\tilde{LoD}, \tilde{AoA}, \tilde{TSFC}$. Since, this has to
done be for each step, $20*3= 60$ surrogate 
model invocations 
for solving the cruise fuel burn equation 
once, are required. Therefore, the low number of steps  
and because of the sufficient accuracy of the solution
with \emph{SciPy}'s default settings for \emph{RK45} are chosen 
to solve the cruise fuel burn mass. \newline


As a final recap, the cruise fuel burn 
equation cannot be solved analytically due to explained reasons 
and thus is solved with an outstanding accuracy employing 
\emph{SciPy}'s numerical ODE solver \emph{RK45} within the 
\emph{solve\_ivp} environment applying the default options.




\section{Mission fuel mass iteration}
\label{sec_Fuel_Mass_iter}

For solving the cruise fuel burn equation \eqref{eq_10} 
the present total mass of the airplane $m_s$, which 
is the starting mass for the cruise segment, is required.
However, the cruise starting mass $m_s$ is not known beforehand.
It depends on the fuel mass of the remaining segments fuel masses. 
These are the fuel 
weights of descent, landing and taxi to parking. As explained 
in section \ref{sec_Cruise_Mission_Fuel}, only the 
cruise fuel weight $m_{fe}$ for
the given mission or missions is calculated by solving the ODE 
\eqref{eq_10}. In this section $m_{fe}$ is denoted as $m_{cr}$
in order to highlight that it is only the fuel mass for 
the cruise segment. The required fuel masses for the remaining
flight segments are obtained through empirical fuel fractions 
from table \ref{tab_1}. The process for computing the total fuel
mission mass 
$m_f$, the cruise fuel $m_{cr}$ and the resulting 
cruise starting mass $m_s$  can be viewed 
in the figure \ref{fig_4_Mass_Workflow}.\newline 
%
%
\begin{figure} [!h]
    \resizebox{\textwidth}{!}{
    
%%% Preamble Requirements %%%
% \usepackage{geometry}
% \usepackage{amsfonts}
% \usepackage{amsmath}
% \usepackage{amssymb}
% \usepackage{tikz}

% Optional packages such as sfmath set through python interface
% \usepackage{sfmath}

% \usetikzlibrary{arrows,chains,positioning,scopes,shapes.geometric,shapes.misc,shadows}

%%% End Preamble Requirements %%%

\input{/home/jav/Progs/Virt_Env/writing/lib/python3.12/site-packages/pyxdsm/diagram_styles.tex}
\begin{tikzpicture}

\matrix[MatrixSetup]{
%Row 0
\node [DataIO] (output_ode_Solv) {$\begin{array}{c}\text{initial starting and mission fuel mass } ($$m_{s,0}, m_{f,0})$$\end{array}$};&
&
&
\\
%Row 1
\node [Function] (ode_Solv) {$\begin{array}{c}\text{ODE solver}\end{array}$};&
\node [DataInter] (ode_Solv-mass_Calc) {$\begin{array}{c}$$m_{cr}$$\end{array}$};&
&
\\
%Row 2
\node [DataInter] (mass_Calc-ode_Solv) {$\begin{array}{c}\text{new starting and mission fuel mass }($$m_{s,n}, m_{f,n})\end{array}$};&
\node [Function] (mass_Calc) {$\begin{array}{c}\text{mass recalculation}\end{array}$};&
&
\node [DataIO] (right_output_mass_Calc) {$m_{s}*, m_{cr}*, m_{f}*$};\\
%Row 3
&
&
&
\\
};

% XDSM process chains


\begin{pgfonlayer}{data}
\path
% Horizontal edges
(ode_Solv) edge [DataLine] (ode_Solv-mass_Calc)
(mass_Calc) edge [DataLine] (mass_Calc-ode_Solv)
(mass_Calc) edge [DataLine] (right_output_mass_Calc)
% Vertical edges
(ode_Solv-mass_Calc) edge [DataLine] (mass_Calc)
(mass_Calc-ode_Solv) edge [DataLine] (ode_Solv)
(ode_Solv) edge [DataLine] (output_ode_Solv);
\end{pgfonlayer}

\end{tikzpicture}

    }
    \caption{Initial mass and mass fix-point iteration workflow}
    \label{fig_4_Mass_Workflow}
\end{figure}
%
%

Since the cruise starting mass $m_s$ is not known 
before knowing the cruise and total fuel masses ($m_{cr}, m_{f}$), 
the initial cruise starting and total fuel mass ($m_{s,0}, m_{f,0}$)
are estimated based on the equations \eqref{eq_17} to 
\eqref{eq_19}. The variables for the equations 
$m_{takeoff,max},\, f_{zero},\, f_{eng,start}$
$m_{oe},\, m_{payload}$ are denoted 
as maximal takeoff, zero fuel, 
engine start, operating empty weight and design
payload, respectively. The fuel fractions 
$fr_1 \,, fr_2\,,fr_3 \,, fr_4$ are for the following 
flight segments, after engine start, taxi to runway, takeoff 
and climb and acceleration, respectively.

\begin{equation}
    \label{eq_17}
    f_{zero} = m_{oe} + m_{payload}
\end{equation}

\begin{equation}
    \label{eq_18}
    m_{f,0} = m_{takeoff,max} - f_{zero}
\end{equation}


\begin{equation}
    \label{eq_19}
    m_{s,0} = \left(f_{zero} + f_{eng,start}\right) \, fr_1 \, fr_2\,fr_3 \, fr_4
\end{equation}



By proving $m_{s,0}$ to the cruise fuel ODE 
\eqref{eq_10} the corresponding cruise fuel $m_{cr}$ is 
obtained, which is passed to the block \emph{mass recalculation}.
Here the new cruise starting and total fuel masses 
($m_{s,n}, m_{f,n}$) are calculated. For computing 
former ($m_{s,n}$) the equation \eqref{eq_19} is still 
valid, however, computing the new total fuel mass $m_{f,n}$
two different possibilities were found. One by using all 
known mass relationships straight forward and solving $m_{f,n}$ 
numerically, e.g. the Newtons method. The second 
method was obtained by inserting the 
mass equations into each and 
performing transformations. With that 
an analytical solution was obtained. Here, only 
the derivation of the analytical solution shall 
be explained.\newline 

The demand for  
$m_{f,n}$ is stated in equation \eqref{eq_20}, where 
$m_{after,cr}$ is the mass after
the cruise segment and $ fr_{cr}$ is introduced as auxiliary
fuel fraction 
for the cruise segment and is given 
in equation \eqref{eq_20_Cruise_Fraction}. In words, the mass after 
the cruise segment divided by the mass after acceleration 
and climb must be equal to the cruise fuel fraction.


\begin{equation}
    \label{eq_20_Cruise_Fraction}
    fr_{cr} =  \frac{m_s - f_{cr}}{m_s}
\end{equation}

\begin{equation}
    \label{eq_20}
    \frac{m_{after,cr}}{m_{s} }- fr_{cr} \overset{!}{=} 0
\end{equation}

The mass after the cruise segment $m_{after,cr}$
 can be written in 
the form of equation \eqref{eq_21}, where 
$fr_{reser}$ is the known fuel reserve fraction, 
provided in table 
\ref{tab_1}. The fuel fractions 
$fr_5\,, fr_6\, , fr_7$ stand for taxi to parking,
landing and descent, respectively and are also 
given in table \ref{tab_1}. The unknown
in this equation is the new total fuel mass 
$m_{f,n}$. Applying insertions and mathematical 
transformations with the equations \eqref{eq_19} to 
\eqref{eq_21}, $m_{f,n}$ can be written as 
equation \eqref{eq_23}, where $fa$ as an 
auxiliary term is given in equation \eqref{eq_22}.


\begin{equation}
    \label{eq_21}
    m_{after,cr} = \frac{f_{zero} + m_{f,n} \, fr_{reser}}
            {fr_5\, fr_6\, fr_7} = 0
\end{equation}

\begin{equation}
    \label{eq_22}
    fa = f_{zero} \, fr_{cr} \, fr_1 \, fr_2\,fr_3 \, fr_4 \, 
    fr_5\, fr_6\, fr_7
\end{equation}

\begin{equation}
    \label{eq_23}
    m_{f,n} = \frac{fa - f_{zero}}{f_{reser} -fr_{cr} 
    \,fr_1 \, fr_2\,fr_3 \, fr_4 \, 
    fr_5\, fr_6\, fr_7 }
\end{equation}


% ==================================================
% =========== RECAP ===============================
% ==================================================

To recap, after an initial guess for
starting cruise and total fuel mass 
$m_{s,0}, m_{f,0}$, they are passed 
to the ODE solver, which outputs 
its corresponding cruise fuel weight $m_{cr}$.
With the equation \eqref{eq_20_Cruise_Fraction} 
an auxiliary fuel fraction for the cruise 
segment is calculated and used in the 
\emph{mass recalculation} box, depicted
 in figure \ref{fig_4_Mass_Workflow}.
Here by employing the equations \eqref{eq_19}
and \eqref{eq_23} the new cruse starting 
and total fuel mass $m_{s,n}, m_{f,n}$ are 
obtained, respectively.
This process is repeated as long 
the convergence criteria 
$\Delta =  m_{f,n} -  m_{f,old} \leq \epsilon$ is met, where 
$\epsilon$ is a tolerance parameter. 
The relative and absolute tolerance parameter are 
set to be $5\mathrm{e}{-5}$ \cite{noauthor_numpys_2021}.
Such an iterative method is also called \emph{fix-point iteration}.
Since the described workflow is iterative, the resulting 
change and their interdependencies  (mass snowball effect)
are taken into account.
In case a convergence is achieved, the final 
values for the cruise starting, cruise fuel and 
total fuel mass \ref{fig_4_Mass_Workflow}
are found ($m_{s}*, m_{cr}*, m_f*$).\newline

As mentioned above, the problem \eqref{eq_20} can 
be solved numerically or analytically. In order 
to verify the analytical solution it 
has been compared successfully with the 
numerical Newtons' method. The presented 
workflow defining the 
auxiliary term $fr_{cr}$ 
\eqref{eq_20_Cruise_Fraction} can be 
further simplified by directly inserting 
$fr_{cr}$ into the equations. This 
results into equation \eqref{eq_63} 
and its derivation with respect to 
the shape parameter is explained 
in more details in section \ref{sec_Shape_Gradients}. 
The results of 
this direct (without auxiliary term) 
approach differed marginally 
from the previous auxiliary method.
Therefore, \emph{Matlab} and \emph{SymPy}
(open source 
Python library) were used in order 
to verify the undertaken 
mathematical transformations. The 
symbolic results confirmed the handcrafted 
analytical solution.\newline


One interesting finding could be observed. 
Consider the
figure \ref{fig_4_Mass_Workflow}, which shows 
the reoccurring or iterative process,  
thus the fix-point iteration.
Depending on applying
the auxiliary or the direct version, different 
numbers of iterations were needed to 
get to the final solutions for 
$m_{s}*, m_{cr}*, m_f*$. As mentioned, also the  
end values for $m_{s}*, m_{cr}*, m_f*$ 
would differ, however only around the 10th 
decimal place. By changing the missions 
definitions, it could be observed that in most cases 
the direct version required more iterations to 
converge. In some cases, more than two
to four times more iterations 
were required. Thus, \emph{missioninformer's} default 
method to solve equation \eqref{eq_20} is 
by employing the auxiliary term $fr_{cr}$.\newline





% ==================================================
% =========== Check violation ======================
% ==================================================
Finally, the \emph{missioninformer's} \emph{constraint 
violation check} shall be 
elaborated. Since the user is completely free 
in defining the desired mission or missions, it is 
possible for an unpracticed user to define an
unrealistic high range or payload. Also, 
possible is that, e.g. the predicted
$\tilde{LoD}$ with the preliminary aircraft synthesis
software turned out to be too optimistic and at the 
detailed Reynold-Averaged-Navier-Stokes-Equations-
based (RANS) aerodynamics analysis. In such cases 
the required fuel masses or other weights can exceed 
the mass constraints set by the top level aircraft 
requirements (TLAR) and the preliminary design.
 Therefore, verifying whether 
given constrains are violated is necessary. The 
workflow with the \emph{constraint violation check} is 
depicted in figure \ref{fig_5_Constraint_Violation}.
As it can be observed, it is performed after each mass 
recalculation. 
%
% ==============================================
% ========== Constraint Viol Workfflow =========
% ==============================================
\begin{figure} [!h]
    \hspace*{-4cm} 
    \resizebox{1.2\textwidth}{!}{
    
%%% Preamble Requirements %%%
% \usepackage{geometry}
% \usepackage{amsfonts}
% \usepackage{amsmath}
% \usepackage{amssymb}
% \usepackage{tikz}

% Optional packages such as sfmath set through python interface
% \usepackage{sfmath}

% \usetikzlibrary{arrows,chains,positioning,scopes,shapes.geometric,shapes.misc,shadows}

%%% End Preamble Requirements %%%

\input{/home/jav/Progs/Virt_Env/writing/lib/python3.12/site-packages/pyxdsm/diagram_styles.tex}
\begin{tikzpicture}

\matrix[MatrixSetup]{
%Row 0
\node [DataIO] (output_ode_Solv) {$\begin{array}{c}\text{initial starting} \\ \text{and mission fuel } \\ \text{mass } ($$m_{s,0}, m_{f,0})$$\end{array}$};&
&
\node [DataIO] (output_constr_Viol) {$\begin{array}{c}\text{mission input} \\ \text{file}\end{array}$};&
&
\\
%Row 1
\node [Function] (ode_Solv) {$\begin{array}{c}\text{ODE solver}\end{array}$};&
\node [DataInter] (ode_Solv-mass_Calc) {$\begin{array}{c}$$m_{cr}$$\end{array}$};&
&
&
\\
%Row 2
\node [DataInter] (mass_Calc-ode_Solv) {$\begin{array}{c}\text{new starting and} \\ \text{mission fuel mass } \\ (m_{s,n}, m_{f,n})\end{array}$};&
\node [Function] (mass_Calc) {$\begin{array}{c}\text{mass recalculation}\end{array}$};&
\node [DataInter] (mass_Calc-constr_Viol) {$\begin{array}{c}m_{s,n}, m_{f,n}\end{array}$};&
&
\node [DataIO] (right_output_mass_Calc) {$\begin{array}{c}m_{s}*, m_{cr}*, \\  m_{f}*\end{array}$};\\
%Row 3
&
&
\node [Function] (constr_Viol) {$\begin{array}{c}\text{check constraint} \\  \text{violation}\end{array}$};&
&
\\
%Row 4
&
&
&
&
\\
};

% XDSM process chains


\begin{pgfonlayer}{data}
\path
% Horizontal edges
(ode_Solv) edge [DataLine] (ode_Solv-mass_Calc)
(mass_Calc) edge [DataLine] (mass_Calc-ode_Solv)
(mass_Calc) edge [DataLine] (mass_Calc-constr_Viol)
(mass_Calc) edge [DataLine] (right_output_mass_Calc)
% Vertical edges
(ode_Solv-mass_Calc) edge [DataLine] (mass_Calc)
(mass_Calc-ode_Solv) edge [DataLine] (ode_Solv)
(mass_Calc-constr_Viol) edge [DataLine] (constr_Viol)
(ode_Solv) edge [DataLine] (output_ode_Solv)
(constr_Viol) edge [DataLine] (output_constr_Viol);
\end{pgfonlayer}

\end{tikzpicture}

    }
    \caption{Second workflow with some more details}
    \label{fig_5_Constraint_Violation}
\end{figure}
%
%
In case of a found constraint 
violation, a colored warning statement will be given 
to highlight an eventual inconsistency with 
the possible incorrectness of 
the mission inputs definition. The following 
checks are made, where 
$m_{takeoff,max}, \,m_{eng,start},$ $m_{land,max}, \, m_{aft,desc}, $
$m_{f,max}, \, m_{f}$
are denoted as the weights of 
maximal takeoff, engine start, maximal landing, after 
descent, maximal fuel and actual required total fuel, respectively.

\begin{equation}
    \label{eq_21_Check_viloation}
        \begin{aligned}
            m_{takeoff,max} <  m_{eng,start}\\
            m_{land,max} < m_{aft,desc}\\
            m_{f,max} < m_{f}
        \end{aligned}
\end{equation}

\section{Surrogate models}
\label{sec_Surrogates}
In very brief words, a surrogate model can be described
as an approximation model. In most cases 
it is used in order to pack computational heavy 
physical based calculations into a simpler 
and execution time friendlier calculation 
environment. Surrogate models are very popular in 
the field of optimization, because 
every optimization algorithm evaluates 
function values with variation of the 
design variables. The design variables 
are the parameters which are changed by the 
optimizing algorithm in order to 
fulfil the minimization or maximization task.
The design variables could consist of only one 
variable, e.g. the wing aspect ratio.
Another example for many design variables
is topology optimization, where 
the number of design variables 
is equivalent to 
the number of the discretized finite elements. For a
real world application, e.g. an aircraft or a car  
the number of finite elements can easily overreach 
millions. The 
optimization algorithm tries to 
minimize the objective function by changing 
the design variables and usually, each 
change demands a new calculation. With 
the variation of the design variables, 
the optimization algorithms collects 
outputs which are compared to each other 
in order to follow a steadily improving 
path. \newline

In case of 
a flow field computation, changing only one parameter, 
e.g. the angle of attack ($AoA$), obligates a new  
expensive computational task. Depending on the applied fidelity and the 
object (airplane, car, turbo machinery) for which the flow field 
is solved, it could acquire seconds till years (
Direct Numerical Simulation). Thus, one objective 
of surrogate models is enabling feasible computational times.
Additionally, surrogate models can be generated and invoked on 
none high performance computers.\newline 

Surrogate models are also known as metamodels or models of models.
Eldred er al. \cite{eldred_second-order_2004}
classified surrogate models 
into three categories: data-fits, reduced order models
 and hierarchical models. 
Reduced-order and hierarchical surrogate models can be 
further classified as physics-based approaches, due to the 
exploitation and simplification of 
governing equations \cite{ahmed_surrogate-based_2009}. Therefore,
these models are considered as \emph{intrusive} methods. 
Data-fit surrogate models, however, are assigned to 
the \emph{black-box} approach, where the outputs
are only based on the inputs, without necessarily 
knowing the underlying equations. Black-box approaches 
are non-intrusive methods and can be further categorized 
into regression and interpolation. The main difference between 
these are, that interpolation must match the training or 
reference data at the given reference points, whereas 
regression is not obligated to match the sample data at 
the reference locations. Two widely 
used interpolation surrogate models are Radial Basis Function (RBF)
and Kriging models. To generate and use 
the black-box intrusive surrogate models, sample data is required which 
for the \emph{missioninformer} is provided by the output of the DLR 
aerodynamic analysis and optimization workflow 
\emph{FSAerOpt} \cite{merle_high-fidelity_2019} involving 
\emph{DLR}'s flow solver \emph{TAU}.\newline

After having the input or training data,
 a surrogate model and its 
intern tuning parameter with 
which the model shall be constructed needs to be defined.
 The generation 
of the surrogate model (offline phase) can claim some noticeable time, 
however, still in feasible manner. Invoking the 
already build surrogate model (online phase) can 
be compared with regard to execution time to
solving a simple linear equation, which is 
satisfactorily feasible. The main disadvantage 
by applying surrogate models 
is the loss of accuracy compared 
to the underlying physics based equations or 
the measured experimental data. For 
assessing the quality of the surrogate model 
some methods are presented in \cite{meckesheimer_computationally_2002}.
However, in \emph{missioninformer} one of the most commonly 
used error measurement, the \emph{root mean square error} (RMSE),
is employed. It is given in equation \eqref{eq_24}, where 
$n, \,y_i, \, \tilde{y_i}$ are denoted as the number of 
validation points (training or sample data obtained 
from workflow of \emph{FSAerOpt} \cite{merle_high-fidelity_2019}), 
the actual value and the predicted value, respectively. \newline


\begin{equation}
    \label{eq_24}
    RMSE = \sqrt{\frac{1}{n} \sum_{i=1}^{n} (y_i - \tilde{y_i})^2}
\end{equation}

The \emph{missioninformer} involves two interpolation surrogate 
models, Kriging and RBF with different options, which are 
described in section \ref{subsec_Kriging_Surrogate} and 
\ref{subsec_RBF_Surrogate}. Both models will be tested
with respect to execution time, robustness and RMSE 
with different model specific options
in section \ref{sec_Diff_Surrogates}. Surrogate models 
within the \emph{missioninformer} are generated for 
$\tilde{LoD}, \tilde{AoA}, \tilde{TSFC}$
and their gradients with 
respect to shape parameter. The input set 
for the mentioned  desired output state and gradient 
variables consists of $Ma, h, m_s$. The state 
surrogate models are invoked for solving 
the equation \eqref{eq_10}, thus needed 
for the cruise fuel mass computation. For the rest of 
the flight segments no surrogate
models are required, since fuel fractions are 
exploited as explained in sections \ref{sec_Cruise_Mission_Fuel}, 
\ref{sec_Solve_ODE} and 
\ref{sec_Fuel_Mass_iter}. Solving the state fuel cruise mass 
invokes the three required surrogate models
to predict
$\tilde{LoD}, \tilde{AoA}, \tilde{TSFC}$  at each 
step undertaken by the chosen \emph{RK45} 
solver \cite{noauthor_scipys_2021}.
The total number 
of state surrogate model invocations for the 
state can be described as equation \eqref{eq_25}.
The number of the gradient surrogate model 
invocations further includes the number of the 
shape parameter. The required number of 
gradient surrogate models and their 
number 
of invocations are given in equations 
\eqref{eq_26} and \eqref{eq_28}, respectively. 

\begin{equation}
    \label{eq_25}
    \#_{invok,state} = 3\, \cdot\, steps_{RK45} 
\end{equation}

\begin{equation}
    \label{eq_26}
    \#_{gener,grad} = 3\, \cdot \,\#_{shape,param}
\end{equation}

\begin{equation}
    \label{eq_28}
    \begin{aligned}
        \#_{invok, grad} &= 3\, \cdot \,steps_{RK45} \, \cdot \, \#_{shape,param} \, 
        \cdot \, 2 \, \cdot \, \#_{fix,pkt,iter} \\
        &\Rightarrow 6\, \cdot \,steps_{RK45} \, \cdot \, \#_{shape,param} \, \cdot \, \#_{fix,pkt,iter}
    \end{aligned}
\end{equation}


In order to understand why equation \eqref{eq_28},
briefly 
missioninfomers implemented method to derive 
gradients shall be mentioned, which in 
detail is explained in subsection \ref{subsec_CDS}.
The gradients in \emph{missioninformer} are 
obtained by exploiting the Central Differencing 
Scheme, which basically adds a very small 
positive and negative perturbation to 
the current state of the present 
desired variable ($LoD, AoA, TSFC$), for which 
the gradient is desired and divides the result 
by two times the size of the perturbation.
Due to the results for both perturbed states, 
positive and negative 
added perturbation, the number 
of surrogate model invocation increases. In simpler terms, 
the surrogate model is invoked once for positive 
added permutation and once for negative added 
perturbation. The perturbation is called step-size. 
How and why \emph{missioninformers} default step-size 
has the value $1\mathrm{e}{-5}$ is described in 
subsection \ref{subsec_CDS} 
and  section
\ref{sec_Step_Size_Study}.\newline

Additionally, 
also the final values for the gradients 
are obtained after a fix-point iteration.
In cases of a barely solvable mission more than 
500 iterations were required, in some 
cases even no solution was obtainable and 
finally in cases where the missions were 
defined properly (not unrealistic high ranges 
or payloads) only $4$ fix-point 
iterations were required. As for the  
tested missions, a correlation could be read 
out. In cases where the constraint violation  
warning was given, more than $4$ fix-point 
iterations could be seen. Therefore, it 
is advised to observe the warning outputs.
Note, the number of the fix-point iterations is also 
depended on the chosen step-size. The number of 
$4$ fix-point iteration is met due to 
the well-fitted step-size of $1\mathrm{e}{-5}$.
The step-size can be changed inside 
the mission input file for each mission. However,
the default value in \emph{missioninformer} is set 
to $1\mathrm{e}{-5}$ and a 
change in its value is assumed to most likely 
have a negative impact on the number 
of fix-point iterations. The reasons for that 
will become clearer
by considering the convergence behavior 
depicted in the step-size study 
in section \ref{sec_Step_Size_Study}.\newline 


Assuming having $126$ shape parameter, $20$ required 
steps for the ODE's integration and $4$ fix-point
iterations, the gradient 
surrogate models are invoked $30240$ times.
Using directly CFD calculations, where one CFD result 
takes easily 3 hours using a 
reasonable amount of high performance 
computer resources.  


Finally, 
it shall be mentioned that the number of fix-point 
iterations is limited to $500$ iterations. In cases, 
where the missions are physically correctly defined 
and no constraint violation warning were observed,
not more than $5$ fix-point iterations were encountered. 
Therefore, it is regarded to be highly unlikely to require 
more than $500$ iterations. When the limit of $500$ iterations 
is reached, the user will be notified in the output 
about the abortion due to the exceeding of the  fix-point 
iteration number's limit. 


\subsection{Kriging models}
\label{subsec_Kriging_Surrogate}
Kriging for using as a surrogate model was originally developed 
for the field of geostatics by Daniel G. Krige, after which the 
method is named \cite{krige_statistical_1951}. The 
term \emph{Kriging} was shaped by Matheron \cite{matheron_principles_1963}, 
who was also the first to formulate it mathematically. Initially after 
its development, Kriging was used to model continuous 
defined functions based on measurement data. The foundation 
of exploiting Kriging models in the Design and Analysis of 
Computer Experiments (DACE) was first developed by 
Sacks et al. \cite{sacks_design_1989}, where 
the input points represent the spatial 
geographical coordinates. In Kriging models a 
deterministic response $y(x)$ is a realization of a stochastic
process $Y(x)$ \cite{sacks_design_1989} and is described in 
the following equation: 

\begin{equation}
    \label{eq_29}
    Y(x) = \sum_{k=1}^{N_f}f_k(x) \beta_k + Z(x) = f^T (x) \beta + Z(x)
\end{equation}

The first term is the global model component, where
$f(x) = [f_1(x), \, f_2(x),\, ....,\,
f_{N_f}(x)]^T$ 
is a vector of $N_f$ 
basis functions and $\beta = [\beta_1, \,\beta_2,\, ..., \,\beta_{N_f}]$
is a vector of the
unknown coefficients. The stochastic component $Z(x)$ is 
treated as the realization of a stationary Gaussian 
function with an expected value of zero ($E[Z(x)]=0$) 
and the covariance is given in equation \eqref{eq_30}, where 
$R$ denotes the correlation function with $R(0) = 1$.
Because of this, Kriging models can be used
as interpolation models 
and thus provide the exact prediction at the 
training or sample data. However, by applying 
regularization, which adds a constant to each 
control point, the interpolation turns into 
regression. The greater the distance 
of the input, for which the predicted 
output is desired is from the training sample data, 
the greater the error in the variance gets.
In other words, in Kriging models the data are assumed 
to be exact, but the function is a realization of a
Gaussian process \cite{viana_special_2014}. The second term 
is denoted as bias, a systematic departure from the linear model 
or localized deviation \cite{simpson_kriging_2001}.

\begin{equation}
    \label{eq_30}
    Cov\, [Z(x_i), \,Z(x_j)] = \sigma^2\, R(x_i, x_j)
\end{equation}

In \emph{1d} the stationary correlation function 
between 
any two points in the input space $y(x_i)$
and $y(x_j)$, depend only on the difference 
between these points ($\Delta x = x_i - x_j$).
For higher-dimensional problems, the correlation function 
in a Kriging model usually 
obeys the \emph{product correlation rule}, given in
equation \eqref{eq_31}, 
where $\theta = \theta^{(k)}, \, k= 1,...., N_d$ is denoted as the 
vector of correlation parameters. The notation $d_{ij}^{(k)}$ is the distance 
between two points in the $k^{th}$ dimension $(\, x_i^{(k)} 
- x_j ^{(k)\,})$. The correlation parameters, are called Kriging 
hyperparameters and can be found by using the 
Maximum Likelihood Estimation (MLE) approach. Large 
$\theta$ values represent a weak spatial correlation, whereas 
small values stand for a strong spatial correlation.
Since \emph{missioninformer} calls \emph{SMARTy} for its 
Kriging operations, different correlations 
functions can be used. However, within this research 
the Gaussian correlation function
is used.


\begin{equation}
    \label{eq_31}
    R_{ij}\,(\theta, d) = \prod_{k = 1}^{N_d} R\,(\theta ^{(k)}, \, 
    d_{ij}^{(k)})
\end{equation}


\subsection{RBF models}
\label{subsec_RBF_Surrogate}
RBF stands for \emph{Radial Basis Function} and is 
black-box surrogate model, which simulates complicated 
design landscapes using a weighted sum of 
simple functions as shown in equation \eqref{eq_32}.
Here $x_0,\, x_s,\, y_s$ are denoted as 
input space, samples location's and output's value, respectively.
The function $\psi (\,||x_0 -c_i||\,)$ is the kernel
function centered at $c_i$ and the norm $||\cdot||$ is 
the Euclidean distance. Usually, the training sample points are used as 
the center, therefore it can be stated: $c = x_s \;\wedge \; N_c = N_s$.
\begin{equation}
    \label{eq_32}
    \tilde{y}\,(x_0, x_s, y_s) = \Psi_{0}^{T} \, w =
    \sum_{i = 1}^{N_c} w_i \, \psi (\,||x_0 -c_i||\,)
\end{equation}

The vector of the unknown coefficients $w$, is obtained by 
solving the system of linear equations given in equation 
\eqref{eq_33}. The gram matrix $\Psi$ is defined as 
$\Psi_{ij} = \psi(|| \, x_{s_i} - x_{s_j}\, ||)$. In other 
words, the gram matrix, is the kernel function evaluated 
at the Euclidean distance between the $i^{th}$ and the 
$j^{th}$ samples. 

\begin{equation}
    \label{eq_33}
    \Psi \, w = y_s
\end{equation}


The \emph{missioninformer} invokes 
Pythons library \emph{SciPy} and \emph{SMARTy} for the RBF calculations. 
\emph{SciPy} enables the user to define $7$ different kernels, 
which are given in 
the equations \eqref{eq_34} to \eqref{eq_35}, 
by changing one input parameter 
\cite{noauthor_rbf_2021}. \emph{SMARTy} uses 
the Thin Plate Spline (TPS) kernel.
The variables $r, \epsilon$ are denoted 
as the Euclidean distance and an adjustable constant 
for Gaussian or multiquadtratic functions,
respectively. Only $6$ of the $7$ kernels were used,
since the quintic kernel took multiple 
hours to provide a solution. The $6$ kernels are used for a 
comparison to 
Kriging. The results of the comparison 
are shown in \ref{sec_Diff_Surrogates}


\begin{align}
    multiquadric &\;=\;  \sqrt{\left(\frac{r}{\epsilon}\right)^2 + 1}  \label{eq_34}\\
    inverse &\;=\; \frac{1}{ \sqrt{\left(\dfrac{r}{\epsilon}\right)^2 + 1}} \\
    gaussian &\;=\; exp \;[\;- \left(\frac{r}{\epsilon}\right)^2\;] \\
    linear &\;=\;  r \\
    cubic &\;=\;  r^3 \\
    quintic &\;=\;  r^5\\
    thin\, plate\, spline &\;=\;  r^2 \, ln\,(r) \label{eq_35}
\end{align}


\newpage

\subsection{Investigations on different surrogate models}
\label{subsec_INvestigate_Surro}

In this subsection, the procedure, which was undertaken to 
investigate on the two most widely used 
surrogate models RBF and Kriging shall be explained.
Note, the results are all presented in section 
\ref{sec_Diff_Surrogates}. In the beginning 
of this section an introduction to RBF 
and Kriging and some 
reasons for the necessity of a reliable 
surrogate model were given. 
However,
the main reason comes through the high dependency 
of the \emph{missioninformers} output, cruise fuel mass and their 
gradients. As explained in the same section,
the number of the invocation of the surrogate models 
is very high. \newline 

Only with a surrogate model, which 
replicates the underlying physical based 
behavior correctly, the results of \emph{missioninformer} 
can be considered correct as well.
Due to this 
heavy dominance $8$ different options for 
Gaussian Kriging and RBF with TPS
were chosen as the investigation parameter, 
see table \ref{tab_3}. 
For further reading, Gaussian Kriging will 
be referred to as normal Kriging or 
only Kriging and 
RBF with TPS as TPS.
The $8$ options 
for Kriging and TPS  were equivalent and are given in 
table \ref{tab_3}. The short names on left side 
of the table defines the option's shortcut and are important 
for interpreting
the plot results in section \ref{sec_Diff_Surrogates}.
The values $-1,0,1,2$ for Augmentation mean
no, constant, linear, and quadratic trend function, 
respectively. In case Regularization is set 
to True, a regularization constant is added
to each control point.\newline

\begin{table}[!htb]
    \centering
    \begin{tabular}{l c c }
        \multicolumn{1}{p{2cm}}{\textbf{Name}} &  \multicolumn{1}{p{2cm}}{\textbf{Augmen \newline tation}} 
            &  \multicolumn{1}{p{2cm}}{\textbf{Regulari \newline zation}}\\
            \hline
            A & $-1$ &False \\
            B & $0$ &False \\
            C & $+1$ &False \\
            D & $+2$ &False \\
            E & $-1$ &True \\
            F & $0$ &True \\
            G & $+1$ &True \\
            H & $+2$ &True \\
        \end{tabular}
        \caption{\emph{SMARTy} Kriging and TPS parameter used for investigation 
        surrogate quality}
        \label{tab_3}
\end{table} 

\FloatBarrier
The investigations were performed 
with two different missions as input and 
are reproduced in table \ref{tab_2}. The masses 
are given in kilogram and the 
cruise altitude $h$ and cruise range $R_{cr}$
in meters. For the central difference step-size the default value 
of $1\mathrm{e}{-5}$ was chosen.  
First the surrogate models Kriging, TPS and RBF 
were only tested for the state condition, meaning 
no gradients were calculated.

\begin{table}[!h]
    \centering
    \begin{tabular}{l| c c c c c c }
        \textbf{Mass} & \multicolumn{1}{c}{\textbf{Mission 1}} &&& 
        \multicolumn{1}{c}{\textbf{Mission 2}}\\
        \hline
        maximal takeoff & $245 \mathrm{e}{3}$ &&&      $245\mathrm{e}{3}$         &&\\
        maximal landing & $192.2 \mathrm{e}{3}$ &&&      $192.2\mathrm{e}{3}$         &&\\
        operating empty & $132.5\mathrm{e}{3}$ &&&      $132.5\mathrm{e}{3} $         &&\\
        manufactures empty & $119.2\mathrm{e}{3} $ &&&      $119.2\mathrm{e}{3} $         &&\\
        maximum zero fuel  & $180.5\mathrm{e}{3} $ &&&      $180.5\mathrm{e}{3} $         &&\\
        maximum fuel & $107.6\mathrm{e}{3}$ &&&      $107.6\mathrm{e}{3}$         &&\\
        design payload & $38.52\mathrm{e}{3}$ &&&      $33.60\mathrm{e}{3} $         &&\\
        maximum payload & $48.00\mathrm{e}{3}$ &&&      $48.00\mathrm{e}{3}  $         &&\\
        \hline
        &  &&&               &&\\
        \textbf{Flying parameters} &  &&&               && \\
        \hline 
        $Ma$ & $0.83$ &&&      $0.82$         &&\\
        $h$ & $10.668\mathrm{e}{3}$ &&&      $11.000\mathrm{e}{3}  $         &&\\
        $R_{cr}$ & $9186.0\mathrm{e}{3} $ &&&      $5185.6\mathrm{e}{3} $         &&\\
        \hline
        &  &&&               &&\\
        \textbf{Others} &  &&&               && \\
        \hline 
        step-size & $1\mathrm{e}{-5} $&&&      $1\mathrm{e}{-5}$         &&\\
        weight-factor & $0.7$ &&&      $0.3$         &&\\
        
    \end{tabular}
    \caption{Missions definitions for investigation on different surrogate models}
    \label{tab_2}
\end{table} 


\FloatBarrier
For \emph{SciPy}'s RBF $7$ different 
kernel functions, which 
are given in the equations \eqref{eq_34} -
\eqref{eq_35}, were tested with 
a \emph{SciPy}s \emph{smooth} value of $0.1$. 
In case of \emph{smooth} value greater zero, 
regularization is employed. Without 
it having defined to be greater zero, 
the generation of a surrogate model failed.
For seeing the used \emph{SciPy} RBF's kernel functions and 
interpreting the \emph{SciPy} RBF investigation results the table 
\ref{tab_4} is provided. Note, the kernel 
quintic was dropped due its high amount of 
calculation time.
%
\begin{table}[!h]
        \centering
        \begin{tabular}{l l}
            \multicolumn{1}{p{2cm}}{\textbf{Name}} &
            \multicolumn{1}{p{2cm}}{\textbf{Kernel}}\\
            \hline
            A & multiquadric  \\
            B &  inverse \\
            C &  gaussian\\
            D &  linear\\
            E & cubic\\
            F &  thin plate\\
        \end{tabular}
        \caption{Different \emph{SciPy} RBF kernels used for surrogate model 
        quality investigation }
        \label{tab_4}
\end{table} 
%
\FloatBarrier
Since in the state 
calculation only 3 surrogate models ($\tilde{LoD}, 
\tilde{AoA},  \tilde{TSFC}$) are required to be 
generated, the research is conducted much faster 
compared to the gradient version. 
The weight 
before cruise or the starting mass for the cruise 
segment $m_s$, the fuel weight only 
for the cruise segment $m_{cr}$, the
total fuel mass $m_{f}$, which is required for the  
whole mission  and the number 
of the fix-point iterations are considered 
for the state investigation. For each mass 
the mean value and the standard deviation
over the 
options given 
in the tables \ref{tab_3} and \ref{tab_4}
following the equations \eqref{eq_36} and 
\eqref{eq_37}, respectively were conducted.
\begin{equation}
    \label{eq_36}
    \bar{x} = \frac{1}{N}\sum_{i=1}^{n}x_i
\end{equation}

\begin{equation}
    \label{eq_37}
    \sigma = \sqrt{\frac{1}{N} \sum_{i}^{N}(x_i - \bar{x})^{2}}
\end{equation}

The reason for monitoring 
the mentioned masses is as follows. Since 
only the cruise flight segment involves 
a physical based equation rather 
than the empirical collected fuel fractions, which 
are constants and thus 
their multiplication results to a 
new constant (commutative behavior), 
only a scaling difference between 
the masses is expected. The fuel 
fractions are used only before and after 
the cruise flight segment. Therefore, 
if mistakes occurs within the investigation, 
plotting all the masses enables to find 
them easily visually. Thus considering the
mentioned masses can be viewed as a 
verification technique. The second 
reason is, the impact of the different 
surrogate model options shall not 
only be looked at the final outcome, 
total fuel mass $f_m$, but also on 
the masses in between. \newline 

The number of fix-point iterations 
for the state allows making 
assumptions about the execution time. 
In general, it can be assumed, the less 
the number of fix-point iterations, the 
less the execution time is. Thus, it 
provides information about the impact 
of the model parameter on the execution time
and convergence behavior. For only Kriging,
8 options were tested, therefore 8 results 
are obtained. From here it is possible to 
compare only within the Kriging options.
For TPS the same is done with its 
own 8 results. For RBF 7 different
kernel versions were tested. With the 
presented method the surrogate 
model can only be compared within its 
own surrogate model options environment.
However, it is desired 
to compare Kriging with TPS and with 
\emph{SciPy}s RBF, which would result in $8+8+7 = 23$ results 
to be compared. Note, for 
further reading RBF will be used 
as an abbreviation for \emph{SciPy}s RBF. 
Since this is too much for one 
single plot, the mean value for
each surrogate model (Kriging, TPS, RBF) is 
calculated and compared. In case 
the mean values matches or exhibits 
a low deviation it can be assumed that 
the corresponding models derive to 
same or resembling solutions and vice versa. \newline


By looking at the standard deviation within 
each model, a robustness analysis 
can be performed. In case the standard deviation (std) 
is low, the respective surrogate model is not influenced much 
by the input parameters and thus can be 
seen as stable. A stable or robust 
might not be the most accurate model, however,
it can be assumed to be more reliable for 
unknown complex underlying functions. Also, 
the robustness of the surrogate models 
(Kriging, TPS, RBF) was compared among 
each other.
All investigations were performed for both missions 
separately and together.
For the gradient
version, the same is done plus the 
execution time for the whole process including 
reading the aerodynamic input files and 
writing out all the gradients.\newline

The next step was to explore the accuracy of 
the surrogate models. For this purpose, 
a set of 38 training data was provided. With 
this set 3 different accuracy investigations were employed.
In the first version 20 sample points were 
removed and the surrogate models were trained 
thus with the remaining 18 sample points 
(V18). The second version, 10 sample 
points were removed and the surrogate models 
were trained with 28 sample points (V28).
In these both cases, the obtained 
model was invoked at the removed samples locations.
The actual functions value for 
$\tilde{LoD}, \tilde{AoA}, \tilde{TSFC}$
at the removed values are known, since
they were part of the initial 38 sample 
points. In order to evaluate 
the error for the surrogate models, 
RMSE according to equation \eqref{eq_24}
was used. For the third investigation, 
the \emph{Leave-one-out cross-validation}
(LOOCV) was performed. The idea 
of this method is, one sample point 
is left out and with the remaining 
sample points a surrogate model is 
trained. In case of 38 sample 
points (V38) there are 38 ways 
to leave one sample point out and 
thus 38 different surrogate models 
can be generated. Each 
generated surrogate model is 
invoked at the
location of the left out sample point
and by exploiting RMSE the error 
was measured.
To be more precise, \emph{missioninformer} 
automatically finds non trimmed sample data and 
does not include it for generating surrogate models.
For this workflow, instead of the overall 38 
sample points, only 35 fully trimmed 
sample points,  were used.  \newline

For the 3 
mentioned versions, each version can 
be tested only with regard to the surrogate 
model (Kriging, TPS, RBF) and inside 
a respective surrogate model there are 
8 options for Kriging and TPS and 7 
options for RBF. However, because 
bad execution time was observed when 
using \emph{SciPy}s RBF, only Kriging and TPS 
were chosen for the mentioned investigation.
In case of RBF's quinitc kernel, even 
after 4 hours no solution was obtained.
The explained process is performed 
for the state and the gradient version.
The options which exhibit the least 
RMSE are collected and presented in the 
results subsection in \ref{subsec_Krig_TPS_Accu}.


\section{Shape Gradients}
\label{sec_Shape_Gradients}
In this section it shall be explained, 
how the gradients of the 
presented cruise fuel burn equation (ODE in equation \eqref{eq_10}), 
can be computed. When the gradients are 
mentioned, then gradients with respect to 
the shape parameter, which are used to 
define the aircraft wing, are meant. For this 
work two main methods were evolved, an 
analytical and a numerical approach. The first one, however 
showed not be successful. A possible explanation for that 
will be given in the upcoming section. The numerical 
approach is based on the 
\emph{Central differencing Scheme} and is 
divided into 2 different workflows 
called direct and indirect, which will 
be elaborated in subsection \eqref{subsec_CDS}.
Furthermore, a step-size study in order to find 
a reasonable step-size and thus 
reliable gradients will be topic of the subsection 
\ref{subsec_CDS}

\subsection{Analytical attempt}
\label{subsec_Anly_Attempt}
Assuming, an analytical solution is possible, 
a fast and maximal computational accurate solution is obtained.
These were the main reasons for trying to find an 
analytical solution. Fast in this regard means, 
the result is received immediately after 
all mathematical operations (summation, multiplication) 
are executed without any necessity of additional repetitive 
loops. The state equation for the cruise fuel burnt is 
written below again and for simplicity will be referred to as  
$ODE$ and the objective is to find $dODE$.

\begin{equation*}
    ODE = \frac{dm_{cr}}{ds} = \frac{g}{a} \: \frac{1}{Ma}\;
    TSFC\, (m_s - m_{fe}) \; \frac{LoD\,tan(\theta)+1}
    {LoD\,cos(AoA) + sin(AoA)}
\end{equation*}

The ODE's gradient $dODE$ with respect to the shape 
parameters is already given in \cite{ilic_goal_2013}. However, 
the same work has been done again in order to 
verify the existing solution. Once by hand and once 
by using Python's library \emph{SymPy} for a symbolic solution.
Ilic \cite{ilic_goal_2013} results could be 
proven to be entirely correct and shall be 
stated in the upcoming equations. Since 
the gradient version is a term loaded equation, 
it will be split into groups, as advised in 
\cite{ilic_goal_2013}. The grouped 
equation as the new basis is given in equation
 \eqref{eq_38} for which its terms definitions 
 are provided in equations \eqref{eq_39}
 to \eqref{eq_41}


\begin{equation}
    \label{eq_38}
    ODE = k_c \, k_a \, k_{m}
\end{equation}

\begin{equation}
    \label{eq_39}
    k_c  = \frac{g}{a} \frac{1}{Ma} \, TSFC
\end{equation}

\begin{equation}
    \label{eq_40}
    k_a  = \frac{LoD \, tan(\theta) +1 }{LoD \, cos(AoA) + sin(AoA)}
\end{equation}

\begin{equation}
    \label{eq_41}
    k_m  = m_s - m_{cr}
\end{equation}

A change in the shape parameter $p$ results 
into a change in $TSFC\,, m_s, m_{cr}, LoD, AoA$, which 
means these variables are not to be considered as constants.
Equations \eqref{eq_42} to \eqref{eq_47} shows 
the general form of the $dODE$, where 
the flight path angle $\theta \neq 0$ is not 
necessarily zero. The equations \eqref{eq_44}
and \eqref{eq_45} results from 
the steady state flight condition. Here 
the denoting of the variables 
$q,\, S,\, W\,, L\,, D \, , T$ is the dynamic pressure,
reference wing area, weight, lift, drag and thrust, 
respectively.


\begin{equation}
    \label{eq_42}
    \frac{dk_a}{dp} = \frac{[ tan(\theta) \, sin(AoA) - cos(AoA)]
    \,\frac{dLoD}{dp} + [LoD \,sin(AoA) - cos (AoA)] 
    \,[LoD \, tan(\theta) + 1] \, \frac{dAoA}{dp} }
    {[LoD \, cos(AoA) + sin (AoA)]^{2}}
\end{equation}


\begin{equation}
    \label{eq_43}
    \frac{dLod}{dp} = \frac{d}{dp} \, \frac{C_L}{C_D} = 
    \frac{1}{C_{D}^{2}} \left(C_D  \frac{dC_L}{dp} - 
    C_L \frac{dC_D}{dp}  \right)
\end{equation}


\begin{equation}
    \label{eq_44}
    L = W \,cos(\theta) - [D + W \, sin(\theta)]\, tan(AoA)
\end{equation}

\begin{equation}
    \label{eq_45}
    C_L = \frac{m\, g}{q \, S}\,[cos(\theta) - sin(\theta) tan(AoA)] -
     C_D \, tan(AoA)
\end{equation}

\begin{multline}
    \label{eq_46}
    % \begin{split}
        \frac{dC_L}{dp} = \frac{g}{q\, S}\, [cos (\theta) - sin(\theta)\, tan(AoA)]
        \frac{dm}{dp} -
         \frac{m \, g}{q \ S^{2}}\,[cos(\theta) - sin (\theta) 
        tan(AoA)] \frac{dS}{dp} \\
        - \frac{1}{cos(AoA)^{2}} \left(\frac{m\, g}{q \, S} sin(\theta) + C_D
        \right) \frac{dAoA}{dp} - tan (AoA) \frac{dC_D}{dp}
    % \end{split}
\end{multline}

\begin{equation}
    \label{eq_47}
    \left(\frac{dC_L}{dp}\right)_{T=0} = \frac{g}{q\, S}cos(\theta) 
    \frac{dm}{dp} - \frac{m\, g}{q\, S^{2}} cos(\theta)\frac{dS}{dp}
\end{equation}

The \emph{missioninformer} assumes the flight 
path angle $\theta$ to be zero and thus the 
above stated equations are simplified to 
the two upcoming equations. Even though 
it has been showed how $LoD$ can be 
derived w.r.t. the shape parameter, however 
since its gradient is provided 
as input data, there is no need 
to its equation.


\begin{equation}
    \label{eq_48}
    k_a (\theta = 0)  = \frac{1 }{LoD \, cos(AoA) + sin(AoA)}
\end{equation}

\begin{equation}
    \label{eq_49}
    \frac{dk_a(\theta = 0)}{dp} =
     \frac{- cos(AoA)
    \,\frac{dLoD}{dp} + [LoD \,sin(AoA) - cos (AoA)] 
    \, \frac{dAoA}{dp} }
    {[LoD \, cos(AoA) + sin (AoA)]^{2}}
\end{equation}

The mass term $k_m$ was the obstacle, why the gradients 
can not be solved  analytically as it will be shown now.
The starting mass $m_s$ can be modeled as 
done in equation \eqref{eq_50}, where 
$m_{struc},\, m_p,\,  m_{cr}(R),\, m_o$ are denoted 
as masses of structure, payload, cruise and other, respectively.
The structure and other masses as well the payload 
are not meant to change by changing the shape parameter.
The only remaining mass, which is desired to 
change with the cruise range $R$ is $m_{cr}(R)$. 
Transforming this idea to mathematical expressions,
the equations \eqref{eq_51} to \eqref{eq_52}
are obtained.
\begin{equation}
    \label{eq_50}
    m_s = m_{struc} + m_p + m_{cr}(R) + m_o
\end{equation}

\begin{align}
    \frac{dm_{struc}}{dp} &= 0 \label{eq_51}\\
    \frac{dm_{p}}{dp} &= 0\\
    \frac{dm_{cr}(R)}{dp} &\neq 0\\
    \frac{dm_{o}}{dp} &= 0 \label{eq_52}
\end{align}

Following these the equation \eqref{eq_53} is 
obtained and by inserting this to the mass definition 
from equation \eqref{eq_41}, the undesired state 
given in 
equation \eqref{eq_54} can be observed. 
\begin{equation}
    \label{eq_53}
    \frac{dm_s}{dp} = \frac{dm_{cr}(R)}{dp}
\end{equation}

\begin{equation}
    \label{eq_54}
    \begin{aligned}
        \frac{dk_m}{dp} &= \frac{dm_s}{dp} - \frac{dm_{cr}(R)}{dp}\\
                    &\rightarrow \frac{dm_{cr}(R)}{dp} -  \frac{dm_{cr}(R)}{dp} \\     
        &\Rightarrow 0
    \end{aligned}
\end{equation}

In this case the term for which the gradient is desired to be calculated 
vanished. Therefore, the approach stated in 
equation \eqref{eq_55} was pursued instead, which leads 
to equation \eqref{eq_55_dkm_dp}. In words, 
the change of the starting mass is assumed to be constant and 
thus is not influenced by the change of any shape parameter.
At this point, it shall clearly be highlighted that 
this assumption is only made in 
order not to lose the cruise fuel mass term. Whether 
this models the physical behavior correctly 
was not known at this stage.

\begin{equation}
    \label{eq_55}
    \frac{dm_s}{dp}  = 0 
\end{equation}

\begin{equation}
    \label{eq_55_dkm_dp}
    \frac{dk_m}{dp}  = \frac{-dm_{cr}}{dp}
\end{equation}

However, after having performed, the complete 
process for obtaining the analytical solution 
for the cruise fuel mass gradients, 
it can be stated that the assumption
made in equation \eqref{eq_55} is 
not correct. For this knowledge to gain, 
the gradients were calculated numerically  
using central finite differences.
Due to the big deviation between 
the analytical and the numerical 
solution, the proposed finding can be given.
Nevertheless, the further progress for 
receiving the final analytical solution 
shall be explained. The variables $m_{f,zero},\,
f_{reser}, \, m_f$ for 
the upcoming equations are 
denoted as zero fuel mass, fuel reserve 
mass, and total 
fuel for all flight segments, respectively. 
As explained in section \ref{sec_Fuel_Mass_iter}
the variable starting with $fr_i$ are fuel 
fractions, where  
$fr_1 \,, fr_2\,,fr_3 \,, fr_4,\, fr_5\,, fr_6\, , fr_7, \,
fr_{cr}$ 
are for the  
flight segments, engine start, taxi to runway, takeoff,
climb and acceleration, taxi to parking,
landing, descent and cruise fuel fraction respectively.


\begin{equation}
    \label{eq_55_Fcr}
    fa \,(fr_{cr}) = f_{zero} \, fr_{cr} \, fr_1 \, fr_2\,fr_3 \, fr_4 \, 
    fr_5\, fr_6\, fr_7
\end{equation}

\begin{equation}
    \label{eq_56}
    m_{f} = \,\frac{fa (fr_{cr}) - m_{f,zero}}{f_{reser} -fr_{cr} 
    \,fr_1 \, fr_2\,fr_3 \, fr_4 \, 
    fr_5\, fr_6\, fr_7 }
\end{equation}

\begin{equation}
    \label{eq_57}
    fr_{cr} =  \frac{m_s - m_{cr}}{m_s}
\end{equation}

\begin{equation}
    \label{eq_58}
    m_s = (m_{f,zero} + m_{f})  \, fr_1 \, fr_2\,fr_3 \, fr_4
\end{equation}


The objective is to solve the equation \eqref{eq_56}. However, the 
problem is that the three equations \eqref{eq_56} to \eqref{eq_58}
are coupled. This can be seen better by rearranging the following 
equation. 

\begin{equation}
    \label{eq_59}
    ff = fr_1 \, fr_2\,fr_3 \, fr_4 ,\, fr_5\, fr_6\, fr_7 
\end{equation}

\begin{equation}
    \label{eq_60}
    \begin{aligned}
        m_f &= \frac{fr_{cr} fr_1 \, fr_2\,fr_3 \, fr_4 ,\, fr_5\, fr_6\, fr_7 -1
        }{f_{reser} - fr_{cr} ,\ fr_1 \, fr_2\,fr_3 \, fr_4 ,\, fr_5\, fr_6\, fr_7 }
        \, m_{f,zero}\\
        &= \frac{fr_{cr} ,\ ff - 1   }{f_{reser}- fr_{cr} ,\ ff  } \, m_{f,zero}
    \end{aligned}
\end{equation}

\begin{equation*}
    fr_{cr} =  \frac{m_s - m_{cr}}{m_s}
\end{equation*}

\begin{equation*}
    m_s = (m_{f,zero} + m_{f})  \, fr_1 \, fr_2\,fr_3 \, fr_4
\end{equation*}


It can be observed that in equation \eqref{eq_60}, the 
output of the ODE, the fuel cruise mass $m_{cr}$ is required. 
This, according to the equation \eqref{eq_57}, requires for its solution 
the cruise starting mass $m_s$. However, for 
getting the starting mass $m_s$, 
according to equation \eqref{eq_58}, the total fuel mass 
$m_f$ is demanded. The problem can be understood visually 
quite easily and thus is provided in figure \ref{fig_6_Coupled}.

\begin{figure}[!h]
        \centering
        \def\svgwidth{0.4\textwidth}
        \input{2_Figures/2_Task/2_Coupled.pdf_tex}
        \caption{Coupled mass equations}
        \label{fig_6_Coupled}
\end{figure}

\FloatBarrier
One way to solve equation is to solve the equation \eqref{eq_62}
with another fix-point iteration. The used 
scalars are given in equations \eqref{eq_59} and \eqref{eq_61}.
The one disadvantage lies 
in the nature of fix-point iterations, its repetitiveness, 
which requires higher execution time. 
The other is, that a more precise solution could 
be obtained. 
\begin{equation}
    \label{eq_61}
    ff_2 =  fr_1 \, fr_2\,fr_3 \, fr_4
\end{equation}

\begin{equation}
    \label{eq_62}
    m_f = m_{f,zero} = \dfrac{\left(1- \dfrac{m_{cr}}{(m_{f,zero} + m_f) \ ff_2}
    \right)ff -1}{f_{reser} - ff \, \left(  1 -\dfrac{m_{cr}}{(m_{f,zero} + m_f) \ ff_2}\right)}
\end{equation}

\vspace{0.17cm}
However, with some trial and error this coupling could be resolved 
and a without the necessity of fix-point 
iterations could be found, which is given in equation 
\eqref{eq_63}. The form of the solution, which was found by 
hand, originally contained a  square root. By making use of 
Matlab and Pythons library \emph{SymPy}, a symbolic solution 
could also be obtained. Matlabs and \emph{Sympy}s solution 
were equivalent in their form, thus also in their results, when 
inserting values for the variables. However, their form 
of solution did not contain any square root. 
The stated equation \eqref{eq_63} was used for further 
progress, since it is assumed to be easier for 
deriving gradients.

\begin{equation}
    \label{eq_63}
    m_f = \dfrac{ff_2 \, m_{f, zero} + ff \, m_{cr} - ff \, ff_2 \, m_{f, zero}}
    {ff_2 \, (ff - f_{reser})}
\end{equation}

In order to highlight why equation \eqref{eq_63} is easier
for deriving gradients, the  considered constants are
placed on the left side and the depended or relevant part 
for derivation are on the right side in equation 
\eqref{eq_64} 

\begin{equation}
    \label{eq_64}
    m_f = \dfrac{ff_2 \, m_{f, zero}  - ff \, ff_2 \, m_{f, zero}}
    {ff_2 \, (ff - f_{reser})} + \dfrac{ff \, m_{cr}(p)}{ff_2 \, (ff - f_{reser})} 
\end{equation}

Recall, $m_{cr}$ is the fuel mass for the cruise segment 
and is the output of the ODE, which now is 
desired to be derived with respect to the shape 
parameters. In equation \eqref{eq_64}, the term 
$m_{cr}(p)$ is indicated to be the only important 
parameter for the derivation. Therefore, the 
derivative of the total fuel mass can 
be written as in equation \eqref{eq_66}, where
the mathematical definition for $dODE$ is given 
in equation \eqref{eq_65}.

\begin{equation}
    \label{eq_65}
    dODE = \frac{d}{dp}\, ODE = \frac{d}{dp}\left(\dfrac{dm_{cr}}{ds}\right)
\end{equation}

\begin{equation}
    \label{eq_66}
    \frac{dm_f}{dp}= 
    \dfrac{ff}{ff_2 \, (ff - f_{reser})} \, \cdot \, dODE
\end{equation}


Finally, by inserting all equations into each 
other the $dODE$'s final version is given 
equation \eqref{eq_67}, where $LoD, AoA, TSFC$
and their gradients are obtained by invoking 
the surrogate models, $k_a \, ,k_m ,\, k_c$
are found in equations \eqref{eq_48}, \eqref{eq_41} and 
\eqref{eq_39}, respectively. The gradients 
$\frac{dk_a}{dp}, \, \frac{dk_m}{dp}$
are found in equations \eqref{eq_49} and \eqref{eq_55_dkm_dp}, 
respectively. The variable $p$ here is defined 
as a scalar for the sake of simplicity. In application 
$p$  in the equations \eqref{eq_65} to \eqref{eq_67} is  
a vector. Thus, the equations 
need to be solved for each component 
of $p$.


\begin{equation}
    \label{eq_67}
    \begin{aligned}
        dODE &= \frac{ODE}{dp} = \frac{d}{dp}\left(\dfrac{dm_{cr}}{ds}\right)\\
        &= k_a \, k_m \, k_c \, \frac{dTSFC}{dp} +k_c \, TSFC \, k_a \, 
        \frac{dk_m}{dp} +k_c  TSFC \, k_m \, \frac{dk_a}{dp} \\
        &= k_c \left(k_a \, k_m \, \frac{dTSFC}{dp}+ 
        TSFC \, k_a \, \frac{dk_m}{dp} + TSFC \, k_m \, 
        \frac{dk_a}{dp}\right)\\
        &= k_c \left(k_a \, k_m \, \frac{dTSFC}{dp}+ TSFC \, k_m \, 
        \frac{dk_a}{dp}\right) - k_c \, TSFC \, k_a \, \frac{dODE}{dp} \\
        &=  k_c\, \dfrac{\left(k_a \, k_m \, \dfrac{dTSFC}{dp}  + TSFC \, k_m \,
         \dfrac{dk_a}{dp}\right)}
        {1+ k_c \, TSFC \, k_a}
    \end{aligned}
\end{equation}

After having seen the formal equations, there are 
two possible ways to calculate the gradients,
the one-shot method and the indirect method.
Both shall be explained with their respective 
workflow in figures \ref{fig_7_Oneshot} and 
\ref{fig_8_Iterative}. 
The one-shot version, depicted in figure 
\ref{fig_7_Oneshot}, will let 
the state fix point-iteration finish as 
discussed in section \ref{sec_Fuel_Mass_iter}.
Based on the converged results it 
will then directly calculate the gradients 
for which only one evaluation is required. Since 
the state fix-point iteration is indispensable
with effectively only one more function call the 
gradients are found. \newline

The iterative workflow is depicted 
in figure \ref{fig_8_Iterative}.
The initial starting and total fuel mass 
$m_{s,0}, m_{f,0}$ are guessed as explained 
in section \ref{sec_Fuel_Mass_iter}. In 
order to calculate the gradient 
of the cruise fuel weight the 
initial cruise fuel weight and cruise
starting mass is passed to the dODE-block, which 
calculates the gradient of the cruse fuel 
weight with respect to the shape parameters. 
This block is not required in the 
one-shot-version nor in the state 
fix-point iteration explained in section 
\ref{sec_Fuel_Mass_iter}. Thus, 
here an additional calculation block is injected.
It is used for computing 
the gradient of the total fuel mass. 
Now the whole process is repeated 
as long as the deviation between the 
previous and the current 
gradient of the total cruise 
fuel mass is below 
a defined threshold. Both methods were 
implemented in \emph{missioninformer}, 
however they are not active, because of 
the explained reasons. 



\begin{sidewaysfigure} [!h]
    \resizebox{1.\textwidth}{!}{
        \hspace*{-5cm} 
    
%%% Preamble Requirements %%%
% \usepackage{geometry}
% \usepackage{amsfonts}
% \usepackage{amsmath}
% \usepackage{amssymb}
% \usepackage{tikz}

% Optional packages such as sfmath set through python interface
% \usepackage{sfmath}

% \usetikzlibrary{arrows,chains,positioning,scopes,shapes.geometric,shapes.misc,shadows}

%%% End Preamble Requirements %%%

\input{/home/jav/Progs/Virt_Env/writing/lib/python3.12/site-packages/pyxdsm/diagram_styles.tex}
\begin{tikzpicture}

\matrix[MatrixSetup]{
%Row 0
\node [DataIO] (output_ode_Solv) {$\begin{array}{c}\text{initial starting} \\ \text{and mission fuel } \\ \text{mass } ($$m_{s,0}, m_{f,0})$$\end{array}$};&
&
&
&
&
&
\\
%Row 1
\node [Function] (ode_Solv) {$\begin{array}{c}\text{ODE}\end{array}$};&
\node [DataInter] (ode_Solv-mass_Calc) {$\begin{array}{c}$$m_{cr,0}$$\end{array}$};&
&
&
&
&
\\
%Row 2
\node [DataInter] (mass_Calc-ode_Solv) {$\begin{array}{c} m_{f,n}, m_{s,n}\end{array}$};&
\node [Function] (mass_Calc) {$\begin{array}{c}\text{mass recalc}\end{array}$};&
\node [DataInter] (mass_Calc-ode_Solv_2) {$\begin{array}{c}\, m_{s}*\end{array}$};&
&
&
&
\node [DataIO] (right_output_mass_Calc) {$\begin{array}{c}m_{s}*, \,  m_{f}* \end{array}$};\\
%Row 3
&
&
\node [Function] (ode_Solv_2) {$\begin{array}{c}\text{ODE}\end{array}$};&
\node [DataInter] (ode_Solv_2-dode_Solv_2) {$\begin{array}{c}m_{cr}* \end{array}$};&
&
&
\node [DataIO] (right_output_ode_Solv_2) {$\begin{array}{c}m_{cr}*\end{array}$};\\
%Row 4
&
&
&
\node [Function] (dode_Solv_2) {$\begin{array}{c}\text{dODE}\end{array}$};&
\node [DataInter] (dode_Solv_2-dx_2) {$\begin{array}{c}\dfrac{dm_{cr}*}{dp}\end{array}$};&
&
\node [DataIO] (right_output_dode_Solv_2) {$\begin{array}{c}\dfrac{dm_{cr}*}{dp}\end{array}$};\\
%Row 5
&
&
&
&
\node [Function] (dx_2) {$\begin{array}{c}\dfrac{dm_f}{dp}\end{array}$};&
&
\node [DataIO] (right_output_dx_2) {$\begin{array}{c}\dfrac{dm_{f}*}{dp}\end{array}$};\\
%Row 6
&
&
&
&
&
&
\\
};

% XDSM process chains


\begin{pgfonlayer}{data}
\path
% Horizontal edges
(ode_Solv) edge [DataLine] (ode_Solv-mass_Calc)
(mass_Calc) edge [DataLine] (mass_Calc-ode_Solv)
(mass_Calc) edge [DataLine] (mass_Calc-ode_Solv_2)
(ode_Solv_2) edge [DataLine] (ode_Solv_2-dode_Solv_2)
(dode_Solv_2) edge [DataLine] (dode_Solv_2-dx_2)
(mass_Calc) edge [DataLine] (right_output_mass_Calc)
(ode_Solv_2) edge [DataLine] (right_output_ode_Solv_2)
(dode_Solv_2) edge [DataLine] (right_output_dode_Solv_2)
(dx_2) edge [DataLine] (right_output_dx_2)
% Vertical edges
(ode_Solv-mass_Calc) edge [DataLine] (mass_Calc)
(mass_Calc-ode_Solv) edge [DataLine] (ode_Solv)
(mass_Calc-ode_Solv_2) edge [DataLine] (ode_Solv_2)
(ode_Solv_2-dode_Solv_2) edge [DataLine] (dode_Solv_2)
(dode_Solv_2-dx_2) edge [DataLine] (dx_2)
(ode_Solv) edge [DataLine] (output_ode_Solv);
\end{pgfonlayer}

\end{tikzpicture}

    }
    \caption{Analytical attempt to calculate gradients - one-shot version}
    \label{fig_7_Oneshot}
\end{sidewaysfigure}

\begin{sidewaysfigure} [!h]
    \resizebox{1\textwidth}{!}{
        \hspace*{-5cm} 
    \input{2_Figures/2_Task/4_Grad.tikz}
    }
    \caption{Analytical attempt to calculate gradients - iterative version}
    \label{fig_8_Iterative}
\end{sidewaysfigure}



% Follwing Command: The figure from previous section shall not 
% be wihtin this section
\FloatBarrier
\subsection{Numerical approach}
\label{subsec_CDS}

As mentioned in the previous section, an analytical 
solution was pursued. However, the approach 
in section \ref{sec_Shape_Gradients} 
had to be verified. In this subsection, 
the numerical approach for computing the gradients 
of the cruise fuel weight and the total fuel weight 
with respect to the shape parameters shall be 
explained. The three most frequently 
employed methods for 
deriving the gradients numerically are 
forward, backwards and central finite 
differences. In the majority of cases 
central finite differences or also known 
as central differencing scheme (CDS), which exhibits 
the best gradient computing accuracy. The equations 
for forward, backwards and central differencing 
are given in equations \eqref{eq_68}, \eqref{eq_69}
and \eqref{eq_70}, respectively. 

\begin{equation}
    \label{eq_68}
    \frac{df(x)}{dx} \approx \frac{\Delta f(x)}{\Delta x}=
     \frac{f(x+h_s) - f(x)}{h_s}
\end{equation}

\begin{equation}
    \label{eq_69}
    \frac{df(x)}{dx} \approx \frac{\Delta f(x)}{\Delta x}=
     \frac{f(x) - f(x-h_s)}{h_s}
\end{equation}

\begin{equation}
    \label{eq_70}
    \frac{df(x)}{dx} \approx \frac{\Delta f(x)}{\Delta x}=
     \frac{f(x+h_s) - f(x -h_s)}{2\, h_s}
\end{equation}

Here the $f(x),\, h_s$ are denoted as any function with 
the input argument $x$ and the step-size, respectively. 
The choice of the step-size $h_s$ is important 
and thus will be dealt with later in more details.  
As pointed with the approximation sign ($\approx$), 
it is only an approximation of the gradient. The reason for 
this can be explained by viewing 
their derivation. In order to get the one of the 
three forms a given function is approximated by 
the Taylor series and since only an infinite 
long Taylor series can give the accurate solution, 
aborting the Taylor series after one or 
two terms cannot be accurate. Here it can also 
be observed that the gradients only around 
the approximated location can be considered 
acceptable. In other words, the higher the step-size 
$h_s$ gets, the lower the 
quality of the approximation gets. The 
approximation error is also called 
truncation error. Since computers are used 
for calculating the derivate another 
error must be considered. The round off error, 
occurs due to the nature of storing 
digits into the computer memory or RAM.
Having a decimal number, it 
consists of multiple digits. Each digit 
claims some memory and in case providing 
all numbers consisting of many digits
all desired memory, two obstacles 
are faced. The memory capacities can be 
overwhelmed quite easy. The second problem is that, 
if only one number already contains more than 100 
digits, it's processing by the
hardware runs into  
infeasibly high execution time. Therefore, 
state-of-the-art accuracy uses double 
precision to store numbers. Already here 
an approximation of the real number is done.
This computer based error is called 
round off error.\newline 

Up to now, two kinds of errors were introduced.  
The third error is called the abortion error. 
An infinite Taylor series would 
remove the abortion 
error, thus it would be vanished. In the introduction 
the forward and backwards finite differing methods 
were stated to be less accurate than the central 
finite differencing scheme. The reason for that can 
be seen by their derivation. In short, 
forward and backwards differing are methods of first order,
 whereas 
the central differencing scheme is a method of 
second order accuracy.\newline

Since the CDS is considered to be more 
accurate in most applications, it is implemented 
in \emph{missioninformer}. The equation \eqref{eq_70} 
shows that CDS can be used as a block-box method, 
where $f$ can by any computation in \emph{missioninformer} 
and $x+h_s,\, x- h_s$ represents the parameter, for which 
the gradient is desired. It needs to be changed 
by adding and subtracting the step-size $h_s$.
However, applying this theory is not straight forward 
for the calculations done in \emph{missioninformer} and 
thus the CDS does not act as a black-box function.
The reasons for that are the subject of the 
upcoming discussion.\newline 

The gradient of the total fuel mass 
$m_f$ with respect to the shape 
parameter is the object of desire
$\frac{dm_f}{\vec{p}}$. The reason 
why the shape parameter variable 
$\vec{p}$ is chosen to be formulated as 
a vector is to highlight, that 
not only one gradient is calculated.
Because no optimization can 
be accomplished with having defined 
only one shape parameter, it is more 
correct to talk about the shape 
parameter vector $\vec{p}$, which 
notation from here on will be continued 
to be followed.
In order to apply the straight forward CDS as described 
with equation \eqref{eq_70}, the relationship for each 
variable $LoD, AoA, TSFC$ to all the shape parameters 
$\vec{p}$, as stated in equation \eqref{eq_72}
must be known.

\begin{equation}
    \label{eq_72}
    LoD(\vec{p}),  \quad AoA(\vec{p}), \quad TSFC(\vec{p})
\end{equation}

However, the encountered problem is that these relationships 
from equation \eqref{eq_72} are not known, thus the 
black box straight forward CDS presented in equation 
\eqref{eq_70} cannot be applied.
The following simple and mathematical correct workaround was found. 

\begin{align}
    LoD_{pertubated} &= h_s \, \dfrac{dLoD}{d\vec{p}} \label{eq_73}\\
    AoA_{pertubated} &=  h_s \, \dfrac{dAoA}{d\vec{p}}\\
    TSFC_{pertubated} &=  h_s \, \dfrac{dTSFC}{d\vec{p}} \label{eq_74}
\end{align}


With these background, two different workflows, for 
solving the gradients of the total fuel mass 
with respect to the shape parameters can be 
discussed. The difference in both occurs in 
using the auxiliary cruise fuel fraction term 
$fr_{cr}$ given in the following equation 
or inserting it directly into the remaining 
equations as presented in section \eqref{sec_Fuel_Mass_iter}

\begin{equation}
    fr_{cr} =  \frac{m_s - f_{cr}}{m_s} \tag{\ref{eq_20_Cruise_Fraction}}
\end{equation}

In case of direct insertion, the equation 
\eqref{eq_63} is obtained, as explained 
in section \ref{sec_Shape_Gradients} and 
is referred to as \emph{direct method}
\begin{equation}
    m_f = \dfrac{ff_2 \, m_{f, zero} + ff \, m_{cr} - ff \, ff_2 \, m_{f, zero}}
    {ff_2 \, (ff - f_{reser})} \tag{\ref{eq_63}}
\end{equation}

For defining the fuel cruise fraction 
$fr_{cr}$ explicitly the method is 
referred to as \emph{indirect method}. 
The indirect method's solution  
is given in equation \eqref{eq_75},
which derivation can be seen in section \ref{sec_Fuel_Mass_iter}
with the equations \eqref{eq_17} to \eqref{eq_20}.

\begin{equation}
    \label{eq_75}
    m_{f} = \frac{fa - f_{zero}}{f_{reser} -fr_{cr} 
    \,fr_1 \, fr_2\,fr_3 \, fr_4 \, 
    fr_5\, fr_6\, fr_7 }
\end{equation}

After analyzing two main differencing could be
observed. The direct method often claims 
more fix-point iterations. In contrast, 
it was found to be more accurate 
by satisfying lower convergence criteria. In case 
of low convergence parameter, the 
indirect method stucked inside a loop, where 
it jumped between two values.
Since the \emph{missioninformer} has a limit 
of maximal allowed iterations of 500, 
the iterative method would stop 
after 500 iterations and thus would 
require more iterations than the 
direct method. However, a relative 
and absolute tolerance \cite{noauthor_numpys_2021} value of 
$5 \mathrm{e}{-5}$ is assumed to be accurate enough.
With this tolerance value, no 
permanent stay for the indirect method inside 
a loop could be observed.
Also, in cases where the missions were 
defined reasonably, only 4 iterations 
were required for obtaining the 
gradient of the total fuel mass $\frac{dm_f*}{d\vec{p}}$ 
per gradient component and perturbation 
direction.
Therefore, the \emph{missioninformer} default method 
is chosen to be the indirect method.\newline 

The parameter which defines the quality of 
the gradients by applying CDS is the 
choice of the step-size $h_s$. In order 
to find an appropriate $h_s$ two different 
missions were defined. One, which 
raised the \emph{constraint violation check} 
to output warning statements and 
one without. Thus, these missions can be 
considered to be hard to be solved (mission 1)
and the opposite (mission 2), 
respectively. For both missions, 
applying both methods (direct and indirect)
a step-size study is performed. In case 
of the harder to solve mission 1, the number 
of maximal fix-point iterations was reached. The 
missions definitions can be 
found in table \ref{tab_5_Mission_Step_Size}.
The difference 
between current table \ref{tab_5_Mission_Step_Size}
and the table \ref{tab_2} from subsection 
\ref{subsec_INvestigate_Surro} is only 
in the definition of the range for mission 1.
In table 
\ref{tab_5_Mission_Step_Size} this has been 
reduced by $1000$ meters.

\begin{table}[!h]
        \centering
        \begin{tabular}{l| c c c c c c }
            \textbf{Mass} & \multicolumn{1}{c}{\textbf{Mission 1}} &&& 
            \multicolumn{1}{c}{\textbf{Mission 2}}\\
            \hline
            maximal takeoff & $245 \mathrm{e}{3}$ &&&      $245\mathrm{e}{3}$         &&\\
            maximal landing & $192.2 \mathrm{e}{3}$ &&&      $192.2\mathrm{e}{3}$         &&\\
            operating empty & $132.5\mathrm{e}{3}$ &&&      $132.5\mathrm{e}{3} $         &&\\
            manufactures empty & $119.2\mathrm{e}{3} $ &&&      $119.2\mathrm{e}{3} $         &&\\
            maximum zero fuel  & $180.5\mathrm{e}{3} $ &&&      $180.5\mathrm{e}{3} $         &&\\
            maximum fuel & $107.6\mathrm{e}{3}$ &&&      $107.6\mathrm{e}{3}$         &&\\
            design payload & $38.52\mathrm{e}{3}$ &&&      $33.60\mathrm{e}{3} $         &&\\
            maximum payload & $48.00\mathrm{e}{3}$ &&&      $48.00\mathrm{e}{3}  $         &&\\
            \hline
            &  &&&               &&\\
            \textbf{Flying parameters} &  &&&               && \\
            \hline 
            $Ma$ & $0.83$ &&&      $0.82$         &&\\
            $h$ & $10.668\mathrm{e}{3}$ &&&      $11.000\mathrm{e}{3}  $         &&\\
            $R_{cr}$ & $10186.0\mathrm{e}{3} $ &&&      $5185.6\mathrm{e}{3} $         &&\\
            \hline
            &  &&&               &&\\
            \textbf{Others} &  &&&               && \\
            \hline 
            step-size & $1\mathrm{e}{-5} $&&&      $1\mathrm{e}{-5}$         &&\\
            weight-factor & $0.7$ &&&      $0.3$         &&\\
        \end{tabular}
        \caption{Missions definitions for investigation on step-size}
        \label{tab_5_Mission_Step_Size}
\end{table} 

\FloatBarrier
The convergence behavior of both missions and 
both methods is presented in section \ref{sec_Step_Size_Study}.
However, the final result is that missionsinformers
default step-size is chosen to be $h_s = 1 \mathrm{e}{-5}$.



% % ---------------- Task 3  ------------------------------

\chapter{Results}
\label{chap_Results}

In this chapter the results of the methods described 
in previous chapter shall be presented. The first 
topic for this purpose is the solution of 
the state ODE from equation \eqref{eq_10}.


\section{Solving the ODE}
\label{sec_Solve_ODE_Methods}

The main equation upon which's solution 
the remaining computations are based on 
is the ODE for the cruise 
fuel burnt mass $m_{cr}$.

\begin{equation}
    \frac{dm_{cr}}{ds} = \frac{g}{a} \, \frac{1}{Ma}\,TSFC \,(m_s + m_{cr})
    \, \frac{1}{
    LoD \, cos(AoA) + sin(AoA)}   \tag{\ref{eq_10}}
\end{equation}

The used technique therefore is by calling 
SciPy's function \emph{solve\_ivp} \cite{noauthor_scipys_2021}.
The library comes along with different solvers and the solution 
of all were investigated. However 
only a selection of the most interesting 
results  
shall be presented here. Having compared all 
available methods inside \emph{solve\_ivp} 
it could be found out that \emph{LSODA} and 
\emph{RK45} offered the least difference 
to the analytical solution obtained by equation 
\eqref{eq_16}. Its 
derivation and explanations can be 
read in sections \ref{sec_Solve_ODE}.
\begin{equation}
    m_{fe}(s = R_{cr}) =m_s \left(1-e^{\frac{-g}{a} \, 
    \frac{1}{Ma} \, TSFC \, 
    \frac{1}{LoD \,cos(AoA) + sin(AoA)} \, s} \right)
    \tag{\ref{eq_16}}
\end{equation}

The analytical solution is only valid, when
assuming that $LoD, AoA, TSFC$
are constant. Comparing the analytical solution 
with \emph{SciPy}'s numerical solution by employing 
all available  methods, it could be observed, that 
$LSODA$ exhibits best results. The figures from 
\ref{fig_9_LSODA} to \ref{fig_12_RK45_Standard} depicts 
the analytical and the numerical solution on the left 
side and their deviation on the right side. One of the 
options within \emph{LSODA} and \emph{RK45}
is to set points where an integration step shall be 
performed. The higher the number of the given 
calculation points is, the higher the accuracy of 
the outcome is. Figures \ref{fig_9_LSODA} and 
\ref{fig_10_RK45} depict the results when 
the user exerted calculation points. From range 
$R = [0;100]$ meter and for the last 100 meter of the 
cruise flight range, 500 uniformly 
calculation points were set. In between, 
$\frac{R_{max}}{8} = \frac{7297989}{8} = 912248 $
calculation points were set. Even though 
a high number of evaluation points 
leads to a more accurate result, recall that for 
each additional evaluation point, the 
surrogate models needs to be invoked. Therefore, 
it is desired to keep the number 
of the control points low, if possible.\newline

%%
\begin{figure}[!h]
    \centering
    \includegraphics[width =0.85\textwidth]{2_Figures/3_Task/1_LSODA.pdf}
    \caption{Solution obtained having used \emph{LSODA} with user defined 
    calculation points}
    \label{fig_9_LSODA}
\end{figure}
%
%
\begin{figure}[!h]
    \centering
    \includegraphics[width =0.85\textwidth]{2_Figures/3_Task/1_RK45.pdf}
    \caption{Solution obtained having used \emph{RK45} with user defined 
    calculation points}
    \label{fig_10_RK45}
\end{figure}


In figures \ref{fig_11_LSODA_Standard} and \ref{fig_12_RK45_Standard},
the same calculations are performed with \emph{LSODA} and \emph{RK45}, 
respectively, but with their methods default number 
of calculation points. The default number of calculation 
points is much lower, e.g. for \emph{RK45} less than 20 sample 
points were required. Comparing all the 4 presented
results it can be stated that the deviation or \emph{delta} 
between the analytical and numerical solution is 
in the order of grams, which is
sufficient for 
the accuracy requirement of the \emph{missioninformer}.
Even though \emph{LSODA} delivers the best performance, 
since \emph{RK45} requires less integrations steps 
points with barley less accuracy, \emph{RK45} with 
its default method for setting calculation 
points is set as default in the \emph{missioninformer}.


\begin{figure}[!h]
    \centering
    \includegraphics[width =0.85\textwidth]{2_Figures/3_Task/2_StandardLSODA.pdf}
    \caption{Solution obtained having used \emph{LSODA} without user defined 
    calculation points}
    \label{fig_11_LSODA_Standard}
\end{figure}


\begin{figure}[!h]
    \centering
    \includegraphics[width =0.85\textwidth]{2_Figures/3_Task/2_StandardRK45.pdf}
    \caption{Solution obtained having used \emph{RK45} without user defined 
    calculation points}
    \label{fig_12_RK45_Standard}
\end{figure}



\FloatBarrier
The results above were presented by assuming 
$LoD, AoA, TSFC$ to be constant. However, 
for \emph{missioninformer} they are not constant, but 
dependent on $Ma, h, m$. In order 
to see the effect of these variables not to 
be constant, all available methods 
within \emph{SciPy}s ODE solver were tested again.
For testing purposes no real aerodynamic 
data output was used, but rather some 
reasonable values, which were only dependent 
on $m$, were generated. This data 
set was the input for a \emph{1d} interpolation, which 
is also available in \emph{SciPy}. Coming to 
the workflow, every time the ODE is solved, the 
values for the variables $LoD, AoA, TSFC$ are 
obtained by the mentioned 1d interpolation.
This process can also be done with a lower 
or higher number of calculation points.
However, no deviation between the lower 
and higher number of calculation points 
version, could 
be observed.\newline 

The figures \ref{fig_13_LSODA_Inter} and 
\ref{fig_14_RK45_Inter} show the results, when the mentioned
1d interpolation and no interpolation is applied.
On the left side, the default mechanism 
for choosing calculation points is plotted.
On the right side, the above explained 
user defined calculation points were 
used for deriving the solution. Both 
sides show a deviation between 
the interpolation active and inactive versions, which 
is called \emph{delta} in the mentioned figures.
Figure \eqref{fig_13_LSODA_Inter} shows 
a strange and unexpected behavior in 
its delta. This is explained as follows.
The number of calculation 
points for the active and inactive 
with the defaults' 
method to define calculation points is 
not equal. 
In the inactive interpolation version, the 
default option for defining 
calculation points 
requires less calculation points.
However, for calculating delta, 
both number of calculation points 
mus be equal, to be valid.
Therefore,
the delta on the left side can 
not be considered to be correct. \newline 


This observation tells us, that \emph{LSODA} is more 
flexible in increasing the number of 
calculation points than RK45. This 
can bee seen by viewing the delta on the 
left side of the figure \ref{fig_14_RK45_Inter}.
Here, a monotonically increasing delta  can be identified. 
The default version of
\emph{RK45} took less than 20 calculation points, 
whereas \emph{LSODA} required around 60 
calculation points.


\begin{figure}[!h]
    \centering
    \includegraphics[width =0.9\textwidth]{2_Figures/3_Task/1_Effect_LSODA_High_Number_Ranges.pdf}
    \caption{Solution obtained having used \emph{LSODA} with and without user defined 
    calculation points and active and inactive interpolation}
    \label{fig_13_LSODA_Inter}
\end{figure}


\begin{figure}[!h]
    \centering
    \includegraphics[width =0.9\textwidth]{2_Figures/3_Task/1_Effect_RK45_High_Number_Ranges.pdf}
    \caption{Solution obtained having used \emph{RK45} with and without user defined 
    calculation points and active and inactive interpolation}
    \label{fig_14_RK45_Inter}
\end{figure}


\FloatBarrier
Comparing \emph{LSODA} with \emph{RK45} directly can be done 
via figure \ref{fig_15_RK45_LSODA}. Important 
for present concern is the right side, where 
the blue and orange graphs are the objects of 
our focus. However, the blue graph is not visible.
The reason for that is, that is has been 
draw before the orange graph has been drawn.
Therefore, the blue graph is beneath the 
covering orange graph. With this it can be 
said, no noticeable deviation in the 
results for active interpolation by having 
defined a ridiculously high number 
of calculation points can be seen. This leads to 
the following conclusion. No real difference 
in the accuracy of \emph{LSODA} and \emph{RK45} is 
exhibited, however, \emph{LSODA} requires more 
calculation points. As a consequence 
of this, \emph{RK45} can be chosen as the 
default method to solve the ODE 
in the \emph{missioninformer}.


\begin{figure}[!h]
    \centering
    \includegraphics[width =\textwidth]{2_Figures/3_Task/6_Compare_User_Default_RK45_LSODA_High_Number_Ranges.pdf}
    \caption{Direct compare \emph{RK45} with \emph{LSODA} for in/active and default/user versions}
    \label{fig_15_RK45_LSODA}
\end{figure}


\newpage
\FloatBarrier
\section{Conducting a step-size-study}
\label{sec_Step_Size_Study}
In section \ref{subsec_CDS} the reasons for 
the necessity of a step-size investigation 
has been described. In short it 
can be said, a wrong chosen step-size 
leads to not useable gradient values 
for the optimization. In this section the 
results of the performed step-size 
study shall be shown and elaborated.        
In order to calculate the gradients 
using different step-sizes, the 
procedure given in 
table \ref{tab_6} was employed.
The left side shows the interval 
from which the step-sizes were taken,  
on the right side, the number of evenly distributed 
points, which shall be used as the step size, 
are given. The investigation was conducted for 
2 different missions as explained in section 
\ref{subsec_CDS}. Both missions can be 
seen in table \ref{tab_5_Mission_Step_Size}.
Mission 1 encountered 
or failed the \emph{constraint violation check}, which 
is explained in section \ref{sec_Fuel_Mass_iter}.
In cases, where 
the \emph{constraint violation check} displayed warning 
messages, the respective mission required 
a high number of fix-point iteration. Mostly, the limit 
of maximal 500 allowed iterations was reached.
In contrast, mission 2 successfully passed 
the \emph{constraint violation check} and thus no 
warning messages were encountered. Also, 
the number of iterations were much lower.
\begin{table}[!h]
    \centering
        \begin{tabular}{l c}
            \multicolumn{1}{p{2cm}}{\textbf{Intervall}} &
            \multicolumn{1}{p{2cm}}{\textbf{\# points}}\\
            \hline
            $[1\mathrm{e}{-1} \; ; 1\mathrm{e}{-3} ]$ & $20$  \\
            $[1\mathrm{e}{-3} \; ; 1\mathrm{e}{-4} ]$ & $10$ \\
            $[1\mathrm{e}{-4} \; ; 1\mathrm{e}{-5} ]$ & $10$\\
            $[1\mathrm{e}{-5} \; ; 1\mathrm{e}{-6} ]$ & $15$\\
            $[1\mathrm{e}{-6} \; ; 1\mathrm{e}{-7} ]$ & $15$\\
            $[1\mathrm{e}{-7} \; ; 1\mathrm{e}{-8} ]$ & $10$\\
        \end{tabular}
        \caption{Chosen step-sizes }
        \label{tab_6}
\end{table}


Furthermore, two different values for the tolerance 
parameter where used for the above-mentioned investigations.
For the first examination the absolute 
and relative tolerance parameter \cite{noauthor_numpys_2021} for 
the direct and indirect gradient calculation 
workflows, which are explained in section \ref{subsec_CDS}, 
are set equal to $5\mathrm{e}{-5}$. For the 
second study, for the direct method, the 
tolerance parameter were set to $5\mathrm{e}{-6}$, 
which can be done since the direct method 
can reach finer tolerances as 
explained in section \ref{subsec_CDS}.
The CDS is computed in application for each
shape parameter. In this study, only five 
selected parameters were investigated.
 All the upcoming 
results are achieved by employing \emph{SciPy}'s RBF
with
multiquadric as kernel, which is given 
in equation \eqref{eq_34}, as the underlying 
surrogate model generator. \newline 

Note that not each 
analyzed result can be shown here, since 
the number of these would claim too many pages.
The outcomes, which describes the 
gained finding will be highlighted instead.
For this propose, the result will be
reviewed for the direct and indirect method, where 
the direct method has a tolerance parameter of $5\mathrm{e}{-6}$.
The results are only shown for mission 2
and for one shape parameter.\newline


The first interesting finding is that both 
methods, direct and indirect, 
retain a so-called plateau with 
respect to step size variations.
In the region of the plateau, 
the types of error, mentioned 
earlier, are in balance. In other words, 
within the plateau, no type of 
error dominates and the total value is small.
 In case of 
a too small step-size the round-off 
error becomes such high, that numerical 
instability is the consequence. In contrast,  
a too large step-size results in very 
high truncation error. In both cases, the 
result by using CDS cannot be considered 
to approximate the real gradient value.
This plateau  
was searched for, in order to be able to  
make statements and recommendations about the 
value for the step-size. This observation 
was seen in all tested cases for mission 2.
The figure \ref{fig_16} and \ref{fig_17} 
show the plateau for the direct and 
indirect method for mission 2, respectively.
It can be observed, that for both 
methods, direct and indirect, the  
plateau can bee seen around the 
step-size of $1\mathrm{e}{-5}$. This 
result was found also for 
remaining step-size studies  
the different shape parameters.\newline 

% =======================================================================
% ========================== SHAPE 0  =============================================
% =======================================================================
% ------------------- SHape -0 Miss 1----------------------
\begin{figure}[!h]
    \centering
    \includegraphics[width =0.9\textwidth]{2_Figures/3_Task/1_Steps/Shape_0/Dir_Miss_2_RBF.pdf}
    \caption{Step-Size-Investigation, Mission 2, Shape parameter $0$, Applied method: Direct, Unequal tolerance parameter}
    \label{fig_16}
\end{figure}


\begin{figure}[!h]
    \centering
    \includegraphics[width =0.9\textwidth]{2_Figures/3_Task/1_Steps/Shape_0/Indirect_Miss_2_RBF.pdf}
    \caption{Step-Size-Investigation, Mission 2, Shape parameter $0$, Applied method: Indirect, Unequal tolerance parameter}
    \label{fig_17}
\end{figure}

\FloatBarrier
However, a clear plateau could only 
be detected for mission 2. The deviation 
of the gradient with changing step-size
for mission 1 
was in the orders of $10^7 \,  kg/m$ and  $10^9\, kg/m$. 
This leads to the conclusion, when 
the \emph{constraint check violation} is 
passed successfully, the step-size 
of $1\mathrm{e}{-5}$ can be used for 
the CDS gradient calculation. Otherwise, when 
receiving warning statements by the 
\emph{constraint check violation}, 
not only can the step-size of $1\mathrm{e}{-5}$ 
not be used, rather CDS as the method 
for obtaining the gradients should not 
be used. This because, when no plateau can 
bee seen, there is no possibility to find 
a reasonable required step-size.\newline


Another conclusion which can be made is 
by comparing the direct and indirect 
results. The 
indirect method seems to be more 
stable within a broader step-size interval.
Also, this occurrence could be seen in 
for mission 2 for all tested shape parameters.
Note the scaling of the y-axis, which 
defined as 3 times the standard deviation. With this 
in mind the stability 
of the gradient by changing the step-size 
within the plateau becomes more tangible.\newline 

As summary, it can be stated that
the iterative indirect method was found 
to be more stable and choosing 
the step-size of $1\mathrm{e}{-5}$
could be proven to be optimal. This value
neither leads to numerical 
instabilities nor does it not comply 
with the accuracy requirements. Therefore,
the default method for solving 
the gradients with respect to the shape 
parameters is the indirect 
method with its default step-size of 
$1\mathrm{e}{-5}$.

\section{Surrogate models}
In this section different 
surrogate models, which 
are listed in section \ref{sec_Surrogates},
will be explored with regard to 
their respective options.
For generating the RBF surrogate model \emph{SciPy} was used 
with its default kernel multiquadtratic, which is given
in section \ref{subsec_Kriging_Surrogate} 
in equation \eqref{eq_34}. Therefore, a \emph{smooth}
value \cite{noauthor_rbf_2021} of $0.1$ was employed, which 
means regression is performed. Note for 
further reading, when the word RBF is
mentioned, it always means \emph{Scipy'}s RBF.
Kriging surrogate models 
were generated by invoking \emph{SMARTy}. Here the chosen default Kriging
kernel was 
Gaussian, Augmentation and regularization were set 
to $1$ and \emph{True}, respectively. 
The training samples 
for generating the surrogate models 
were provided  by the \emph{DLR}'s analysis and 
optimization workflow \emph{FSAerOpt} 
\cite{merle_high-fidelity_2019}. It is 
able to calculate fully trimmed states 
and consistent adjoint based gradients involving the 
flow solver \emph{TAU}. A total of $38$
training samples distributed in the 
$Ma,h,m$ space using the Halton Design Of 
Experiments (DOE) method, which 
are re depicted 
in the figures \ref{fig_56_Mass_Mach}
to \ref{fig_57_Mach_H},  were provided 
for this and the upcoming surrogate 
models studies.
\newline 

\begin{figure}[!h]
    %\vspace{0.5cm}
    \begin{minipage}[h]{0.46\textwidth}
        \centering
        \includegraphics[width =\textwidth]{2_Figures/3_Task/2_Interpol_Model/00.Sampling.MassHalfModel[kg].MachNumber[-].png}
        \caption{Halton DOE for half mass and Mach number ($m/2, \, Ma$)}
        \label{fig_56_Mass_Mach}    
    \end{minipage}
    \hfill
    \begin{minipage}{0.46\textwidth}
        \centering
        \includegraphics[width =\textwidth]{2_Figures/3_Task/2_Interpol_Model/01.Sampling.MassHalfModel[kg].Altitude[m].png}
        \caption{Halton DOE for half mass and altitude ($m/2, \, h$)}
        \label{fig_57_Mass_H}    
    \end{minipage}
\end{figure} 

\begin{figure}
    \centering
    \includegraphics[width =0.46\textwidth]{2_Figures/3_Task/2_Interpol_Model/02.Sampling.MachNumber[-].Altitude[m].png}
    \caption{Halton DOE for Mach number and altitude ($Ma, \, h$)}
    \label{fig_57_Mach_H}    
\end{figure}

\FloatBarrier
The input for the surrogate model is 3 dimensional, 
meaning Mach numbers, altitudes and masses 
$Ma, h, m$ are provided. With this 
set of input variables the desired predicted 
outcome for the output variables $LoD, AoA, TSFC$
are generated. Note that the output
variables are called $LoD, AoA, TSFC$, but 
their respective interpolated 
values are denoted as $\tilde{LoD}, \tilde{AoA}, \tilde{TSFC}$.
This means 3 different surrogate 
models 
are trained. Because of the 3 dimensional input 
and the 1 dimensional output, plotting the 
surrogate models would require 4 axes. Therefore, for 
showing  results, plots at constant
Mach numbers, altitudes and masses ($Ma, h, m$)
are going to be presented. Also note that the mass, when 
solving the ODE is equivalent to the cruise 
starting mass $m_s$. \newline

Since more research has been done than can 
be showed here explicitly with figures, only 
some results will be shown. This issue is understood 
better by highlighting the fact that the 
explorations are performed for state as  
well as the gradients. In case of the 
state surrogate models only 3 surrogate models are 
constructed. In case of the gradients,
$126$ shape parameter were given, thus 
$126*3 = 378$ surrogate models could be presented.
Additionally, to recall, only sections where 
of $Ma, h, m$ is constant can be visualized 
properly. Therefore, some 
results, which 
are supposed to summarize the findings, are chosen
to be displayed. The structure of presenting the results 
is as follows. On the left and right side the RBF and 
Kriging surrogate interpolation models are depicted 
for the same constant parameter of of $Ma, h, m$, respectively.
Therefore, interpolation models for $LoD, AoA, TSFC$
will be 
given at two different respective constant 
parameter values for $Ma, h, m$. In other words, 
for each input parameter $Ma, h, m$, 
one parameter is chosen to be constant 
at a specific value. This value is changed 
two times and isoplanes through 
the entire space of the 
interpolation models (RBF, Kriging) are sliced and shown 
next to each other. In total, 6 different constant 
values for each input parameter $Ma, h, m$ were
chosen. They are uniformly distributed between 
the lowest and highest values of the respective 
input parameter ($Ma, h, m$).\newline

% ------------------------------- 01 ----------------------------------------
For interpreting the figures, be reminded that the angle 
of attack is also often referred to as $AOA = \alpha$
and note that the y-axis is equally color-coded.
Figures \ref{fig_56} to \ref{fig_59} depict 
the interpolation models for the constant
altitude $10058.4$ m  and $11887.2$ m, for 
the output variables $LoD, AoA, TSFC$.
In figures \ref{fig_56} and \ref{fig_57} 
noticeable differences in the interpolation 
models for $AoA$ can be observed. The Kriging 
model on the right side of depicts a 
smoother transition than RBF. This observation 
can bee seen in all figures 
with a constant altitude from \ref{fig_56}
to \ref{fig_67}. This holds also, when varying 
the altitude or even the output variable.
In other words, each interpolation 
model for the different desired 
output variables $LoD, AoA, TSFC$ 
confirms this observation. A possible 
reason for this occurrence could be that the
Kriging value for 
regularization is automatically tuned in \emph{SMARTy} and can 
be higher 
than the value for \emph{SciPy}'s RBF.
The answer to the 
question, which models predict the underlying 
physical phenomenons accurately requires a deep understanding 
of the relationship of all the input and output 
parameters and can be partly answered in 
subsection \ref{subsec_Krig_TPS_Accu}\newline

% % ====================================================================
% % ================== Const H =============================================
% % =======================================================================
% % ------------------- AoA---------------------
\begin{figure}[!h]
    %\vspace{0.5cm}
    \begin{minipage}[h]{0.46\textwidth}
        \centering
        \includegraphics[width =\textwidth]{2_Figures/3_Task/2_Interpol_Model/Regular/RBF/Const_H/Mach_Mass_AoA_1.png}
        \caption{Surrogate interpolation model for $AoA$ at constant altitude $h = 10058.4$ [m], RBF}
        \label{fig_56}    
    \end{minipage}
    \hfill
    \begin{minipage}{0.46\textwidth}
        \centering
        \includegraphics[width =\textwidth]{2_Figures/3_Task/2_Interpol_Model/Regular/Kriging/Const_H/Mach_Mass_AoA_1.png}
        \caption{Surrogate interpolation model for $AoA$ at constant altitude $h = 10058.4$ [m], Kriging}
        \label{fig_57}    
    \end{minipage}
\end{figure} 

\begin{figure}[!h]
    %\vspace{0.5cm}
    \begin{minipage}[h]{0.46\textwidth}
        \centering
        \includegraphics[width =\textwidth]{2_Figures/3_Task/2_Interpol_Model/Regular/RBF/Const_H/Mach_Mass_AoA_6.png}
        \caption{Surrogate interpolation model for $AoA$ at constant altitude $h = 11887.2$ [m], RBF}
        \label{fig_58}    
    \end{minipage}
    \hfill
    \begin{minipage}{0.46\textwidth}
        \centering
        \includegraphics[width =\textwidth]{2_Figures/3_Task/2_Interpol_Model/Regular/Kriging/Const_H/Mach_Mass_AoA_6.png}
        \caption{Surrogate interpolation model for $AoA$ at constant altitude $h = 11887.2$ [m], Kriging}
        \label{fig_59}    
    \end{minipage}
\end{figure} 

% ------------------- TSFC---------------------
\begin{figure}[!h]
    %\vspace{0.5cm}
    \begin{minipage}[h]{0.46\textwidth}
        \centering
        \includegraphics[width =\textwidth]{2_Figures/3_Task/2_Interpol_Model/Regular/RBF/Const_H/Mach_Mass_TSFC_1.png}
        \caption{Surrogate interpolation model for $TSFC$ at constant altitude $h = 10058.4$ [m], RBF}
        \label{fig_60}    
    \end{minipage}
    \hfill
    \begin{minipage}{0.46\textwidth}
        \centering
        \includegraphics[width =\textwidth]{2_Figures/3_Task/2_Interpol_Model/Regular/Kriging/Const_H/Mach_Mass_TSFC_1.png}
        \caption{Surrogate interpolation model for $TSFC$ at constant altitude $h = 10058.4$ [m], Kriging}
        \label{fig_61}    
    \end{minipage}
\end{figure} 

\begin{figure}[!h]
    %\vspace{0.5cm}
    \begin{minipage}[h]{0.46\textwidth}
        \centering
        \includegraphics[width =\textwidth]{2_Figures/3_Task/2_Interpol_Model/Regular/RBF/Const_H/Mach_Mass_TSFC_6.png}
        \caption{Surrogate interpolation model for $TSFC$ at constant altitude $h = 11887.2$ [m], RBF}
        \label{fig_62}    
    \end{minipage}
    \hfill
    \begin{minipage}{0.46\textwidth}
        \centering
        \includegraphics[width =\textwidth]{2_Figures/3_Task/2_Interpol_Model/Regular/Kriging/Const_H/Mach_Mass_TSFC_6.png}
        \caption{Surrogate interpolation model for $TSFC$ at constant altitude $h = 11887.2$ [m], Kriging}
        \label{fig_63}    
    \end{minipage}
\end{figure} 

% ------------------- LoD---------------------

\begin{figure}[!h]
    %\vspace{0.5cm}
    \begin{minipage}[h]{0.46\textwidth}
        \centering
        \includegraphics[width =\textwidth]{2_Figures/3_Task/2_Interpol_Model/Regular/RBF/Const_H/Mach_Mass_LoD_1.png}
        \caption{Surrogate interpolation model for $LoD$ at constant altitude $h = 10058.4$ [m], RBF}
        \label{fig_64}    
    \end{minipage}
    \hfill
    \begin{minipage}{0.46\textwidth}
        \centering
        \includegraphics[width =\textwidth]{2_Figures/3_Task/2_Interpol_Model/Regular/Kriging/Const_H/Mach_Mass_LoD_1.png}
        \caption{Surrogate interpolation model for $LoD$ at constant altitude $h = 10058.4$ [m], Kriging}
        \label{fig_65}    
    \end{minipage}
\end{figure} 

\begin{figure}[!h]
    %\vspace{0.5cm}
    \begin{minipage}[h]{0.46\textwidth}
        \centering
        \includegraphics[width =\textwidth]{2_Figures/3_Task/2_Interpol_Model/Regular/RBF/Const_H/Mach_Mass_LoD_6.png}
        \caption{Surrogate interpolation model for $LoD$ at constant altitude $h = 11887.2$ [m], RBF}
        \label{fig_66}    
    \end{minipage}
    \hfill
    \begin{minipage}{0.46\textwidth}
        \centering
        \includegraphics[width =\textwidth]{2_Figures/3_Task/2_Interpol_Model/Regular/Kriging/Const_H/Mach_Mass_LoD_6.png}
        \caption{Surrogate interpolation model for $LoD$ at constant altitude $h = 11887.24$ [m], Kriging}
        \label{fig_67}    
    \end{minipage}
\end{figure} 




% % ====================================================================
% % ================== Ccnst Mach =============================================
% % =======================================================================

\FloatBarrier
The figures from \ref{fig_68} to \ref{fig_75} show the interpolations 
models, when a cut at constant Mach numbers is performed. 
Here the same effect as described above for 
the constant altitude case can be observed. Kriging 
is smoother in the curve of its predicted values 
than RBF is. However, in the case of constant Mach numbers,
one more phenomena can be observed. The actual 
values of for the desired output variables 
$LoD, AoA, TSFC$ are different when 
comparing RBF (left side) with Kriging (right side).
This effect is clearly visible, e.g. in figures
\ref{fig_72} and \ref{fig_73}.\newline

% % ------------------- AoA---------------------
\begin{figure}[!h]
    %\vspace{0.5cm}
    \begin{minipage}[h]{0.46\textwidth}
        \centering
        \includegraphics[width =\textwidth]{2_Figures/3_Task/2_Interpol_Model/Regular/RBF/Const_Mach/H_Mass_AoA_1.png}
        \caption{Surrogate interpolation model for $AoA$ at constant $Ma = 0.815$ [-], RBF}
        \label{fig_68}    
    \end{minipage}
    \hfill
    \begin{minipage}{0.46\textwidth}
        \centering
        \includegraphics[width =\textwidth]{2_Figures/3_Task/2_Interpol_Model/Regular/Kriging/Const_Mach/H_Mass_AoA_1.png}
        \caption{Surrogate interpolation model for $AoA$ at constant $Ma = 0.815$ [-], Kriging}
        \label{fig_69}    
    \end{minipage}
\end{figure} 

\begin{figure}[!h]
    %\vspace{0.5cm}
    \begin{minipage}[h]{0.46\textwidth}
        \centering
        \includegraphics[width =\textwidth]{2_Figures/3_Task/2_Interpol_Model/Regular/RBF/Const_Mach/H_Mass_AoA_6.png}
        \caption{Surrogate interpolation model for $AoA$ at constant $Ma = 0.845$ [-], RBF}
        \label{fig_70}    
    \end{minipage}
    \hfill
    \begin{minipage}{0.46\textwidth}
        \centering
        \includegraphics[width =\textwidth]{2_Figures/3_Task/2_Interpol_Model/Regular/Kriging/Const_Mach/H_Mass_AoA_6.png}
        \caption{Surrogate interpolation model for $AoA$ at constant $Ma = 0.845$ [-], Kriging}
        \label{fig_71}    
    \end{minipage}
\end{figure} 


% % ------------------- LoD---------------------

\begin{figure}[!h]
    %\vspace{0.5cm}
    \begin{minipage}[h]{0.46\textwidth}
        \centering
        \includegraphics[width =\textwidth]{2_Figures/3_Task/2_Interpol_Model/Regular/RBF/Const_Mach/H_Mass_LoD_1.png}
        \caption{Surrogate interpolation model for $LoD$ at constant $Ma = 0.815$ [-], RBF}
        \label{fig_72}    
    \end{minipage}
    \hfill
    \begin{minipage}{0.46\textwidth}
        \centering
        \includegraphics[width =\textwidth]{2_Figures/3_Task/2_Interpol_Model/Regular/Kriging/Const_Mach/H_Mass_LoD_1.png}
        \caption{Surrogate interpolation model for $LoD$ at constant $Ma = 0.815$ [-], Kriging}
        \label{fig_73}    
    \end{minipage}
\end{figure} 

\begin{figure}[!h]
    %\vspace{0.5cm}
    \begin{minipage}[h]{0.46\textwidth}
        \centering
        \includegraphics[width =\textwidth]{2_Figures/3_Task/2_Interpol_Model/Regular/RBF/Const_Mach/H_Mass_LoD_6.png}
        \caption{Surrogate interpolation model for $LoD$ at constant $Ma = 0.845$ [-], RBF}
        \label{fig_74}    
    \end{minipage}
    \hfill
    \begin{minipage}{0.46\textwidth}
        \centering
        \includegraphics[width =\textwidth]{2_Figures/3_Task/2_Interpol_Model/Regular/Kriging/Const_Mach/H_Mass_LoD_6.png}
        \caption{Surrogate interpolation model for $LoD$ at constant $Ma = 11887.24$ [m], Kriging}
        \label{fig_75}    
    \end{minipage}
\end{figure} 


% % ====================================================================
% % ================== Const Mass =============================================
% % =======================================================================
\FloatBarrier
The figures from \ref{fig_76} to \ref{fig_87} show the interpolations 
models, when a cut at constant mass is made. 
These are slices, where the RBF plots (left side)
for the output variables $LoD,AoA, TSFC$
are also smooth. 
However, Kriging (right side) remains 
smooth, which suggests, that RBF and Kriging trends
must not necessarily be contradicting.
The second difference 
with regard to the constant Mach number and altitude slices is, 
that comparing
RBF (left side) and Kriging (right side) small 
noticeable differences in the values for the output variables 
$LoD, AoA, TSFC$ in the overall space can be observed.
 This can be seen, e.g. by 
comparing figure \ref{fig_78} and \ref{fig_79}. 
Consider the upper left corners and the whole 
upper edges. Here the clear contrast 
in the colors and thus in the values 
for $AoA$ can bee seen. These  differences, can 
also become extreme when comparing figures \ref{fig_86}
and \ref{fig_87}. 
Employing RBF the values for $LoD$ are all 
greater or equal to $LoD \geq 21.046$. In contrast, 
the Kriging model (right side) also exhibits 
$LoD$ values which are around $19$. This highlights, 
the interpolation for $LoD$ highly 
depends on the underlying surrogate interpolation model.\newline

% % ------------------- AoA---------------------
\begin{figure}[!h]
    %\vspace{0.5cm}
    \begin{minipage}[h]{0.46\textwidth}
        \centering
        \includegraphics[width =\textwidth]{2_Figures/3_Task/2_Interpol_Model/Regular/RBF/Const_Mass/Mach_H_AoA_1.png}
        \caption{Surrogate interpolation model for $AoA$ at constant mass $m = 171000$ [kg], RBF}
        \label{fig_76}    
    \end{minipage}
    \hfill
    \begin{minipage}{0.46\textwidth}
        \centering
        \includegraphics[width =\textwidth]{2_Figures/3_Task/2_Interpol_Model/Regular/Kriging/Const_Mass/Mach_H_AoA_1.png}
        \caption{Surrogate interpolation model for $AoA$ at constant mass $m = 171000$ [kg], Kriging}
        \label{fig_77}    
    \end{minipage}
\end{figure} 

\begin{figure}[!h]
    %\vspace{0.5cm}
    \begin{minipage}[h]{0.46\textwidth}
        \centering
        \includegraphics[width =\textwidth]{2_Figures/3_Task/2_Interpol_Model/Regular/RBF/Const_Mass/Mach_H_AoA_6.png}
        \caption{Surrogate interpolation model for $AoA$ at constant mass $m = 245000$ [kg], RBF}
        \label{fig_78}    
    \end{minipage}
    \hfill
    \begin{minipage}{0.46\textwidth}
        \centering
        \includegraphics[width =\textwidth]{2_Figures/3_Task/2_Interpol_Model/Regular/Kriging/Const_Mass/Mach_H_AoA_6.png}
        \caption{Surrogate interpolation model for $AoA$ at constant mass $m = 245000$ [kg], Kriging}
        \label{fig_79}    
    \end{minipage}
\end{figure} 

% % ------------------- TSFC---------------------
\begin{figure}[!h]
    %\vspace{0.5cm}
    \begin{minipage}[h]{0.46\textwidth}
        \centering
        \includegraphics[width =\textwidth]{2_Figures/3_Task/2_Interpol_Model/Regular/RBF/Const_Mass/Mach_H_TSFC_1.png}
        \caption{Surrogate interpolation model for $TSFC$ at constant mass $m = 171000$ [kg], RBF}
        \label{fig_80}    
    \end{minipage}
    \hfill
    \begin{minipage}{0.46\textwidth}
        \centering
        \includegraphics[width =\textwidth]{2_Figures/3_Task/2_Interpol_Model/Regular/Kriging/Const_Mass/Mach_H_TSFC_1.png}
        \caption{Surrogate interpolation model for $TSFC$ at constant mass $m = 171000$ [kg], Kriging}
        \label{fig_81}    
    \end{minipage}
\end{figure} 

\begin{figure}[!h]
    %\vspace{0.5cm}
    \begin{minipage}[h]{0.46\textwidth}
        \centering
        \includegraphics[width =\textwidth]{2_Figures/3_Task/2_Interpol_Model/Regular/RBF/Const_Mass/Mach_H_TSFC_6.png}
        \caption{Surrogate interpolation model for $TSFC$ at constant mass $m = 245000$ [kg], RBF}
        \label{fig_82}    
    \end{minipage}
    \hfill
    \begin{minipage}{0.46\textwidth}
        \centering
        \includegraphics[width =\textwidth]{2_Figures/3_Task/2_Interpol_Model/Regular/Kriging/Const_Mass/Mach_H_TSFC_6.png}
        \caption{Surrogate interpolation model for $TSFC$ at constant mass $m = 245000$ [kg], Kriging}
        \label{fig_83}    
    \end{minipage}
\end{figure} 

% ------------------- LoD---------------------

\begin{figure}[!h]
    %\vspace{0.5cm}
    \begin{minipage}[h]{0.46\textwidth}
        \centering
        \includegraphics[width =\textwidth]{2_Figures/3_Task/2_Interpol_Model/Regular/RBF/Const_Mass/Mach_H_LoD_1.png}
        \caption{Surrogate interpolation model for $LoD$ at constant mass $m = 171000$ [kg], RBF}
        \label{fig_84}    
    \end{minipage}
    \hfill
    \begin{minipage}{0.46\textwidth}
        \centering
        \includegraphics[width =\textwidth]{2_Figures/3_Task/2_Interpol_Model/Regular/Kriging/Const_Mass/Mach_H_LoD_1.png}
        \caption{Surrogate interpolation model for $LoD$ at constant mass $m = 171000$ [kg], Kriging}
        \label{fig_85}    
    \end{minipage}
\end{figure} 

\begin{figure}[!h]
    %\vspace{0.5cm}
    \begin{minipage}[h]{0.46\textwidth}
        \centering
        \includegraphics[width =\textwidth]{2_Figures/3_Task/2_Interpol_Model/Regular/RBF/Const_Mass/Mach_H_LoD_6.png}
        \caption{Surrogate interpolation model for $LoD$ at constant mass $m = 245000$ [kg], RBF}
        \label{fig_86}    
    \end{minipage}
    \hfill
    \begin{minipage}{0.46\textwidth}
        \centering
        \includegraphics[width =\textwidth]{2_Figures/3_Task/2_Interpol_Model/Regular/Kriging/Const_Mass/Mach_H_LoD_6.png}
        \caption{Surrogate interpolation model for $LoD$ at constant mass $Ma = 11887.24$ [m], Kriging}
        \label{fig_87}    
    \end{minipage}
\end{figure} 


\FloatBarrier
This section can be concluded with the following.
Clearly, there are differences in the outcome 
of the predicted values and trends for the desired 
output variables $LoD, TSFC, AOA$. As a
consequence, once more it can be said that the 
choice of the surrogate model heavily determines 
the results of any further calculation.
\section{Evaluating different surrogate models}
\label{sec_Diff_Surrogates}
In this section the different surrogate models 
Kriging (Gaussian), TPS and RBF are 
explored, which are all described 
in section \ref{sec_Surrogates}. Since most 
explanations for the upcoming explorations 
were given in subsection \ref{subsec_INvestigate_Surro}, 
only necessary details will be repeated.
The two missions which are going 
to be used for the following investigations are 
shown in table \ref{tab_2}.
Both missions passed 
the constraints check 
violation successfully and thus 
the number of the fix point iterations 
for calculating the gradients is low.
The table \ref{tab_7} shows the 
options for Kriging and TPS and 
\emph{SciPy}s RBF. For all \emph{SciPy}s RBF 
kernels a \emph{smooth} value of 
$0.1$ \cite{noauthor_rbf_2021} was used.  More
information about the settings and kernel 
is given in section \ref{sec_Surrogates}. \newline 

%
\begin{table}[!h]
        \centering
        \begin{tabular}{l c c |l }
            \multicolumn{3}{c|}{\textbf{\emph{SMARTy} 
            Kriging and TPS}} & 
            \multicolumn{1}{l}{\textbf{\emph{SciPy}}} \\
            \hline
%           ------------ Second Title -----------------------
            \multicolumn{1}{l}{\textbf{Name}} & 
            \multicolumn{1}{c}{\textbf{Augmentation}} &
            \multicolumn{1}{c}{\textbf{Regularization}} &
            \multicolumn{1}{|c}{\textbf{RBF Kernel}}\\
            \hline
            A & $-1$ &False &  multiquadric       \\
            B & $0$ &False  & inverse      \\
            C & $+1$ &False &  gaussian        \\
            D & $+2$ &False &  linear       \\
            E & $-1$ &True  &  cubic   \\
            F & $0$ &True   &  thin plate   \\
            G & $+1$ &True  &   -  \\
            H & $+2$ &True  &  -   \\
        \end{tabular}
        \caption{Kriging, TPS and RBF parameter used for investigation 
        surrogate quality}
        \label{tab_7}
\end{table} 
%
\FloatBarrier
Similarly to the results sections before, much 
research has been done. However, not 
all results can be shown, rather 
results which are supposed to sum the 
found results up will be shown and elaborated.
Recall, 35 effective or fully trimmed sample points 
were given as the input data for training the surrogate 
models. The investigations were performed 
for state and gradients as well. In 
subsection \ref{subsec_INvestigate_Surro} 
three different versions were introduced, \emph{V38,V28}
and \emph{V18}, where each version only effectively
have three less sample points than its declarations name.
According to subsection 
\ref{subsec_INvestigate_Surro} for V38 the
\emph{Leave-one-out cross-validation} is applied. Whereas 
for V28 and V18 the \emph{Root Mean Square Error} (RMSE)
is calculated by having 10 and 20 additional 
sample points, respectively. For
this additional sample points the true value of output variables 
$AoA, LoD, TSFC$ and their gradients is known.\newline


The first exploration, which will be considered is 
by using the 38 sample data (effective 35 trimmed samples).
For this V38, the total missions fuel mass $m_f$, 
the cruise fuel weight $m_{cr}$ and the cruise 
starting mass $m_s$ were monitored. However, 
since only for the cruise segment 
a physical based equation is applied, it is sufficient 
to only portray the curve of the total 
fuel mass $m_f$ with the different 
surrogate models.
As concluded in section \ref{sec_Step_Size_Study}, the indirect 
method is preferred to the direct method. Therefore, 
the indirect method is used in order to perform this 
surrogate investigation.
Figure \ref{fig_88} and \ref{fig_89} exhibit the total 
fuel mass $m_f$ of mission 1, by 
employing Kriging and TPS, respectively. 
Figures \ref{fig_90} and \ref{fig_91} show
the same for mission 2.
The 
horizontal axis gives information about the chosen 
model intern parameter, which can be read from table \ref{tab_7}.
In both cases it can be observed, 
changing the model intern parameter does have an impact 
of the outcome ($m_{f}$). For both surrogate models
with their different options the deviation is less than 
$260$ kilogram. In case of Kriging and TPS in figures \ref{fig_88}
and \ref{fig_91}, respectively it is only $25$ kilogram. Such a deviation
is considered to be acceptable.

\begin{figure}[!h]
    %\vspace{0.5cm}
    \begin{minipage}[h]{0.46\textwidth}
        \centering
        \includegraphics[width =\textwidth]{2_Figures/3_Task/3_Surrogate/1_Fleiss/1_Krig/1_Miss_1_Final.pdf}
        \caption{Final mission fuel weight $m_f$, Mission 1, Kriging variants}
        \label{fig_88}    
    \end{minipage}
    \hfill
    \begin{minipage}{0.46\textwidth}
        \centering
        \includegraphics[width =\textwidth]{2_Figures/3_Task/3_Surrogate/1_Fleiss/2_TPS/1_Miss_1_Final.pdf}
        \caption{Final mission fuel weight $m_f$, Mission 1, TPS variants}
        \label{fig_89}    
    \end{minipage}
\end{figure} 


\begin{figure}[!h]
    %\vspace{0.5cm}
    \begin{minipage}[h]{0.46\textwidth}
        \centering
        \includegraphics[width =\textwidth]{2_Figures/3_Task/3_Surrogate/1_Fleiss/1_Krig/1_Miss_2_Final.pdf}
        \caption{Final mission fuel weight $m_f$, Mission 2, Kriging variants}
        \label{fig_90}    
    \end{minipage}
    \hfill
    \begin{minipage}{0.46\textwidth}
        \centering
        \includegraphics[width =\textwidth]{2_Figures/3_Task/3_Surrogate/1_Fleiss/2_TPS/1_Miss_2_Final.pdf}
        \caption{Final mission fuel weight $m_f$, Mission 2, TPS variants}
        \label{fig_91}    
    \end{minipage}
\end{figure} 

 \FloatBarrier
Figures \ref{fig_92} to \ref{fig_95} exhibits the number of 
the fix point iterations, which were required for 
only mass calculation for the states, as explained 
in section \ref{sec_Fuel_Mass_iter}. It can be 
seen that in no event more than 4 state fix point 
iterations are required. The difference between the 
gradient and the state fix point iterations can 
be simplified as follows. The gradient fix point iterations number 
is approximately twice as high as the state version. This is so, because 
the CDS evaluates the functions 
value at two points, thus the state fix point iteration 
is run twice.\newline

Figures \ref{fig_96} to \ref{fig_99} depict the same 
kind of study for RBF and its options.
It can be observed, that RBF predictions highly depend on the chosen 
kernel. Also, the number of the state fix point iterations are much 
higher.
Figures \ref{fig_100} and \ref{fig_101} show a comparison 
with all surrogate models and their options for 
mission 1 and 2, respectively. It can be seen, that 
only in case of mission 1, depending 
on RBF's kernel a match with Kriging and TPS 
occurs. In general, it can be stated that the
\emph{SciPy} RBF's output is noticeably different to 
\emph{SMARTy}'s Krigings and 
TPS output. However, Krigings and TPS output 
exhibit a comparatively low deviation. Furthermore, 
it can be said that RBF has a much higher demand of 
state fix point iterations, which makes is 
unfeasible for application purposes.

\FloatBarrier
 \begin{figure}[!h]
    %\vspace{0.5cm}
    \begin{minipage}[h]{0.46\textwidth}
        \centering
        \includegraphics[width =\textwidth]{2_Figures/3_Task/3_Surrogate/1_Fleiss/1_Krig/4_Miss_1_Nr_Iter.pdf}
        \caption{Number of the state fix-point iterations, Mission 1, Kriging variants}
        \label{fig_92}    
    \end{minipage}
    \hfill
    \begin{minipage}{0.46\textwidth}
        \centering
        \includegraphics[width =\textwidth]{2_Figures/3_Task/3_Surrogate/1_Fleiss/2_TPS/4_Miss_1_Nr_Iter.pdf}
        \caption{Number of the state fix-point, Mission 1, TPS variants}
        \label{fig_93}    
    \end{minipage}
\end{figure} 

\begin{figure}[!h]
    %\vspace{0.5cm}
    \begin{minipage}[h]{0.46\textwidth}
        \centering
        \includegraphics[width =\textwidth]{2_Figures/3_Task/3_Surrogate/1_Fleiss/1_Krig/4_Miss_2_Nr_Iter.pdf}
        \caption{Number of the state fix-point iterations, Mission 2, Kriging variants}
        \label{fig_94}    
    \end{minipage}
    \hfill
    \begin{minipage}{0.46\textwidth}
        \centering
        \includegraphics[width =\textwidth]{2_Figures/3_Task/3_Surrogate/1_Fleiss/2_TPS/4_Miss_2_Nr_Iter.pdf}
        \caption{Number of the state fix-point, Mission 2, TPS variants}
        \label{fig_95}    
    \end{minipage}
\end{figure} 

% -------------------- RBF STATE -------------------------------



\begin{figure}[!h]
    %\vspace{0.5cm}
    \begin{minipage}[h]{0.46\textwidth}
        \centering
        \includegraphics[width =\textwidth]{2_Figures/3_Task/3_Surrogate/1_Fleiss/3_RBF/1_Miss_1_Final.pdf}
        \caption{Final mission fuel weight $m_f$, Mission 1, RBF variants}
        \label{fig_96}    
    \end{minipage}
    \hfill
    \begin{minipage}{0.46\textwidth}
        \centering
        \includegraphics[width =\textwidth]{2_Figures/3_Task/3_Surrogate/1_Fleiss/3_RBF/1_Miss_2_Final.pdf}
        \caption{Final mission fuel weight $m_f$, Mission 2, RBF variants}
        \label{fig_97}    
    \end{minipage}
\end{figure} 


\begin{figure}[!h]
    %\vspace{0.5cm}
    \begin{minipage}[h]{0.46\textwidth}
        \centering
        \includegraphics[width =\textwidth]{2_Figures/3_Task/3_Surrogate/1_Fleiss/3_RBF/4_Miss_1_Nr_Iter.pdf}
        \caption{Number of state fix point iterations, Mission 1, Kriging variants}
        \label{fig_98}    
    \end{minipage}
    \hfill
    \begin{minipage}{0.46\textwidth}
        \centering
        \includegraphics[width =\textwidth]{2_Figures/3_Task/3_Surrogate/1_Fleiss/3_RBF/4_Miss_2_Nr_Iter.pdf}
        \caption{Number of state fix point iterations, Mission 2, TPS variants}
        \label{fig_99}    
    \end{minipage}
\end{figure} 

\begin{figure}[!h]
    %\vspace{0.5cm}
    \begin{minipage}[h]{0.46\textwidth}
        \centering
        \includegraphics[width =\textwidth]{2_Figures/3_Task/3_Surrogate/1_Fleiss/4_Compare/2_All/1_Miss_1_Final.pdf}
        \caption{Total fuel mass $m_{f}$ for all surrogate models, Mission 1}
        \label{fig_100}    
    \end{minipage}
    \hfill
    \begin{minipage}{0.46\textwidth}
        \centering
        \includegraphics[width =\textwidth]{2_Figures/3_Task/3_Surrogate/1_Fleiss/4_Compare/2_All/1_Miss_2_Final.pdf}
        \caption{Total fuel mass $m_{f}$ for all surrogate models, Mission 2}
        \label{fig_101}    
    \end{minipage}
\end{figure} 

\FloatBarrier
The next step is to consider the effect of a decreasing number 
of training samples and its effect on the surrogate models.
This is done only for Kriging and TPS, since 
\emph{SciPy}'s RBF showed 
a too high sensitivity on the chosen kernel and also 
required too many fix point iterations. 
Kriging and TPS have the same 8 options. 
The total fuel mass $m_{f}$ is plotted 
in the figures \ref{fig_102} to \ref{fig_105}. 
Comparing  $m_{f}$ for mission 1 
with Kriging 
and TPS via figures \ref{fig_102} and \ref{fig_103}, respectively, 
the following can be observed. The vertical scaling axis 
of Kriging is finer, which means, a high change on the left 
figure results in a low change on the right figure (TPS).
The maximum difference in the version which could occur 
for Kriging is $350$ kg and for TPS is $1600$ kg. Therefore, 
the absolute deviation needs to be calculated, which are 
shown in figures \ref{fig_104} and \ref{fig_105} for 
Kriging and TPS, respectively. In the mentioned figures, 
the green and red curves are more interesting to us, since 
it is assumed that V38 is the most accurate version and 
thus is considered as the basis. For mission 1, Kriging 
V18 and V28 are clearly closer to V38. In other words, 
Kriging tends to deliver with fewer data 
results closer to the fine sampling outputs than TPS. In case 
Kriging would also be the most accurate method
(see subsection \ref{subsec_Krig_TPS_Accu}), it 
could be preferred for training surrogate models 
with a fewer number of training data.

\begin{figure}[!h]
    %\vspace{0.5cm}
    \begin{minipage}[h]{0.46\textwidth}
        \centering
        \includegraphics[width =\textwidth]{2_Figures/3_Task/3_Surrogate/5_Fleiss_V_All/1_Regular/1_Mission_1_Krig_Fuel_Weight.pdf}
        \caption{Total fuel mass $m_{f}$ for all Versions, Mission 1, Kriging}
        \label{fig_102}    
    \end{minipage}
    \hfill
    \begin{minipage}{0.46\textwidth}
        \centering
        \includegraphics[width =\textwidth]{2_Figures/3_Task/3_Surrogate/5_Fleiss_V_All/1_Regular/3_Mission_1_TPS_Fuel_Weight.pdf}
        \caption{Total fuel mass $m_{f}$ for all Versions, Mission 1, TPS}
        \label{fig_103}    
    \end{minipage}
\end{figure} 

\begin{figure}[!h]
    %\vspace{0.5cm}
    \begin{minipage}[h]{0.46\textwidth}
        \centering
        \includegraphics[width =\textwidth]{2_Figures/3_Task/3_Surrogate/5_Fleiss_V_All/1_Regular/2_Mission_1_Delta_Krig_Fuel_Weight.pdf}
        \caption{ Deviations of the different versions of $m_{f}$, Mission 1, Kriging}
        \label{fig_104}    
    \end{minipage}
    \hfill
    \begin{minipage}{0.46\textwidth}
        \centering
        \includegraphics[width =\textwidth]{2_Figures/3_Task/3_Surrogate/5_Fleiss_V_All/1_Regular/4_Mission_1_Delta_TPS_Fuel_Weight.pdf}
        \caption{ Deviations of the different versions of $m_{f}$, Mission 1, TPS}
        \label{fig_105}    
    \end{minipage}
\end{figure} 



\FloatBarrier
The same also must be tested for mission 2. The results are 
presented in figures \ref{fig_106} to \ref{fig_109}. 
The observations made for mission 1 are confirmed by having 
investigated on mission 2.

\begin{figure}[!h]
    %\vspace{0.5cm}
    \begin{minipage}[h]{0.46\textwidth}
        \centering
        \includegraphics[width =\textwidth]{2_Figures/3_Task/3_Surrogate/5_Fleiss_V_All/1_Regular/1_Mission_2_Krig_Fuel_Weight.pdf}
        \caption{Total fuel mass $m_{f}$ for all Versions, Mission 2, Kriging}
        \label{fig_106}    
    \end{minipage}
    \hfill
    \begin{minipage}{0.46\textwidth}
        \centering
        \includegraphics[width =\textwidth]{2_Figures/3_Task/3_Surrogate/5_Fleiss_V_All/1_Regular/3_Mission_2_TPS_Fuel_Weight.pdf}
        \caption{Total fuel mass $m_{f}$ for all Versions, Mission 2, TPS}
        \label{fig_107}    
    \end{minipage}
\end{figure} 

\begin{figure}[!h]
    %\vspace{0.5cm}
    \begin{minipage}[h]{0.46\textwidth}
        \centering
        \includegraphics[width =\textwidth]{2_Figures/3_Task/3_Surrogate/5_Fleiss_V_All/1_Regular/2_Mission_2_Delta_Krig_Fuel_Weight.pdf}
        \caption{Deviations of the different versions of $m_{f}$, Mission 2, Kriging}
        \label{fig_108}    
    \end{minipage}
    \hfill
    \begin{minipage}{0.46\textwidth}
        \centering
        \includegraphics[width =\textwidth]{2_Figures/3_Task/3_Surrogate/5_Fleiss_V_All/1_Regular/4_Mission_2_Delta_TPS_Fuel_Weight.pdf}
        \caption{Deviations of the different versions of $m_{f}$, Mission 2, TPS}
        \label{fig_109}    
    \end{minipage}
\end{figure} 

\FloatBarrier
Up to now, only state solutions were discussed. However, the whole 
investigation process was also done for the gradients.
Since the gradient investigation does not lead to
further findings, only 
the most important results will be shown. Furthermore,
only mission 1 is displayed as a representation of both missions
The figures \ref{fig_110} and \ref{fig_111} 
depict the mean curve of the gradient 
of the total fuel masses w.r.t. all shape parameter.
Mean curve means, that the results of the 8 options 
were used for calculating their respective mean values.\newline

\begin{figure}[!h]
    %\vspace{0.5cm}
    \begin{minipage}[h]{0.46\textwidth}
        \centering
        \includegraphics[width =\textwidth]{2_Figures/3_Task/3_Surrogate/5_Fleiss_V_All/2_Grad/1_Mean_Miss_1_Grad_Krig.pdf}
        \caption{Total fuel mass gradients w.r.t. shape parameters, All versions, Mission 1, Kriging}
        \label{fig_110}    
    \end{minipage}
    \hfill
    \begin{minipage}{0.46\textwidth}
        \centering
        \includegraphics[width =\textwidth]{2_Figures/3_Task/3_Surrogate/5_Fleiss_V_All/2_Grad/2_Mean_Miss_1_Grad_TPS.pdf}
        \caption{Total fuel mass gradients w.r.t. shape parameters, All versions, Mission 1, TPS}
        \label{fig_111}    
    \end{minipage}
\end{figure} 

\FloatBarrier
Next, the required execution time by having chosen 
the different surrogate models is going to be 
displayed. It is measured 
right after invoking \emph{missioninformer} till
the state and gradients for both missions were computed entirely
and stored to the hard disk. In other words, it is 
the execution time of \emph{missioninformer} itself, which 
is measured with different surrogate models.
The results are depicted in figure \ref{fig_112}.
It can be observed that the Kriging versions 
are also more stable in the run time 
than TPS. The impact of different Kriging 
options does not have high impact of 
its execution time. TPS on the other hand, 
can be a little faster than Kriging with some options, however
also can be much slower with other options.
Also, depending on the sample data size, 
a high deviation can be observed, when 
regularization is activated (E to H).


\begin{figure}[!h]
        \centering
        \includegraphics[width =0.8\textwidth]{2_Figures/3_Task/3_Surrogate/5_Fleiss_V_All/2_Grad/7_Exec_Time.pdf}
        \caption{Missioninformers execution time with different surrogate models}
        \label{fig_112}    
\end{figure} 

\FloatBarrier
\subsection{Kriging and TPS accuracy}
\label{subsec_Krig_TPS_Accu}
The objective of this subchapter is to 
present quantifiable information about the accuracy
of Kriging and TPS models 
depending on their user-defined options. The question is, 
which model and which option leads to the 
lowest RMSE. The answer 
to this question is found using 
\emph{SMARTy}'s automated model 
selection capability. The workflow for this 
purpose was repeated shortly in the introduction 
of this section and is given in more detail 
in subsection \ref{subsec_INvestigate_Surro}.
Therefore, the results shall be 
presented straight forward. On the left 
side the best option set for the state 
is given. Here also the declaration 
of the 16 tested models is provided, which 
is valid for the gradient investigation, depicted
on the right side, as well.
For the state calculation, only three 
surrogate models $\tilde{LoD}, \tilde{AoA}, \tilde{TSFC}$
are generated. Therefore, out of the total 16 
options for each of the three different 
surrogate models, only one offers the least 
RMSE. Therefore, for the states, each option 
only once can be found to be the best.
In contrast, the gradient has 
multiple shape parameters and each shape parameter 
has his own best option. Due to this, one 
option can be found to have the least value of 
RMSE multiple times per associated output 
parameter $LoD, AoA, TSFC$.\newline

The figure \ref{fig_113}, \ref{fig_115} and 
\ref{fig_117} depict the state version for 
V18, V28 and V38 respectively. The figures 
\ref{fig_114}, \ref{fig_116} and \ref{fig_118} 
their respective gradient version. Each figure 
on the left side contains a legend, which 
is valid for both sides. A clear distriction 
between Kriging and TPS models can be
made with the letters
\emph{A - H} and \emph{I-P}, respectively.
 Unfortunately,
no absolutely clear winner can be filtered out of these 
results. The distributions are too different for this purpose.
However, a suggestion can still be  made by 
starting with  options, which should not be 
taken into consideration.
Viewing all the gradient results (right side), 
options \emph{C, I, J, K} and \emph{L} show 
the least number of best RMSE values.
Also, the same options were not found 
to be the best choice for the states. 
The option \emph{A} is also not to be found 
as the best choice for states nor 
does it have a good result for the 
gradients for V18 and V28.  
Option \emph{A} is clearly no suggestion, but 
does not perform as bad as the other 
options mentioned before. The remaining 
possible options are: 
\emph{B, D, E, F, G, H, L, M, N, O} and 
\emph{P}. From these options 
\emph{F, H, M} and \emph{N} exhibit the 
best state results. These are 2 Kriging and 2 TPS models.
3 out of these 4 models include regularization and 
have the augmentation value of $1$. From here on, the most 
appealing option by including the gradients, 
is option \emph{N}.  
Furthermore, 
figures \ref{fig_105} 
and \ref{fig_109} should be considered 
in this examination. They  
depict the deviation for the 
different versions (V18, V28, V38) 
for \emph{SMARTy}'s TPS with 
an augmentation of $-1$. For both 
missions the deviation is high. This  
option 
is declared as \emph{J} in 
the figures \ref{fig_113} and 
\ref{fig_118} and also here, 
unsatisfactory results are found. \newline

In conclusion, it can be said, no real winner 
could be found out for the options. However, 
using regularization and an augmentation 
of value $1$ can be recommended. If a final 
decision had to be made, option \emph{N},
the TPS model with regularization
and an augmentation of $1$ should be chosen.
Note that performing this model selection study 
took more than 3 days on the workstation described 
in section \ref{sec_Motivation}.
This means performing
model selection for each new set of training samples 
within an aerodynamic optimization 
is not feasible. Therefore, TPS with 
active regularization and an augmentation 
of $1$ is set as the default model 
in \emph{missioninformer}.

\begin{figure}[!h]
    %\vspace{0.5cm}
    \begin{minipage}[h]{0.46\textwidth}
        \centering
        \includegraphics[width =\textwidth]{2_Figures/3_Task/3_Surrogate/6_Best_Models/state_Model_18.pdf}
        \caption{Lowest RMSE by having compared Krigings and TPS 8 options, state V18}
        \label{fig_113}    
    \end{minipage}
    \hfill
    \begin{minipage}{0.46\textwidth}
        \centering
        \includegraphics[width =\textwidth]{2_Figures/3_Task/3_Surrogate/6_Best_Models/grad_Model_18.pdf}
        \caption{Lowest RMSE by having compared Krigings and TPS 8 options, gradient V18}
        \label{fig_114}    
    \end{minipage}
    \vspace{0.5cm}
    % ------------------------------------ V28 ----------------
    \begin{minipage}[h]{0.46\textwidth}
        \centering
        \includegraphics[width =\textwidth]{2_Figures/3_Task/3_Surrogate/6_Best_Models/state_Model_28.pdf}
        \caption{Lowest RMSE by having compared Krigings and TPS 8 options, state V28}
        \label{fig_115}    
    \end{minipage}
    \hfill
    \begin{minipage}{0.46\textwidth}
        \centering
        \includegraphics[width =\textwidth]{2_Figures/3_Task/3_Surrogate/6_Best_Models/grad_Model_28.pdf}
        \caption{Lowest RMSE by having compared Krigings and TPS 8 options, gradient V28}
        \label{fig_116}    
    \end{minipage}
    \vspace{0.5cm}
    % ------------------------------------ V38 ----------------
    \begin{minipage}[h]{0.46\textwidth}
        \centering
        \includegraphics[width =\textwidth]{2_Figures/3_Task/3_Surrogate/6_Best_Models/state_Model_38.pdf}
        \caption{Lowest RMSE by having compared Krigings and TPS 8 options, state V38}
        \label{fig_117}    
    \end{minipage}
    \hfill
    \begin{minipage}{0.46\textwidth}
        \centering
        \includegraphics[width =\textwidth]{2_Figures/3_Task/3_Surrogate/6_Best_Models/grad_Model_38.pdf}
        \caption{Lowest RMSE by having compared Krigings and TPS 8 options, gradient V38}
        \label{fig_118}    
    \end{minipage}
\end{figure} 



% % ---------------- Task 4 ------------------------------
\chapter{Discussion}
The \emph{missioninformer}, a tool, was 
proposed to  help to fulfill the  
environmental demands for reducing air traffic pollution. Clearly, the aircraft 
sector contributed and contributes 
to the current environmental crisis which we 
are facing. Since an aircraft has become 
indispensable, not only for the individual, 
but also for the global economy, the solution 
for a healthier environment cannot be 
banning aircraft, but rather improving it 
with regard to CO2, NOx and SO2 emissions. 
The \emph{missioninformer}, a lightweight tool was 
written from scratch. 
Its major task is 
to calculate the consumed fuel for
one or multiple individual 
missions for one aircraft. For the most 
important part of the mission, the cruise segment, 
physics-based equations are applied. In section \ref{sec_Solve_ODE}
they are introduced, the precision 
with which they are solved is satisfactory and 
is presented in section \ref{sec_Solve_ODE_Methods}.
They are 
fed with interpolated aerodynamic data, where 
the interpolation models were generated 
by using high fidelity RANS-based training samples.\newline


The next step \emph{missioninformer} 
went, was also to include the remaining 
flight segments by employing fuel fractions.
All the used fuel fractions 
were given in section \ref{sec_Cruise_Mission_Fuel}
in table \ref{tab_1}. 
At this stage, the \emph{missioninformer} 
can predict the fuel required for a whole 
flight mission. For real world applications, 
the user must be able to define the flown mission 
himself. The \emph{missioninformer} meets that need 
and in a structured way, so less effort is required, 
which is described in section \ref{sec_introduc_Miss} \newline


Since the \emph{missioninformer} is a leightweight code and 
because it can be used as a black box, it already
can be implemented easily into a global gradient
free optimization workflow. Since gradient-based 
optimization is known to be faster, especially
with a high number of design variables, 
\emph{missioninformer} was enhanced with the ability 
to calculate gradients. These are gradients of 
the fuel mass of the whole flight mission 
with respect to all arbitrarily given shape parameters. 
To guarantee a high accuracy of  \emph{missioninformer}'s 
state and gradient solutions many different studies 
were conducted.\newline

In order to solve the ODE for the cruise flight segment, two tested solvers exhibited 
particular high accuracies in their results. One of them (\emph{RK45}) also 
demanded few integration steps and thus excels 
in terms of execution time. Therefore, 
(\emph{RK45}) is well suited for research and industrial applications.
Based on this reasoning it is used as a standard ODE solver.\newline


For the gradient calculation, by employing central differencing, 
it can be stated that the iterative indirect method was found to be more
stable. Also, choosing the step-size of $1\mathrm{e}{-5}$ 
could be proven to be optimal. 
This value neither
leads to numerical instabilities nor does it 
not comply with the accuracy requirements.
Therefore, the default method for solving the gradients
with respect to the shape parameters
is the indirect method with its default step-size of $1\mathrm{e}{-5}$.\newline 

Having compared \emph{SciPy}'s RBF and \emph{SMARTy}'s Kriging (Gaussian)
and TPS
interpolation models for different model options, the following can be stated.
Clearly, there are differences in the
outcome of the predicted values and trends for the
desired output variables $LoD, TSFC , AOA$.
As a consequence, once more it can be said that 
the choice of the surrogate model heavily
determines the results of any further calculation. \newline

After having performed a computational intensive 
model selection investigation, the following 
conclusion is made.
No real winner 
could be found out for the options. However, 
using regularization and an augmentation 
of value $1$ can be recommended. If a final 
decision had to be made,
the TPS model with regularization
and an augmentation of $1$ should be chosen.
Note that performing this model selection study 
took more than 3 days on the workstation described 
in section \ref{sec_Motivation}.
This means performing
model selection for each new set of training samples 
within an aerodynamic optimization 
is not feasible. Therefore, TPS with 
active regularization and an augmentation 
of $1$ is set as the default model 
in \emph{missioninformer}.\newline

For the sake 
of applicational feasibility, execution time for the \emph{missioninformer}
with different surrogate models and their respective 
different intern parameters was measured. 
With that it can be stated, the \emph{missioninformer}
suits well for a fast and easy implementation 
into a gradient-based and gradient-free 
optimization workflow, where it completes 
its calculation within few minutes. 
Having introduced the \emph{missioninformer} broadly, 
it clearly can be considered as a possible tool 
at hand to be integrated into new digital design methods 
to reduce the 
negative impact of aviation on the environment, which 
can not be neglected nor overlooked.
 

% -------------------Conclusions and outlook--------------------------------------

\chapter{Conclusions and outlook}
A tool for calculating the total fuel mass and 
its gradient w.r.t. arbitrary shape parameter and for any 
user defined mission or multiple missions is presented.
It was generated from scratch and can be used 
as black box. 
It could be observed, that in case of two 
missions, the runtime was between $50$ to $800$ seconds 
(figure \ref{fig_112}) depending on the employed 
interpolation models and including the gradient computation.
 The following surrogate models 
were intensively investigated with different options:
RBF from the \emph{SciPy} library, Kriging (Gaussian) and TPS
from \emph{DLR}'s toolbox \emph{SMARTy}. RBF 
could clearly be filtered out from the list of the 
attractive surrogate model options, mainly 
due to its long execution times and 
highly options dependent results. In order to solve 
the ODE, which is required for 
computing the fuel fraction of the
cruise segment, different numerical solvers were used. 
By having 
applied simplifications an analytical solution of 
the ODE was introduced. The analytical solution 
was compared with different numerical solvers solutions. 
With these comparisons, 
statements about the accuracy of the numerical 
ODE solver could be made. The solution 
of the total fuel mass incorporates
mass snowball effects though the explained 
fix-point iteration. For the gradients, also 
an analytical approach was desired.
%
However, after comparing the analytically 
determined gradients with central finite 
differences results, the assumption was made for the derivation
were found to be not tenable.
%
A detailed analysis 
for finding an appropriate value for the step-size 
of the central differencing scheme was undertaken. Its 
results could be shown to be unambiguous. \newline 

As an outlook, the analytical solution for the gradients 
should be revisited. It failed due to the elaboration of the mass
term. It is assumed that 
there is a way to reformulate this term and thus obtain 
the gradients analytically. An alternative could be
employing automatic differentiation.
Up to now, only the most important flight segment, the cruise 
segment is described by equations which 
are based on physical observations. The currently 
used fuel fractions for the other  
flight segments could be replaced
by physics-based equations as well. Also in practice,
step-climbs are performed 
in order to fly with an optimal $LoD$. This means, 
the assumption of constant altitude,
which  is made in \emph{missioninformer} 
needs to be adapted. Next, state and gradient calculations are
not consistent due to the chosen approach with multiple
independent surrogate models. \emph{Missioninformer} should be 
tested in an aerodynamic gradient-based optimization 
to find out how many 
training points are necessary to achieve significant
reduction in the missions fuel consumption. Finally, 
the gradients of the missions fuel w.r.t. 
the structural mass needs to be calculated in order 
to use \emph{missioninformer} within 
a multi-disciplinary-optimization workflow.

% =====================================================================
% ========================= Bib =======================================
% =====================================================================
% % % define the bib file
\bibliography{3_Bibtex/bib.bib}


\end{document}