
\section{Shape Gradients}
\label{sec_Shape_Gradients}
In this section it shall be explained, 
how the gradients of the 
presented cruise fuel burn equation (ODE in equation \eqref{eq_10}), 
can be computed. When the gradients are 
mentioned, then gradients with respect to 
the shape parameter, which are used to 
define the aircraft wing, are meant. For this 
work two main methods were evolved, an 
analytical and a numerical approach. The first one, however 
showed not be successful. A possible explanation for that 
will be given in the upcoming section. The numerical 
approach is based on the 
\emph{Central differencing Scheme} and is 
divided into 2 different workflows 
called direct and indirect, which will 
be elaborated in subsection \eqref{subsec_CDS}.
Furthermore, a step-size study in order to find 
a reasonable step-size and thus 
reliable gradients will be topic of the subsection 
\ref{subsec_CDS}

\subsection{Analytical attempt}
\label{subsec_Anly_Attempt}
Assuming, an analytical solution is possible, 
a fast and maximal computational accurate solution is obtained.
These were the main reasons for trying to find an 
analytical solution. Fast in this regard means, 
the result is received immediately after 
all mathematical operations (summation, multiplication) 
are executed without any necessity of additional repetitive 
loops. The state equation for the cruise fuel burnt is 
written below again and for simplicity will be referred to as  
$ODE$ and the objective is to find $dODE$.

\begin{equation*}
    ODE = \frac{dm_{cr}}{ds} = \frac{g}{a} \: \frac{1}{Ma}\;
    TSFC\, (m_s - m_{fe}) \; \frac{LoD\,tan(\theta)+1}
    {LoD\,cos(AoA) + sin(AoA)}
\end{equation*}

The ODE's gradient $dODE$ with respect to the shape 
parameters is already given in \cite{ilic_goal_2013}. However, 
the same work has been done again in order to 
verify the existing solution. Once by hand and once 
by using Python's library \emph{SymPy} for a symbolic solution.
Ilic \cite{ilic_goal_2013} results could be 
proven to be entirely correct and shall be 
stated in the upcoming equations. Since 
the gradient version is a term loaded equation, 
it will be split into groups, as advised in 
\cite{ilic_goal_2013}. The grouped 
equation as the new basis is given in equation
 \eqref{eq_38} for which its terms definitions 
 are provided in equations \eqref{eq_39}
 to \eqref{eq_41}


\begin{equation}
    \label{eq_38}
    ODE = k_c \, k_a \, k_{m}
\end{equation}

\begin{equation}
    \label{eq_39}
    k_c  = \frac{g}{a} \frac{1}{Ma} \, TSFC
\end{equation}

\begin{equation}
    \label{eq_40}
    k_a  = \frac{LoD \, tan(\theta) +1 }{LoD \, cos(AoA) + sin(AoA)}
\end{equation}

\begin{equation}
    \label{eq_41}
    k_m  = m_s - m_{cr}
\end{equation}

A change in the shape parameter $p$ results 
into a change in $TSFC\,, m_s, m_{cr}, LoD, AoA$, which 
means these variables are not to be considered as constants.
Equations \eqref{eq_42} to \eqref{eq_47} shows 
the general form of the $dODE$, where 
the flight path angle $\theta \neq 0$ is not 
necessarily zero. The equations \eqref{eq_44}
and \eqref{eq_45} results from 
the steady state flight condition. Here 
the denoting of the variables 
$q,\, S,\, W\,, L\,, D \, , T$ is the dynamic pressure,
reference wing area, weight, lift, drag and thrust, 
respectively.


\begin{equation}
    \label{eq_42}
    \frac{dk_a}{dp} = \frac{[ tan(\theta) \, sin(AoA) - cos(AoA)]
    \,\frac{dLoD}{dp} + [LoD \,sin(AoA) - cos (AoA)] 
    \,[LoD \, tan(\theta) + 1] \, \frac{dAoA}{dp} }
    {[LoD \, cos(AoA) + sin (AoA)]^{2}}
\end{equation}


\begin{equation}
    \label{eq_43}
    \frac{dLod}{dp} = \frac{d}{dp} \, \frac{C_L}{C_D} = 
    \frac{1}{C_{D}^{2}} \left(C_D  \frac{dC_L}{dp} - 
    C_L \frac{dC_D}{dp}  \right)
\end{equation}


\begin{equation}
    \label{eq_44}
    L = W \,cos(\theta) - [D + W \, sin(\theta)]\, tan(AoA)
\end{equation}

\begin{equation}
    \label{eq_45}
    C_L = \frac{m\, g}{q \, S}\,[cos(\theta) - sin(\theta) tan(AoA)] -
     C_D \, tan(AoA)
\end{equation}

\begin{multline}
    \label{eq_46}
    % \begin{split}
        \frac{dC_L}{dp} = \frac{g}{q\, S}\, [cos (\theta) - sin(\theta)\, tan(AoA)]
        \frac{dm}{dp} -
         \frac{m \, g}{q \ S^{2}}\,[cos(\theta) - sin (\theta) 
        tan(AoA)] \frac{dS}{dp} \\
        - \frac{1}{cos(AoA)^{2}} \left(\frac{m\, g}{q \, S} sin(\theta) + C_D
        \right) \frac{dAoA}{dp} - tan (AoA) \frac{dC_D}{dp}
    % \end{split}
\end{multline}

\begin{equation}
    \label{eq_47}
    \left(\frac{dC_L}{dp}\right)_{T=0} = \frac{g}{q\, S}cos(\theta) 
    \frac{dm}{dp} - \frac{m\, g}{q\, S^{2}} cos(\theta)\frac{dS}{dp}
\end{equation}

The \emph{missioninformer} assumes the flight 
path angle $\theta$ to be zero and thus the 
above stated equations are simplified to 
the two upcoming equations. Even though 
it has been showed how $LoD$ can be 
derived w.r.t. the shape parameter, however 
since its gradient is provided 
as input data, there is no need 
to its equation.


\begin{equation}
    \label{eq_48}
    k_a (\theta = 0)  = \frac{1 }{LoD \, cos(AoA) + sin(AoA)}
\end{equation}

\begin{equation}
    \label{eq_49}
    \frac{dk_a(\theta = 0)}{dp} =
     \frac{- cos(AoA)
    \,\frac{dLoD}{dp} + [LoD \,sin(AoA) - cos (AoA)] 
    \, \frac{dAoA}{dp} }
    {[LoD \, cos(AoA) + sin (AoA)]^{2}}
\end{equation}

The mass term $k_m$ was the obstacle, why the gradients 
can not be solved  analytically as it will be shown now.
The starting mass $m_s$ can be modeled as 
done in equation \eqref{eq_50}, where 
$m_{struc},\, m_p,\,  m_{cr}(R),\, m_o$ are denoted 
as masses of structure, payload, cruise and other, respectively.
The structure and other masses as well the payload 
are not meant to change by changing the shape parameter.
The only remaining mass, which is desired to 
change with the cruise range $R$ is $m_{cr}(R)$. 
Transforming this idea to mathematical expressions,
the equations \eqref{eq_51} to \eqref{eq_52}
are obtained.
\begin{equation}
    \label{eq_50}
    m_s = m_{struc} + m_p + m_{cr}(R) + m_o
\end{equation}

\begin{align}
    \frac{dm_{struc}}{dp} &= 0 \label{eq_51}\\
    \frac{dm_{p}}{dp} &= 0\\
    \frac{dm_{cr}(R)}{dp} &\neq 0\\
    \frac{dm_{o}}{dp} &= 0 \label{eq_52}
\end{align}

Following these the equation \eqref{eq_53} is 
obtained and by inserting this to the mass definition 
from equation \eqref{eq_41}, the undesired state 
given in 
equation \eqref{eq_54} can be observed. 
\begin{equation}
    \label{eq_53}
    \frac{dm_s}{dp} = \frac{dm_{cr}(R)}{dp}
\end{equation}

\begin{equation}
    \label{eq_54}
    \begin{aligned}
        \frac{dk_m}{dp} &= \frac{dm_s}{dp} - \frac{dm_{cr}(R)}{dp}\\
                    &\rightarrow \frac{dm_{cr}(R)}{dp} -  \frac{dm_{cr}(R)}{dp} \\     
        &\Rightarrow 0
    \end{aligned}
\end{equation}

In this case the term for which the gradient is desired to be calculated 
vanished. Therefore, the approach stated in 
equation \eqref{eq_55} was pursued instead, which leads 
to equation \eqref{eq_55_dkm_dp}. In words, 
the change of the starting mass is assumed to be constant and 
thus is not influenced by the change of any shape parameter.
At this point, it shall clearly be highlighted that 
this assumption is only made in 
order not to lose the cruise fuel mass term. Whether 
this models the physical behavior correctly 
was not known at this stage.

\begin{equation}
    \label{eq_55}
    \frac{dm_s}{dp}  = 0 
\end{equation}

\begin{equation}
    \label{eq_55_dkm_dp}
    \frac{dk_m}{dp}  = \frac{-dm_{cr}}{dp}
\end{equation}

However, after having performed, the complete 
process for obtaining the analytical solution 
for the cruise fuel mass gradients, 
it can be stated that the assumption
made in equation \eqref{eq_55} is 
not correct. For this knowledge to gain, 
the gradients were calculated numerically  
using central finite differences.
Due to the big deviation between 
the analytical and the numerical 
solution, the proposed finding can be given.
Nevertheless, the further progress for 
receiving the final analytical solution 
shall be explained. The variables $m_{f,zero},\,
f_{reser}, \, m_f$ for 
the upcoming equations are 
denoted as zero fuel mass, fuel reserve 
mass, and total 
fuel for all flight segments, respectively. 
As explained in section \ref{sec_Fuel_Mass_iter}
the variable starting with $fr_i$ are fuel 
fractions, where  
$fr_1 \,, fr_2\,,fr_3 \,, fr_4,\, fr_5\,, fr_6\, , fr_7, \,
fr_{cr}$ 
are for the  
flight segments, engine start, taxi to runway, takeoff,
climb and acceleration, taxi to parking,
landing, descent and cruise fuel fraction respectively.


\begin{equation}
    \label{eq_55_Fcr}
    fa \,(fr_{cr}) = f_{zero} \, fr_{cr} \, fr_1 \, fr_2\,fr_3 \, fr_4 \, 
    fr_5\, fr_6\, fr_7
\end{equation}

\begin{equation}
    \label{eq_56}
    m_{f} = \,\frac{fa (fr_{cr}) - m_{f,zero}}{f_{reser} -fr_{cr} 
    \,fr_1 \, fr_2\,fr_3 \, fr_4 \, 
    fr_5\, fr_6\, fr_7 }
\end{equation}

\begin{equation}
    \label{eq_57}
    fr_{cr} =  \frac{m_s - m_{cr}}{m_s}
\end{equation}

\begin{equation}
    \label{eq_58}
    m_s = (m_{f,zero} + m_{f})  \, fr_1 \, fr_2\,fr_3 \, fr_4
\end{equation}


The objective is to solve the equation \eqref{eq_56}. However, the 
problem is that the three equations \eqref{eq_56} to \eqref{eq_58}
are coupled. This can be seen better by rearranging the following 
equation. 

\begin{equation}
    \label{eq_59}
    ff = fr_1 \, fr_2\,fr_3 \, fr_4 ,\, fr_5\, fr_6\, fr_7 
\end{equation}

\begin{equation}
    \label{eq_60}
    \begin{aligned}
        m_f &= \frac{fr_{cr} fr_1 \, fr_2\,fr_3 \, fr_4 ,\, fr_5\, fr_6\, fr_7 -1
        }{f_{reser} - fr_{cr} ,\ fr_1 \, fr_2\,fr_3 \, fr_4 ,\, fr_5\, fr_6\, fr_7 }
        \, m_{f,zero}\\
        &= \frac{fr_{cr} ,\ ff - 1   }{f_{reser}- fr_{cr} ,\ ff  } \, m_{f,zero}
    \end{aligned}
\end{equation}

\begin{equation*}
    fr_{cr} =  \frac{m_s - m_{cr}}{m_s}
\end{equation*}

\begin{equation*}
    m_s = (m_{f,zero} + m_{f})  \, fr_1 \, fr_2\,fr_3 \, fr_4
\end{equation*}


It can be observed that in equation \eqref{eq_60}, the 
output of the ODE, the fuel cruise mass $m_{cr}$ is required. 
This, according to the equation \eqref{eq_57}, requires for its solution 
the cruise starting mass $m_s$. However, for 
getting the starting mass $m_s$, 
according to equation \eqref{eq_58}, the total fuel mass 
$m_f$ is demanded. The problem can be understood visually 
quite easily and thus is provided in figure \ref{fig_6_Coupled}.

\begin{figure}[!h]
        \centering
        \def\svgwidth{0.4\textwidth}
        \input{2_Figures/2_Task/2_Coupled.pdf_tex}
        \caption{Coupled mass equations}
        \label{fig_6_Coupled}
\end{figure}

\FloatBarrier
One way to solve equation is to solve the equation \eqref{eq_62}
with another fix-point iteration. The used 
scalars are given in equations \eqref{eq_59} and \eqref{eq_61}.
The one disadvantage lies 
in the nature of fix-point iterations, its repetitiveness, 
which requires higher execution time. 
The other is, that a more precise solution could 
be obtained. 
\begin{equation}
    \label{eq_61}
    ff_2 =  fr_1 \, fr_2\,fr_3 \, fr_4
\end{equation}

\begin{equation}
    \label{eq_62}
    m_f = m_{f,zero} = \dfrac{\left(1- \dfrac{m_{cr}}{(m_{f,zero} + m_f) \ ff_2}
    \right)ff -1}{f_{reser} - ff \, \left(  1 -\dfrac{m_{cr}}{(m_{f,zero} + m_f) \ ff_2}\right)}
\end{equation}

\vspace{0.17cm}
However, with some trial and error this coupling could be resolved 
and a without the necessity of fix-point 
iterations could be found, which is given in equation 
\eqref{eq_63}. The form of the solution, which was found by 
hand, originally contained a  square root. By making use of 
Matlab and Pythons library \emph{SymPy}, a symbolic solution 
could also be obtained. Matlabs and \emph{Sympy}s solution 
were equivalent in their form, thus also in their results, when 
inserting values for the variables. However, their form 
of solution did not contain any square root. 
The stated equation \eqref{eq_63} was used for further 
progress, since it is assumed to be easier for 
deriving gradients.

\begin{equation}
    \label{eq_63}
    m_f = \dfrac{ff_2 \, m_{f, zero} + ff \, m_{cr} - ff \, ff_2 \, m_{f, zero}}
    {ff_2 \, (ff - f_{reser})}
\end{equation}

In order to highlight why equation \eqref{eq_63} is easier
for deriving gradients, the  considered constants are
placed on the left side and the depended or relevant part 
for derivation are on the right side in equation 
\eqref{eq_64} 

\begin{equation}
    \label{eq_64}
    m_f = \dfrac{ff_2 \, m_{f, zero}  - ff \, ff_2 \, m_{f, zero}}
    {ff_2 \, (ff - f_{reser})} + \dfrac{ff \, m_{cr}(p)}{ff_2 \, (ff - f_{reser})} 
\end{equation}

Recall, $m_{cr}$ is the fuel mass for the cruise segment 
and is the output of the ODE, which now is 
desired to be derived with respect to the shape 
parameters. In equation \eqref{eq_64}, the term 
$m_{cr}(p)$ is indicated to be the only important 
parameter for the derivation. Therefore, the 
derivative of the total fuel mass can 
be written as in equation \eqref{eq_66}, where
the mathematical definition for $dODE$ is given 
in equation \eqref{eq_65}.

\begin{equation}
    \label{eq_65}
    dODE = \frac{d}{dp}\, ODE = \frac{d}{dp}\left(\dfrac{dm_{cr}}{ds}\right)
\end{equation}

\begin{equation}
    \label{eq_66}
    \frac{dm_f}{dp}= 
    \dfrac{ff}{ff_2 \, (ff - f_{reser})} \, \cdot \, dODE
\end{equation}


Finally, by inserting all equations into each 
other the $dODE$'s final version is given 
equation \eqref{eq_67}, where $LoD, AoA, TSFC$
and their gradients are obtained by invoking 
the surrogate models, $k_a \, ,k_m ,\, k_c$
are found in equations \eqref{eq_48}, \eqref{eq_41} and 
\eqref{eq_39}, respectively. The gradients 
$\frac{dk_a}{dp}, \, \frac{dk_m}{dp}$
are found in equations \eqref{eq_49} and \eqref{eq_55_dkm_dp}, 
respectively. The variable $p$ here is defined 
as a scalar for the sake of simplicity. In application 
$p$  in the equations \eqref{eq_65} to \eqref{eq_67} is  
a vector. Thus, the equations 
need to be solved for each component 
of $p$.


\begin{equation}
    \label{eq_67}
    \begin{aligned}
        dODE &= \frac{ODE}{dp} = \frac{d}{dp}\left(\dfrac{dm_{cr}}{ds}\right)\\
        &= k_a \, k_m \, k_c \, \frac{dTSFC}{dp} +k_c \, TSFC \, k_a \, 
        \frac{dk_m}{dp} +k_c  TSFC \, k_m \, \frac{dk_a}{dp} \\
        &= k_c \left(k_a \, k_m \, \frac{dTSFC}{dp}+ 
        TSFC \, k_a \, \frac{dk_m}{dp} + TSFC \, k_m \, 
        \frac{dk_a}{dp}\right)\\
        &= k_c \left(k_a \, k_m \, \frac{dTSFC}{dp}+ TSFC \, k_m \, 
        \frac{dk_a}{dp}\right) - k_c \, TSFC \, k_a \, \frac{dODE}{dp} \\
        &=  k_c\, \dfrac{\left(k_a \, k_m \, \dfrac{dTSFC}{dp}  + TSFC \, k_m \,
         \dfrac{dk_a}{dp}\right)}
        {1+ k_c \, TSFC \, k_a}
    \end{aligned}
\end{equation}

After having seen the formal equations, there are 
two possible ways to calculate the gradients,
the one-shot method and the indirect method.
Both shall be explained with their respective 
workflow in figures \ref{fig_7_Oneshot} and 
\ref{fig_8_Iterative}. 
The one-shot version, depicted in figure 
\ref{fig_7_Oneshot}, will let 
the state fix point-iteration finish as 
discussed in section \ref{sec_Fuel_Mass_iter}.
Based on the converged results it 
will then directly calculate the gradients 
for which only one evaluation is required. Since 
the state fix-point iteration is indispensable
with effectively only one more function call the 
gradients are found. \newline

The iterative workflow is depicted 
in figure \ref{fig_8_Iterative}.
The initial starting and total fuel mass 
$m_{s,0}, m_{f,0}$ are guessed as explained 
in section \ref{sec_Fuel_Mass_iter}. In 
order to calculate the gradient 
of the cruise fuel weight the 
initial cruise fuel weight and cruise
starting mass is passed to the dODE-block, which 
calculates the gradient of the cruse fuel 
weight with respect to the shape parameters. 
This block is not required in the 
one-shot-version nor in the state 
fix-point iteration explained in section 
\ref{sec_Fuel_Mass_iter}. Thus, 
here an additional calculation block is injected.
It is used for computing 
the gradient of the total fuel mass. 
Now the whole process is repeated 
as long as the deviation between the 
previous and the current 
gradient of the total cruise 
fuel mass is below 
a defined threshold. Both methods were 
implemented in \emph{missioninformer}, 
however they are not active, because of 
the explained reasons. 



\begin{sidewaysfigure} [!h]
    \resizebox{1.\textwidth}{!}{
        \hspace*{-5cm} 
    
%%% Preamble Requirements %%%
% \usepackage{geometry}
% \usepackage{amsfonts}
% \usepackage{amsmath}
% \usepackage{amssymb}
% \usepackage{tikz}

% Optional packages such as sfmath set through python interface
% \usepackage{sfmath}

% \usetikzlibrary{arrows,chains,positioning,scopes,shapes.geometric,shapes.misc,shadows}

%%% End Preamble Requirements %%%

\input{/home/jav/Progs/Virt_Env/writing/lib/python3.12/site-packages/pyxdsm/diagram_styles.tex}
\begin{tikzpicture}

\matrix[MatrixSetup]{
%Row 0
\node [DataIO] (output_ode_Solv) {$\begin{array}{c}\text{initial starting} \\ \text{and mission fuel } \\ \text{mass } ($$m_{s,0}, m_{f,0})$$\end{array}$};&
&
&
&
&
&
\\
%Row 1
\node [Function] (ode_Solv) {$\begin{array}{c}\text{ODE}\end{array}$};&
\node [DataInter] (ode_Solv-mass_Calc) {$\begin{array}{c}$$m_{cr,0}$$\end{array}$};&
&
&
&
&
\\
%Row 2
\node [DataInter] (mass_Calc-ode_Solv) {$\begin{array}{c} m_{f,n}, m_{s,n}\end{array}$};&
\node [Function] (mass_Calc) {$\begin{array}{c}\text{mass recalc}\end{array}$};&
\node [DataInter] (mass_Calc-ode_Solv_2) {$\begin{array}{c}\, m_{s}*\end{array}$};&
&
&
&
\node [DataIO] (right_output_mass_Calc) {$\begin{array}{c}m_{s}*, \,  m_{f}* \end{array}$};\\
%Row 3
&
&
\node [Function] (ode_Solv_2) {$\begin{array}{c}\text{ODE}\end{array}$};&
\node [DataInter] (ode_Solv_2-dode_Solv_2) {$\begin{array}{c}m_{cr}* \end{array}$};&
&
&
\node [DataIO] (right_output_ode_Solv_2) {$\begin{array}{c}m_{cr}*\end{array}$};\\
%Row 4
&
&
&
\node [Function] (dode_Solv_2) {$\begin{array}{c}\text{dODE}\end{array}$};&
\node [DataInter] (dode_Solv_2-dx_2) {$\begin{array}{c}\dfrac{dm_{cr}*}{dp}\end{array}$};&
&
\node [DataIO] (right_output_dode_Solv_2) {$\begin{array}{c}\dfrac{dm_{cr}*}{dp}\end{array}$};\\
%Row 5
&
&
&
&
\node [Function] (dx_2) {$\begin{array}{c}\dfrac{dm_f}{dp}\end{array}$};&
&
\node [DataIO] (right_output_dx_2) {$\begin{array}{c}\dfrac{dm_{f}*}{dp}\end{array}$};\\
%Row 6
&
&
&
&
&
&
\\
};

% XDSM process chains


\begin{pgfonlayer}{data}
\path
% Horizontal edges
(ode_Solv) edge [DataLine] (ode_Solv-mass_Calc)
(mass_Calc) edge [DataLine] (mass_Calc-ode_Solv)
(mass_Calc) edge [DataLine] (mass_Calc-ode_Solv_2)
(ode_Solv_2) edge [DataLine] (ode_Solv_2-dode_Solv_2)
(dode_Solv_2) edge [DataLine] (dode_Solv_2-dx_2)
(mass_Calc) edge [DataLine] (right_output_mass_Calc)
(ode_Solv_2) edge [DataLine] (right_output_ode_Solv_2)
(dode_Solv_2) edge [DataLine] (right_output_dode_Solv_2)
(dx_2) edge [DataLine] (right_output_dx_2)
% Vertical edges
(ode_Solv-mass_Calc) edge [DataLine] (mass_Calc)
(mass_Calc-ode_Solv) edge [DataLine] (ode_Solv)
(mass_Calc-ode_Solv_2) edge [DataLine] (ode_Solv_2)
(ode_Solv_2-dode_Solv_2) edge [DataLine] (dode_Solv_2)
(dode_Solv_2-dx_2) edge [DataLine] (dx_2)
(ode_Solv) edge [DataLine] (output_ode_Solv);
\end{pgfonlayer}

\end{tikzpicture}

    }
    \caption{Analytical attempt to calculate gradients - one-shot version}
    \label{fig_7_Oneshot}
\end{sidewaysfigure}

\begin{sidewaysfigure} [!h]
    \resizebox{1\textwidth}{!}{
        \hspace*{-5cm} 
    \input{2_Figures/2_Task/4_Grad.tikz}
    }
    \caption{Analytical attempt to calculate gradients - iterative version}
    \label{fig_8_Iterative}
\end{sidewaysfigure}


