
\section{Surrogate models}
\label{sec_Surrogates}
In very brief words, a surrogate model can be described
as an approximation model. In most cases 
it is used in order to pack computational heavy 
physical based calculations into a simpler 
and execution time friendlier calculation 
environment. Surrogate models are very popular in 
the field of optimization, because 
every optimization algorithm evaluates 
function values with variation of the 
design variables. The design variables 
are the parameters which are changed by the 
optimizing algorithm in order to 
fulfil the minimization or maximization task.
The design variables could consist of only one 
variable, e.g. the wing aspect ratio.
Another example for many design variables
is topology optimization, where 
the number of design variables 
is equivalent to 
the number of the discretized finite elements. For a
real world application, e.g. an aircraft or a car  
the number of finite elements can easily overreach 
millions. The 
optimization algorithm tries to 
minimize the objective function by changing 
the design variables and usually, each 
change demands a new calculation. With 
the variation of the design variables, 
the optimization algorithms collects 
outputs which are compared to each other 
in order to follow a steadily improving 
path. \newline

In case of 
a flow field computation, changing only one parameter, 
e.g. the angle of attack ($AoA$), obligates a new  
expensive computational task. Depending on the applied fidelity and the 
object (airplane, car, turbo machinery) for which the flow field 
is solved, it could acquire seconds till years (
Direct Numerical Simulation). Thus, one objective 
of surrogate models is enabling feasible computational times.
Additionally, surrogate models can be generated and invoked on 
none high performance computers.\newline 

Surrogate models are also known as metamodels or models of models.
Eldred er al. \cite{eldred_second-order_2004}
classified surrogate models 
into three categories: data-fits, reduced order models
 and hierarchical models. 
Reduced-order and hierarchical surrogate models can be 
further classified as physics-based approaches, due to the 
exploitation and simplification of 
governing equations \cite{ahmed_surrogate-based_2009}. Therefore,
these models are considered as \emph{intrusive} methods. 
Data-fit surrogate models, however, are assigned to 
the \emph{black-box} approach, where the outputs
are only based on the inputs, without necessarily 
knowing the underlying equations. Black-box approaches 
are non-intrusive methods and can be further categorized 
into regression and interpolation. The main difference between 
these are, that interpolation must match the training or 
reference data at the given reference points, whereas 
regression is not obligated to match the sample data at 
the reference locations. Two widely 
used interpolation surrogate models are Radial Basis Function (RBF)
and Kriging models. To generate and use 
the black-box intrusive surrogate models, sample data is required which 
for the \emph{missioninformer} is provided by the output of the DLR 
aerodynamic analysis and optimization workflow 
\emph{FSAerOpt} \cite{merle_high-fidelity_2019} involving 
\emph{DLR}'s flow solver \emph{TAU}.\newline

After having the input or training data,
 a surrogate model and its 
intern tuning parameter with 
which the model shall be constructed needs to be defined.
 The generation 
of the surrogate model (offline phase) can claim some noticeable time, 
however, still in feasible manner. Invoking the 
already build surrogate model (online phase) can 
be compared with regard to execution time to
solving a simple linear equation, which is 
satisfactorily feasible. The main disadvantage 
by applying surrogate models 
is the loss of accuracy compared 
to the underlying physics based equations or 
the measured experimental data. For 
assessing the quality of the surrogate model 
some methods are presented in \cite{meckesheimer_computationally_2002}.
However, in \emph{missioninformer} one of the most commonly 
used error measurement, the \emph{root mean square error} (RMSE),
is employed. It is given in equation \eqref{eq_24}, where 
$n, \,y_i, \, \tilde{y_i}$ are denoted as the number of 
validation points (training or sample data obtained 
from workflow of \emph{FSAerOpt} \cite{merle_high-fidelity_2019}), 
the actual value and the predicted value, respectively. \newline


\begin{equation}
    \label{eq_24}
    RMSE = \sqrt{\frac{1}{n} \sum_{i=1}^{n} (y_i - \tilde{y_i})^2}
\end{equation}

The \emph{missioninformer} involves two interpolation surrogate 
models, Kriging and RBF with different options, which are 
described in section \ref{subsec_Kriging_Surrogate} and 
\ref{subsec_RBF_Surrogate}. Both models will be tested
with respect to execution time, robustness and RMSE 
with different model specific options
in section \ref{sec_Diff_Surrogates}. Surrogate models 
within the \emph{missioninformer} are generated for 
$\tilde{LoD}, \tilde{AoA}, \tilde{TSFC}$
and their gradients with 
respect to shape parameter. The input set 
for the mentioned  desired output state and gradient 
variables consists of $Ma, h, m_s$. The state 
surrogate models are invoked for solving 
the equation \eqref{eq_10}, thus needed 
for the cruise fuel mass computation. For the rest of 
the flight segments no surrogate
models are required, since fuel fractions are 
exploited as explained in sections \ref{sec_Cruise_Mission_Fuel}, 
\ref{sec_Solve_ODE} and 
\ref{sec_Fuel_Mass_iter}. Solving the state fuel cruise mass 
invokes the three required surrogate models
to predict
$\tilde{LoD}, \tilde{AoA}, \tilde{TSFC}$  at each 
step undertaken by the chosen \emph{RK45} 
solver \cite{noauthor_scipys_2021}.
The total number 
of state surrogate model invocations for the 
state can be described as equation \eqref{eq_25}.
The number of the gradient surrogate model 
invocations further includes the number of the 
shape parameter. The required number of 
gradient surrogate models and their 
number 
of invocations are given in equations 
\eqref{eq_26} and \eqref{eq_28}, respectively. 

\begin{equation}
    \label{eq_25}
    \#_{invok,state} = 3\, \cdot\, steps_{RK45} 
\end{equation}

\begin{equation}
    \label{eq_26}
    \#_{gener,grad} = 3\, \cdot \,\#_{shape,param}
\end{equation}

\begin{equation}
    \label{eq_28}
    \begin{aligned}
        \#_{invok, grad} &= 3\, \cdot \,steps_{RK45} \, \cdot \, \#_{shape,param} \, 
        \cdot \, 2 \, \cdot \, \#_{fix,pkt,iter} \\
        &\Rightarrow 6\, \cdot \,steps_{RK45} \, \cdot \, \#_{shape,param} \, \cdot \, \#_{fix,pkt,iter}
    \end{aligned}
\end{equation}


In order to understand why equation \eqref{eq_28},
briefly 
missioninfomers implemented method to derive 
gradients shall be mentioned, which in 
detail is explained in subsection \ref{subsec_CDS}.
The gradients in \emph{missioninformer} are 
obtained by exploiting the Central Differencing 
Scheme, which basically adds a very small 
positive and negative perturbation to 
the current state of the present 
desired variable ($LoD, AoA, TSFC$), for which 
the gradient is desired and divides the result 
by two times the size of the perturbation.
Due to the results for both perturbed states, 
positive and negative 
added perturbation, the number 
of surrogate model invocation increases. In simpler terms, 
the surrogate model is invoked once for positive 
added permutation and once for negative added 
perturbation. The perturbation is called step-size. 
How and why \emph{missioninformers} default step-size 
has the value $1\mathrm{e}{-5}$ is described in 
subsection \ref{subsec_CDS} 
and  section
\ref{sec_Step_Size_Study}.\newline

Additionally, 
also the final values for the gradients 
are obtained after a fix-point iteration.
In cases of a barely solvable mission more than 
500 iterations were required, in some 
cases even no solution was obtainable and 
finally in cases where the missions were 
defined properly (not unrealistic high ranges 
or payloads) only $4$ fix-point 
iterations were required. As for the  
tested missions, a correlation could be read 
out. In cases where the constraint violation  
warning was given, more than $4$ fix-point 
iterations could be seen. Therefore, it 
is advised to observe the warning outputs.
Note, the number of the fix-point iterations is also 
depended on the chosen step-size. The number of 
$4$ fix-point iteration is met due to 
the well-fitted step-size of $1\mathrm{e}{-5}$.
The step-size can be changed inside 
the mission input file for each mission. However,
the default value in \emph{missioninformer} is set 
to $1\mathrm{e}{-5}$ and a 
change in its value is assumed to most likely 
have a negative impact on the number 
of fix-point iterations. The reasons for that 
will become clearer
by considering the convergence behavior 
depicted in the step-size study 
in section \ref{sec_Step_Size_Study}.\newline 


Assuming having $126$ shape parameter, $20$ required 
steps for the ODE's integration and $4$ fix-point
iterations, the gradient 
surrogate models are invoked $30240$ times.
Using directly CFD calculations, where one CFD result 
takes easily 3 hours using a 
reasonable amount of high performance 
computer resources.  


Finally, 
it shall be mentioned that the number of fix-point 
iterations is limited to $500$ iterations. In cases, 
where the missions are physically correctly defined 
and no constraint violation warning were observed,
not more than $5$ fix-point iterations were encountered. 
Therefore, it is regarded to be highly unlikely to require 
more than $500$ iterations. When the limit of $500$ iterations 
is reached, the user will be notified in the output 
about the abortion due to the exceeding of the  fix-point 
iteration number's limit. 
