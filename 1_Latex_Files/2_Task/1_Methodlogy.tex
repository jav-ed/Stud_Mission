
\chapter{Methodology}
\label{chap_Methodlogy}
In this chapter the employed equations, iterations and 
surrogate models for 
obtaining the state and gradients of the burnt fuel and their theoretical 
background shall be given. Additionally, the 
way of exploring methods, the reasons for 
having chosen the methods and the proceeding to obtain solutions
shall be explained. Furthermore, some workaround which was 
required will be mentioned, e.g. in the upcoming section, 
where a small tool was developed in order to get 
the atmospherical relationships.

% =====================================================================
% ============= Atmospheric ===========================================
% =====================================================================
\section{Atmospheric conversions}
One term inside the cruise fuel burn equation
for the computation of the cruise segment fuel
is the speed of sound ($a$), which 
is not given by the user. However, what the user provides is 
the altitude. By making use of one of the 3 common atmospheric models,
ICAO-, US- Standard Atmosphere or Norm Atmosphere DIN 5450/LN 9300, 
a relationship between pressure, density and altitude 
($p, \rho, h$), can be obtained. For our purposes, the
\emph{ICAO-Standard Atmosphere(ISA)} and 
\emph{Norm atmosphere DIN 5450/LN 9300} were used within an altitude range 
of 0-20 km (troposphere, lower stratosphere) 
and 20-32 km (upper stratosphere), respectively. With the mentioned 
atmospheric models, altitude can be obtained by providing 
pressure and density and vice versa. The \emph{missioninformer} is
capable of both and by following the equation \eqref{eq_1_Speed_Sound} 
the speed of sound ($a$) can be calculated. The \emph{heat capacity ratio} is chosen 
to be $\kappa = 1.4$. The reviewed process is 
depicted in figure \ref{fig_3_Atmospheric}

\begin{equation}
    a = \sqrt{\kappa \frac{p}{\rho}}
    \label{eq_1_Speed_Sound}
\end{equation}

\begin{figure}[!h]
    \def\svgwidth{\linewidth}
    \input{2_Figures/2_Task/1_atmospheric.pdf_tex}
    \caption{Atmospheric model and calculation of the speed of sound $a$}
    \label{fig_3_Atmospheric}
\end{figure}


%--------------- Von hier aus erneut lesen ---------------------
 
% =====================================================================
% ============= Cruise and Mission Fuel ===============================
% =====================================================================
\section{Cruise and mission fuel mass}
\label{sec_Cruise_Mission_Fuel}
In this section the derivation of the fuel burn equation 
for the cruise segment and the applied solution method shall be 
explained. Since the \emph{missioninformer} considers the whole 
mission it will also be elaborated how the 
fuel masses for the remaining flight segments are 
obtained.\newline 

The derivation of the cruise segment fuel burn equation, which 
takes the form of an Ordinary Differential Equation (ODE) was provided by 
Ilic in an \emph{DLR}-intern report 
\cite{ilic_goal_2013}. For its derivation a constant altitude 
()$h = const.$) is assumed. The cruise fuel burnt 
mass is the mass of fuel, which is required for the airplane in order 
to reach or fly a given range. The ODE is given in equation \eqref{eq_2_ODE_Orig}, 
where $m_{fe}$ is the burnt fuel mass, called fuel expended accordingly 
to the original 
report \cite{ilic_goal_2013}. The derivation is applied with respect to 
the range $ds$, $g,a,Ma,TSFC$ are denoted as the gravitational constant, 
the speed of sound, the Mach number, and $TSFC$ thrust specific fuel 
consumption, respectively. 
\begin{equation}
    \label{eq_2_ODE_Orig}
        \frac{dm_{fe}}{ds} = \frac{g}{a} \: \frac{1}{Ma}\;
        TSFC\, (m_s - m_{fe}) \; \frac{\frac{C_L}{C_D}\,tan(\theta)+1}
        {\frac{C_L}{C_D}\,cos(AoA) + sin(AoA)}
\end{equation}

The fraction $\frac{C_L}{C_D} = \frac{L}{D} = LoD$ is called aerodynamical 
performance or glide ratio. $AoA, \theta, m_s$ are denoted as the 
angle of attack, flight path angle and cruise starting mass. The presented 
equation is only one possible equation formulation, which can be employed to find the 
cruise fuel burnt mass. In order to 
understand why this is so, its derivation shall be explained briefly. In 
his report \cite{ilic_goal_2013}, the author starts with the equilibrium equations 
for steady flight. By inserting them into each other and using 
equation transformations Ilic \cite{ilic_goal_2013} comes to 
an expression for the thrust as: 
$ T = f(m,g,LoD,AoA, \theta)$. At this stage, the change 
of the fuel mass with the distance traveled can be 
expressed in the following two ways: 

\begin{equation}
    \label{eq_3_}
    \frac{dm_{fr}}{ds} = - \frac{TSFC}{Ma\, a \, cos(\theta) }\, T
\end{equation}

\begin{equation}
    \label{eq_4}
    \frac{dm_{fe}}{ds} = \frac{TSFC}{Ma\, a \, cos(\theta) } \,T
\end{equation}

Because the present total airplane mass m 
in the thrust expression $ T = f(m,g,LoD,AoA, \theta)$ can be expressed in
two ways, two equations (ODEs) are possible 
as outcome to describe the cruise fuel mass burnt. One is over the 
remaining fuel mass $m_{fr}$

\begin{equation}
    \label{eq_5}
    m = m_e + m_{fr}  ,
\end{equation}
where $m_e$ is the end mass or the mass right before landing. 
The second ODE is obtained by using the expended fuel mass as: 
\begin{equation}
    \label{eq_6_}
    m = m_s - m_{fe}  ,
\end{equation}
where $m_s$ is the cruise start mass or the mass just 
before cruise, after climb and acceleration. 
According to common field specific knowledge, masses in aerospace 
can be further 
distributed into detailed definitions, e.g. the takeoff mass  
can be further split up in operating empty weight, 
payload and fuel weight. The operating empty weight itself 
can be split up into airframe structure, propulsion 
group, airframe services and equipment, fixes equipment, 
removable equipment, standard items, standard item 
variation and operational items \cite{torenbeek_synthesis_1982}.
Depending on the field of research and the desired 
level on accuracy the 
mass division can be continued. However, the start mass 
$m_s$ from equation \eqref{eq_6_} contains the fuel weight 
for all segments which are followed after climb and
acceleration, cruise till taxi to parking.
During the flight this fuel mass 
is going to be reduced. With this background the two 
upcoming equations can be looked at:

\begin{equation}
    \label{eq_7_}
    \frac{dm_{fr}}{ds} = - \frac{g}{a} \, \frac{1}{Ma}\,TSFC \,(m_e + m_{fr})
    \, \frac{\frac{C_L}{C_D}\,tan(\theta) + 1}{
        \frac{C_L}{C_D} \, cos(AoA) + sin(AoA)}
\end{equation}

\begin{equation}
    \label{eq_8_}
    \frac{dm_{fe}}{ds} = \frac{g}{a} \, \frac{1}{Ma}\,TSFC \,(m_s - m_{fe})
    \, \frac{\frac{C_L}{C_D}\,tan(\theta) + 1}{
        \frac{C_L}{C_D} \, cos(AoA) + sin(AoA)}
\end{equation}

The difference between both comes through their derivation and 
thus the thought process or undertaken assumptions. 
Equation \eqref{eq_7_} works with remaining fuel, 
the end mass $m_{e}$ is known, 
the fuel is calculated and with these 
the start mass $m_{s}$ is obtained, see 
equation \eqref{eq_5}. This version lets 
the start mass be variable and requires 
the end mass as input. For equation 
\eqref{eq_8_} the start mass $m_{s}$ is 
passed as an input argument, the expended or required 
fuel for given cruise range is calculated. \newline 

Comparing 
the solution methods for the ODEs \eqref{eq_7_} 
and \eqref{eq_8_}, the first is integrated backwards
from cruise range end to zero.
In other words, the initial condition is given 
with the assumption that the remaining fuel 
at the end of the range equals zero $m_{fr}(R=R_{cr,max}) =0$, 
where $R$ is denoted as the range. As mentioned in 
section \ref{sec_1_State}, \emph{pyMission} follows 
this approach for solving the ODE.
In contrast, for defining the initial condition 
for equation \eqref{eq_8_} the fuel 
mass at the cruise start or cruise range zero 
is provided. 
It can be understood, that before the 
cruise even starts, no fuel in the cruise 
section is burnt. Note, the presented ODEs 
are only valid for a cruise segment. Therefore, 
just at the start of the cruise segment, neglecting 
other flight segments for this consideration, no 
fuel within the cruise segment is burnt.
Thus, the initial condition is formulated as 
equation \eqref{eq_8_Inital_Condit}. In this way, the ODE is 
integrated forward from 0 to given range.\newline

\begin{equation}
    \label{eq_8_Inital_Condit}
    m_{fe}(R_{cr}=0) = 0
\end{equation}

For this work the viewpoint of equation \ref{eq_8_} 
considering  
fuel expended or burnt is going 
to be applied in order to derive the upcoming 
solutions. Furthermore, the flight path angle is 
set to zero 
$\theta = 0$, see equation \eqref{eq_9}. Additionally, with the introduced 
relationship $LoD = \frac{C_L}{C_D}$ the final 
ODE can be formulated as equation \eqref{eq_10} \newline

\begin{equation}
    \label{eq_9}  
    \frac{dm_{fe}}{ds} = \frac{g}{a} \, \frac{1}{Ma}\,TSFC \,(m_s - m_{fe})
    \, \frac{1}{
        \frac{C_L}{C_D} \, cos(AoA) + sin(AoA)}  
\end{equation}


\begin{equation}
    \label{eq_10}  
    \frac{dm_{fe}}{ds} = \frac{g}{a} \, \frac{1}{Ma}\,TSFC \,(m_s - m_{fe})
    \, \frac{1}{
        LoD \, cos(AoA) + sin(AoA)}  
\end{equation}

Before explaining the methodology for solving the chosen ODE 
\eqref{eq_10}, the inclusion of the remaining flight segments shall 
be explained. In \cite{roskam_airplane_1986} fuel 
fractions, which are the ratio of the aircraft total 
weight at the end of a flight segment to the weight at the start 
of the same segment,
were provided. The employed fuel fractions 
for the associated flight segment are presented in the 
table \ref{tab_1}. The obvious advantage of applying 
empirical based fractions instead of solving physical based equations 
for flight segments other than cruise, is the 
absence of time and thus money consuming modeling
and solving of 
physical based equations and as consequence a 
fast as well as easy implementation.
Also, since only a multiplication is performed, the output 
in \emph{Python} is obtained within more than a feasible time. The 
disadvantage however, is a loss of accuracy. Given the fact that 
most fuel is burnt during the cruise segment, this seems 
a valid approach. For understanding, why the reserve fuel 
fraction is set to $0.05$, consider the 
equation \eqref{eq_56} in the
subsection \ref{subsec_Anly_Attempt}.

\begin{table}[!h]
    \centering
    \begin{tabular}{|l|c|c|}
        \hline
        Flight segment & Fuel fraction\\
        \hline
        engine start & 0.99 \\
        taxi to runway & 0.99 \\
        takeoff & 0.995 \\
        climb and acceleration & 0.98 \\
        descent &  0.99 \\
        landing &  0.992 \\
        taxi to parking &  0.99\\
        reserve &  0.05\\
        \hline
    \end{tabular}
    \caption{Fuel fractions for non cruise flight segments }
	\label{tab_1}
\end{table}



