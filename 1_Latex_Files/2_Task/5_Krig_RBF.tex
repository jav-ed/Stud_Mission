
\subsection{Kriging models}
\label{subsec_Kriging_Surrogate}
Kriging for using as a surrogate model was originally developed 
for the field of geostatics by Daniel G. Krige, after which the 
method is named \cite{krige_statistical_1951}. The 
term \emph{Kriging} was shaped by Matheron \cite{matheron_principles_1963}, 
who was also the first to formulate it mathematically. Initially after 
its development, Kriging was used to model continuous 
defined functions based on measurement data. The foundation 
of exploiting Kriging models in the Design and Analysis of 
Computer Experiments (DACE) was first developed by 
Sacks et al. \cite{sacks_design_1989}, where 
the input points represent the spatial 
geographical coordinates. In Kriging models a 
deterministic response $y(x)$ is a realization of a stochastic
process $Y(x)$ \cite{sacks_design_1989} and is described in 
the following equation: 

\begin{equation}
    \label{eq_29}
    Y(x) = \sum_{k=1}^{N_f}f_k(x) \beta_k + Z(x) = f^T (x) \beta + Z(x)
\end{equation}

The first term is the global model component, where
$f(x) = [f_1(x), \, f_2(x),\, ....,\,
f_{N_f}(x)]^T$ 
is a vector of $N_f$ 
basis functions and $\beta = [\beta_1, \,\beta_2,\, ..., \,\beta_{N_f}]$
is a vector of the
unknown coefficients. The stochastic component $Z(x)$ is 
treated as the realization of a stationary Gaussian 
function with an expected value of zero ($E[Z(x)]=0$) 
and the covariance is given in equation \eqref{eq_30}, where 
$R$ denotes the correlation function with $R(0) = 1$.
Because of this, Kriging models can be used
as interpolation models 
and thus provide the exact prediction at the 
training or sample data. However, by applying 
regularization, which adds a constant to each 
control point, the interpolation turns into 
regression. The greater the distance 
of the input, for which the predicted 
output is desired is from the training sample data, 
the greater the error in the variance gets.
In other words, in Kriging models the data are assumed 
to be exact, but the function is a realization of a
Gaussian process \cite{viana_special_2014}. The second term 
is denoted as bias, a systematic departure from the linear model 
or localized deviation \cite{simpson_kriging_2001}.

\begin{equation}
    \label{eq_30}
    Cov\, [Z(x_i), \,Z(x_j)] = \sigma^2\, R(x_i, x_j)
\end{equation}

In \emph{1d} the stationary correlation function 
between 
any two points in the input space $y(x_i)$
and $y(x_j)$, depend only on the difference 
between these points ($\Delta x = x_i - x_j$).
For higher-dimensional problems, the correlation function 
in a Kriging model usually 
obeys the \emph{product correlation rule}, given in
equation \eqref{eq_31}, 
where $\theta = \theta^{(k)}, \, k= 1,...., N_d$ is denoted as the 
vector of correlation parameters. The notation $d_{ij}^{(k)}$ is the distance 
between two points in the $k^{th}$ dimension $(\, x_i^{(k)} 
- x_j ^{(k)\,})$. The correlation parameters, are called Kriging 
hyperparameters and can be found by using the 
Maximum Likelihood Estimation (MLE) approach. Large 
$\theta$ values represent a weak spatial correlation, whereas 
small values stand for a strong spatial correlation.
Since \emph{missioninformer} calls \emph{SMARTy} for its 
Kriging operations, different correlations 
functions can be used. However, within this research 
the Gaussian correlation function
is used.


\begin{equation}
    \label{eq_31}
    R_{ij}\,(\theta, d) = \prod_{k = 1}^{N_d} R\,(\theta ^{(k)}, \, 
    d_{ij}^{(k)})
\end{equation}


\subsection{RBF models}
\label{subsec_RBF_Surrogate}
RBF stands for \emph{Radial Basis Function} and is 
black-box surrogate model, which simulates complicated 
design landscapes using a weighted sum of 
simple functions as shown in equation \eqref{eq_32}.
Here $x_0,\, x_s,\, y_s$ are denoted as 
input space, samples location's and output's value, respectively.
The function $\psi (\,||x_0 -c_i||\,)$ is the kernel
function centered at $c_i$ and the norm $||\cdot||$ is 
the Euclidean distance. Usually, the training sample points are used as 
the center, therefore it can be stated: $c = x_s \;\wedge \; N_c = N_s$.
\begin{equation}
    \label{eq_32}
    \tilde{y}\,(x_0, x_s, y_s) = \Psi_{0}^{T} \, w =
    \sum_{i = 1}^{N_c} w_i \, \psi (\,||x_0 -c_i||\,)
\end{equation}

The vector of the unknown coefficients $w$, is obtained by 
solving the system of linear equations given in equation 
\eqref{eq_33}. The gram matrix $\Psi$ is defined as 
$\Psi_{ij} = \psi(|| \, x_{s_i} - x_{s_j}\, ||)$. In other 
words, the gram matrix, is the kernel function evaluated 
at the Euclidean distance between the $i^{th}$ and the 
$j^{th}$ samples. 

\begin{equation}
    \label{eq_33}
    \Psi \, w = y_s
\end{equation}


The \emph{missioninformer} invokes 
Pythons library \emph{SciPy} and \emph{SMARTy} for the RBF calculations. 
\emph{SciPy} enables the user to define $7$ different kernels, 
which are given in 
the equations \eqref{eq_34} to \eqref{eq_35}, 
by changing one input parameter 
\cite{noauthor_rbf_2021}. \emph{SMARTy} uses 
the Thin Plate Spline (TPS) kernel.
The variables $r, \epsilon$ are denoted 
as the Euclidean distance and an adjustable constant 
for Gaussian or multiquadtratic functions,
respectively. Only $6$ of the $7$ kernels were used,
since the quintic kernel took multiple 
hours to provide a solution. The $6$ kernels are used for a 
comparison to 
Kriging. The results of the comparison 
are shown in \ref{sec_Diff_Surrogates}


\begin{align}
    multiquadric &\;=\;  \sqrt{\left(\frac{r}{\epsilon}\right)^2 + 1}  \label{eq_34}\\
    inverse &\;=\; \frac{1}{ \sqrt{\left(\dfrac{r}{\epsilon}\right)^2 + 1}} \\
    gaussian &\;=\; exp \;[\;- \left(\frac{r}{\epsilon}\right)^2\;] \\
    linear &\;=\;  r \\
    cubic &\;=\;  r^3 \\
    quintic &\;=\;  r^5\\
    thin\, plate\, spline &\;=\;  r^2 \, ln\,(r) \label{eq_35}
\end{align}


\newpage

\subsection{Investigations on different surrogate models}
\label{subsec_INvestigate_Surro}

In this subsection, the procedure, which was undertaken to 
investigate on the two most widely used 
surrogate models RBF and Kriging shall be explained.
Note, the results are all presented in section 
\ref{sec_Diff_Surrogates}. In the beginning 
of this section an introduction to RBF 
and Kriging and some 
reasons for the necessity of a reliable 
surrogate model were given. 
However,
the main reason comes through the high dependency 
of the \emph{missioninformers} output, cruise fuel mass and their 
gradients. As explained in the same section,
the number of the invocation of the surrogate models 
is very high. \newline 

Only with a surrogate model, which 
replicates the underlying physical based 
behavior correctly, the results of \emph{missioninformer} 
can be considered correct as well.
Due to this 
heavy dominance $8$ different options for 
Gaussian Kriging and RBF with TPS
were chosen as the investigation parameter, 
see table \ref{tab_3}. 
For further reading, Gaussian Kriging will 
be referred to as normal Kriging or 
only Kriging and 
RBF with TPS as TPS.
The $8$ options 
for Kriging and TPS  were equivalent and are given in 
table \ref{tab_3}. The short names on left side 
of the table defines the option's shortcut and are important 
for interpreting
the plot results in section \ref{sec_Diff_Surrogates}.
The values $-1,0,1,2$ for Augmentation mean
no, constant, linear, and quadratic trend function, 
respectively. In case Regularization is set 
to True, a regularization constant is added
to each control point.\newline

\begin{table}[!htb]
    \centering
    \begin{tabular}{l c c }
        \multicolumn{1}{p{2cm}}{\textbf{Name}} &  \multicolumn{1}{p{2cm}}{\textbf{Augmen \newline tation}} 
            &  \multicolumn{1}{p{2cm}}{\textbf{Regulari \newline zation}}\\
            \hline
            A & $-1$ &False \\
            B & $0$ &False \\
            C & $+1$ &False \\
            D & $+2$ &False \\
            E & $-1$ &True \\
            F & $0$ &True \\
            G & $+1$ &True \\
            H & $+2$ &True \\
        \end{tabular}
        \caption{\emph{SMARTy} Kriging and TPS parameter used for investigation 
        surrogate quality}
        \label{tab_3}
\end{table} 

\FloatBarrier
The investigations were performed 
with two different missions as input and 
are reproduced in table \ref{tab_2}. The masses 
are given in kilogram and the 
cruise altitude $h$ and cruise range $R_{cr}$
in meters. For the central difference step-size the default value 
of $1\mathrm{e}{-5}$ was chosen.  
First the surrogate models Kriging, TPS and RBF 
were only tested for the state condition, meaning 
no gradients were calculated.

\begin{table}[!h]
    \centering
    \begin{tabular}{l| c c c c c c }
        \textbf{Mass} & \multicolumn{1}{c}{\textbf{Mission 1}} &&& 
        \multicolumn{1}{c}{\textbf{Mission 2}}\\
        \hline
        maximal takeoff & $245 \mathrm{e}{3}$ &&&      $245\mathrm{e}{3}$         &&\\
        maximal landing & $192.2 \mathrm{e}{3}$ &&&      $192.2\mathrm{e}{3}$         &&\\
        operating empty & $132.5\mathrm{e}{3}$ &&&      $132.5\mathrm{e}{3} $         &&\\
        manufactures empty & $119.2\mathrm{e}{3} $ &&&      $119.2\mathrm{e}{3} $         &&\\
        maximum zero fuel  & $180.5\mathrm{e}{3} $ &&&      $180.5\mathrm{e}{3} $         &&\\
        maximum fuel & $107.6\mathrm{e}{3}$ &&&      $107.6\mathrm{e}{3}$         &&\\
        design payload & $38.52\mathrm{e}{3}$ &&&      $33.60\mathrm{e}{3} $         &&\\
        maximum payload & $48.00\mathrm{e}{3}$ &&&      $48.00\mathrm{e}{3}  $         &&\\
        \hline
        &  &&&               &&\\
        \textbf{Flying parameters} &  &&&               && \\
        \hline 
        $Ma$ & $0.83$ &&&      $0.82$         &&\\
        $h$ & $10.668\mathrm{e}{3}$ &&&      $11.000\mathrm{e}{3}  $         &&\\
        $R_{cr}$ & $9186.0\mathrm{e}{3} $ &&&      $5185.6\mathrm{e}{3} $         &&\\
        \hline
        &  &&&               &&\\
        \textbf{Others} &  &&&               && \\
        \hline 
        step-size & $1\mathrm{e}{-5} $&&&      $1\mathrm{e}{-5}$         &&\\
        weight-factor & $0.7$ &&&      $0.3$         &&\\
        
    \end{tabular}
    \caption{Missions definitions for investigation on different surrogate models}
    \label{tab_2}
\end{table} 


\FloatBarrier
For \emph{SciPy}'s RBF $7$ different 
kernel functions, which 
are given in the equations \eqref{eq_34} -
\eqref{eq_35}, were tested with 
a \emph{SciPy}s \emph{smooth} value of $0.1$. 
In case of \emph{smooth} value greater zero, 
regularization is employed. Without 
it having defined to be greater zero, 
the generation of a surrogate model failed.
For seeing the used \emph{SciPy} RBF's kernel functions and 
interpreting the \emph{SciPy} RBF investigation results the table 
\ref{tab_4} is provided. Note, the kernel 
quintic was dropped due its high amount of 
calculation time.
%
\begin{table}[!h]
        \centering
        \begin{tabular}{l l}
            \multicolumn{1}{p{2cm}}{\textbf{Name}} &
            \multicolumn{1}{p{2cm}}{\textbf{Kernel}}\\
            \hline
            A & multiquadric  \\
            B &  inverse \\
            C &  gaussian\\
            D &  linear\\
            E & cubic\\
            F &  thin plate\\
        \end{tabular}
        \caption{Different \emph{SciPy} RBF kernels used for surrogate model 
        quality investigation }
        \label{tab_4}
\end{table} 
%
\FloatBarrier
Since in the state 
calculation only 3 surrogate models ($\tilde{LoD}, 
\tilde{AoA},  \tilde{TSFC}$) are required to be 
generated, the research is conducted much faster 
compared to the gradient version. 
The weight 
before cruise or the starting mass for the cruise 
segment $m_s$, the fuel weight only 
for the cruise segment $m_{cr}$, the
total fuel mass $m_{f}$, which is required for the  
whole mission  and the number 
of the fix-point iterations are considered 
for the state investigation. For each mass 
the mean value and the standard deviation
over the 
options given 
in the tables \ref{tab_3} and \ref{tab_4}
following the equations \eqref{eq_36} and 
\eqref{eq_37}, respectively were conducted.
\begin{equation}
    \label{eq_36}
    \bar{x} = \frac{1}{N}\sum_{i=1}^{n}x_i
\end{equation}

\begin{equation}
    \label{eq_37}
    \sigma = \sqrt{\frac{1}{N} \sum_{i}^{N}(x_i - \bar{x})^{2}}
\end{equation}

The reason for monitoring 
the mentioned masses is as follows. Since 
only the cruise flight segment involves 
a physical based equation rather 
than the empirical collected fuel fractions, which 
are constants and thus 
their multiplication results to a 
new constant (commutative behavior), 
only a scaling difference between 
the masses is expected. The fuel 
fractions are used only before and after 
the cruise flight segment. Therefore, 
if mistakes occurs within the investigation, 
plotting all the masses enables to find 
them easily visually. Thus considering the
mentioned masses can be viewed as a 
verification technique. The second 
reason is, the impact of the different 
surrogate model options shall not 
only be looked at the final outcome, 
total fuel mass $f_m$, but also on 
the masses in between. \newline 

The number of fix-point iterations 
for the state allows making 
assumptions about the execution time. 
In general, it can be assumed, the less 
the number of fix-point iterations, the 
less the execution time is. Thus, it 
provides information about the impact 
of the model parameter on the execution time
and convergence behavior. For only Kriging,
8 options were tested, therefore 8 results 
are obtained. From here it is possible to 
compare only within the Kriging options.
For TPS the same is done with its 
own 8 results. For RBF 7 different
kernel versions were tested. With the 
presented method the surrogate 
model can only be compared within its 
own surrogate model options environment.
However, it is desired 
to compare Kriging with TPS and with 
\emph{SciPy}s RBF, which would result in $8+8+7 = 23$ results 
to be compared. Note, for 
further reading RBF will be used 
as an abbreviation for \emph{SciPy}s RBF. 
Since this is too much for one 
single plot, the mean value for
each surrogate model (Kriging, TPS, RBF) is 
calculated and compared. In case 
the mean values matches or exhibits 
a low deviation it can be assumed that 
the corresponding models derive to 
same or resembling solutions and vice versa. \newline


By looking at the standard deviation within 
each model, a robustness analysis 
can be performed. In case the standard deviation (std) 
is low, the respective surrogate model is not influenced much 
by the input parameters and thus can be 
seen as stable. A stable or robust 
might not be the most accurate model, however,
it can be assumed to be more reliable for 
unknown complex underlying functions. Also, 
the robustness of the surrogate models 
(Kriging, TPS, RBF) was compared among 
each other.
All investigations were performed for both missions 
separately and together.
For the gradient
version, the same is done plus the 
execution time for the whole process including 
reading the aerodynamic input files and 
writing out all the gradients.\newline

The next step was to explore the accuracy of 
the surrogate models. For this purpose, 
a set of 38 training data was provided. With 
this set 3 different accuracy investigations were employed.
In the first version 20 sample points were 
removed and the surrogate models were trained 
thus with the remaining 18 sample points 
(V18). The second version, 10 sample 
points were removed and the surrogate models 
were trained with 28 sample points (V28).
In these both cases, the obtained 
model was invoked at the removed samples locations.
The actual functions value for 
$\tilde{LoD}, \tilde{AoA}, \tilde{TSFC}$
at the removed values are known, since
they were part of the initial 38 sample 
points. In order to evaluate 
the error for the surrogate models, 
RMSE according to equation \eqref{eq_24}
was used. For the third investigation, 
the \emph{Leave-one-out cross-validation}
(LOOCV) was performed. The idea 
of this method is, one sample point 
is left out and with the remaining 
sample points a surrogate model is 
trained. In case of 38 sample 
points (V38) there are 38 ways 
to leave one sample point out and 
thus 38 different surrogate models 
can be generated. Each 
generated surrogate model is 
invoked at the
location of the left out sample point
and by exploiting RMSE the error 
was measured.
To be more precise, \emph{missioninformer} 
automatically finds non trimmed sample data and 
does not include it for generating surrogate models.
For this workflow, instead of the overall 38 
sample points, only 35 fully trimmed 
sample points,  were used.  \newline

For the 3 
mentioned versions, each version can 
be tested only with regard to the surrogate 
model (Kriging, TPS, RBF) and inside 
a respective surrogate model there are 
8 options for Kriging and TPS and 7 
options for RBF. However, because 
bad execution time was observed when 
using \emph{SciPy}s RBF, only Kriging and TPS 
were chosen for the mentioned investigation.
In case of RBF's quinitc kernel, even 
after 4 hours no solution was obtained.
The explained process is performed 
for the state and the gradient version.
The options which exhibit the least 
RMSE are collected and presented in the 
results subsection in \ref{subsec_Krig_TPS_Accu}.
