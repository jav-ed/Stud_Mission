
% =====================================================================
% ============= Solve ODE - Anly ======================================
% =====================================================================
\section{Method for solving the ODE}
\label{sec_Solve_ODE}
The chosen ODE \eqref{eq_10} can be solved numerically 
with different kind of solvers. However, in order to 
have a possibility to validate the
quality of the numerical 
solutions, some simplifications shall be introduced. 
With these, it is possible to derive an 
analytical solution. The variables 
$LoD, AoA, TSFC, m_{fe}$ are all functions 
of the present total mass, which is shown in 
equation \eqref{eq_13}. Since the start 
point for the ODE to be valid is at cruise, the 
current total mass is the cruise fuel starting 
mass $m = m_s$. However, itself 
has a dependency on the expended fuel mass,
which will be elaborated in 
section \ref{sec_Fuel_Mass_iter} and makes 
an analytical explicit solution hard to be derived. Additionally, no 
equations for $LoD(m_s), \,AoA(m_s), \,TSFC(m_s)$, which 
would describe a relationship between the variable 
and the total present mass, are 
given and therefore formulating an analytical solution becomes 
impossible.\newline

\begin{equation}
    \label{eq_13}  
    \frac{dm_{fe(m)}}{ds} = \frac{g}{a} \, \frac{1}{Ma}\,TSFC(m) \,(m_s + m_{fe}(m))
    \, \frac{1}{
        LoD(m) \, cos(AoA(m)) + sin(AoA(m))}  
\end{equation}

At this point the mentioned 
simplifications for an analytical solution 
can be mentioned. Assuming $LoD, AoA, TSFC$
to be constant allows to find an analytical
solution, as is presented in the upcoming equations. Consider the 
equation \eqref{eq_11}, where the 
A and B are constants and x is the 
variable for which the solution it is 
pursued.


\begin{equation}
    \label{eq_11}
    \frac{dx}{ds} + A\,x = B
\end{equation}

The solution to this problem \eqref{eq_11} can be 
obtained by using the integrating factor 
given in \eqref{eq_12}, where $C$ is a new 
constant. By replacing $x = m_{fe}$ equation 
\eqref{eq_14} is obtained. 
\begin{equation}
    \label{eq_12}
    x = \frac{B}{A} + C\, e^{-A\, s}
\end{equation}
 
\begin{equation}
    \label{eq_14}
    m_{fe} = \frac{B}{A} + C\, e^{-A\, s}
\end{equation}

It is understood that, the consumed fuel mass 
at the start or at cruise range zero 
has to be zero $m_{fe}(R_{cr} =0) = 0$, which 
is the initial condition to the equation \eqref{eq_14}. 
Inserting this observation leads to equation 
\eqref{eq_15}

\begin{equation}
    \label{eq_15}
    \begin{aligned}
        m_{fe}(s = R_{cr} =0) = 0 =  \frac{B}{A} + C\\
        \Rightarrow  C = - \frac{B}{A}
    \end{aligned}
\end{equation}

Defining $A$ and $B$ as equations \eqref{eq_15_A} and 
\eqref{eq_15_B}, respectively
and incorporating equation \eqref{eq_15}, the 
analytical solution for the cruise fuel burn can be 
written as equation \eqref{eq_16}. 

\begin{equation}
    \label{eq_15_A}
    A = \frac{g}{a} \, \frac{1}{Ma}\,TSFC \, \frac{1}{
        LoD \, cos(AoA) + sin(AoA)}
\end{equation}

\begin{equation}
    \label{eq_15_B}
    B = A \, m_s
\end{equation}



\begin{equation}
    \label{eq_16}
    \begin{aligned}
        m_{fe}(s = R_{cr} =0) = \frac{B}{A} + \frac{-B}{A}\, e^{-A\,s}\\
        \Leftrightarrow  m_{fe} = \frac{B}{A} (1-e^{-A\,s})\\
        \Leftrightarrow  m_{fe} = \frac{B}{A} (1-e^{-A\,s})\\
        \Leftrightarrow  m_{fe} = \frac{A \, m_s}{A} (1-e^{-A\,s})\\
        \Rightarrow m_{fe}(s = R_{cr}) =m_s \left(1-e^{\frac{-g}{a} \, 
            \frac{1}{Ma} \, TSFC \, 
            \frac{1}{LoD \,cos(AoA) + sin(AoA)} \, s} \right)
    \end{aligned}
\end{equation}

This analytical expression in
equation \eqref{eq_16} is compared with the output of 
different numerical solvers of Pythons 
open source library \emph{SciPy}. The 
different numerical ODE solvers are within the  
method \emph{solving initial value problem}  
\emph{(solve\_ivp)} and each solver comes
with several 
options, which are 
documented in \cite{noauthor_scipys_2021}. 
The results and the comparison 
is given in section \ref{sec_Solve_ODE_Methods}. However,
the conclusion of this investigation is that,  
\emph{SciPy}'s Runge Kutta of 
order 5 (\emph{RK45}) can be employed without any 
noticeable loss of accuracy.
\emph{RK45} assumes an accuracy of the fourth-order
method, but steps are taken using the 
fifth-order accurate formula. The occurring deviations to 
the analytically obtained solution \eqref{eq_16} are less 
than milligrams, which exhibits a sufficient 
level of 
accuracy.\newline


\emph{SciPy} allows the user to define the number 
and the location of steps. However, having compared 
its default option with less than $20$ steps
to a ridiculous high number 
of $1460597$ steps, no noticeable deviation 
could be observed. Fewer steps at which calculations 
happen results into a faster execution time. This 
becomes even more clear when recalling that
each step invokes surrogate  evaluations. In case 
of $20$ used steps undertaken, in order to 
solve the cruise fuel consumption ODE, for 
each step in this iterative process, the 
corresponding values 
for $LoD, AoA, TSFC$ with the
input set of $Ma,h, mass$ are necessary.
In simpler terms, each surrogate model
obtain an input set of $Ma,h, mass$ and 
provides the corresponding output for 
$LoD, AoA, TSFC$, which are predictions, thus 
$\tilde{LoD}, \tilde{AoA}, \tilde{TSFC}$. Since, this has to
done be for each step, $20*3= 60$ surrogate 
model invocations 
for solving the cruise fuel burn equation 
once, are required. Therefore, the low number of steps  
and because of the sufficient accuracy of the solution
with \emph{SciPy}'s default settings for \emph{RK45} are chosen 
to solve the cruise fuel burn mass. \newline


As a final recap, the cruise fuel burn 
equation cannot be solved analytically due to explained reasons 
and thus is solved with an outstanding accuracy employing 
\emph{SciPy}'s numerical ODE solver \emph{RK45} within the 
\emph{solve\_ivp} environment applying the default options.


