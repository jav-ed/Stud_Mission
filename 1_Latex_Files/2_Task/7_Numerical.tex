% Follwing Command: The figure from previous section shall not 
% be wihtin this section
\FloatBarrier
\subsection{Numerical approach}
\label{subsec_CDS}

As mentioned in the previous section, an analytical 
solution was pursued. However, the approach 
in section \ref{sec_Shape_Gradients} 
had to be verified. In this subsection, 
the numerical approach for computing the gradients 
of the cruise fuel weight and the total fuel weight 
with respect to the shape parameters shall be 
explained. The three most frequently 
employed methods for 
deriving the gradients numerically are 
forward, backwards and central finite 
differences. In the majority of cases 
central finite differences or also known 
as central differencing scheme (CDS), which exhibits 
the best gradient computing accuracy. The equations 
for forward, backwards and central differencing 
are given in equations \eqref{eq_68}, \eqref{eq_69}
and \eqref{eq_70}, respectively. 

\begin{equation}
    \label{eq_68}
    \frac{df(x)}{dx} \approx \frac{\Delta f(x)}{\Delta x}=
     \frac{f(x+h_s) - f(x)}{h_s}
\end{equation}

\begin{equation}
    \label{eq_69}
    \frac{df(x)}{dx} \approx \frac{\Delta f(x)}{\Delta x}=
     \frac{f(x) - f(x-h_s)}{h_s}
\end{equation}

\begin{equation}
    \label{eq_70}
    \frac{df(x)}{dx} \approx \frac{\Delta f(x)}{\Delta x}=
     \frac{f(x+h_s) - f(x -h_s)}{2\, h_s}
\end{equation}

Here the $f(x),\, h_s$ are denoted as any function with 
the input argument $x$ and the step-size, respectively. 
The choice of the step-size $h_s$ is important 
and thus will be dealt with later in more details.  
As pointed with the approximation sign ($\approx$), 
it is only an approximation of the gradient. The reason for 
this can be explained by viewing 
their derivation. In order to get the one of the 
three forms a given function is approximated by 
the Taylor series and since only an infinite 
long Taylor series can give the accurate solution, 
aborting the Taylor series after one or 
two terms cannot be accurate. Here it can also 
be observed that the gradients only around 
the approximated location can be considered 
acceptable. In other words, the higher the step-size 
$h_s$ gets, the lower the 
quality of the approximation gets. The 
approximation error is also called 
truncation error. Since computers are used 
for calculating the derivate another 
error must be considered. The round off error, 
occurs due to the nature of storing 
digits into the computer memory or RAM.
Having a decimal number, it 
consists of multiple digits. Each digit 
claims some memory and in case providing 
all numbers consisting of many digits
all desired memory, two obstacles 
are faced. The memory capacities can be 
overwhelmed quite easy. The second problem is that, 
if only one number already contains more than 100 
digits, it's processing by the
hardware runs into  
infeasibly high execution time. Therefore, 
state-of-the-art accuracy uses double 
precision to store numbers. Already here 
an approximation of the real number is done.
This computer based error is called 
round off error.\newline 

Up to now, two kinds of errors were introduced.  
The third error is called the abortion error. 
An infinite Taylor series would 
remove the abortion 
error, thus it would be vanished. In the introduction 
the forward and backwards finite differing methods 
were stated to be less accurate than the central 
finite differencing scheme. The reason for that can 
be seen by their derivation. In short, 
forward and backwards differing are methods of first order,
 whereas 
the central differencing scheme is a method of 
second order accuracy.\newline

Since the CDS is considered to be more 
accurate in most applications, it is implemented 
in \emph{missioninformer}. The equation \eqref{eq_70} 
shows that CDS can be used as a block-box method, 
where $f$ can by any computation in \emph{missioninformer} 
and $x+h_s,\, x- h_s$ represents the parameter, for which 
the gradient is desired. It needs to be changed 
by adding and subtracting the step-size $h_s$.
However, applying this theory is not straight forward 
for the calculations done in \emph{missioninformer} and 
thus the CDS does not act as a black-box function.
The reasons for that are the subject of the 
upcoming discussion.\newline 

The gradient of the total fuel mass 
$m_f$ with respect to the shape 
parameter is the object of desire
$\frac{dm_f}{\vec{p}}$. The reason 
why the shape parameter variable 
$\vec{p}$ is chosen to be formulated as 
a vector is to highlight, that 
not only one gradient is calculated.
Because no optimization can 
be accomplished with having defined 
only one shape parameter, it is more 
correct to talk about the shape 
parameter vector $\vec{p}$, which 
notation from here on will be continued 
to be followed.
In order to apply the straight forward CDS as described 
with equation \eqref{eq_70}, the relationship for each 
variable $LoD, AoA, TSFC$ to all the shape parameters 
$\vec{p}$, as stated in equation \eqref{eq_72}
must be known.

\begin{equation}
    \label{eq_72}
    LoD(\vec{p}),  \quad AoA(\vec{p}), \quad TSFC(\vec{p})
\end{equation}

However, the encountered problem is that these relationships 
from equation \eqref{eq_72} are not known, thus the 
black box straight forward CDS presented in equation 
\eqref{eq_70} cannot be applied.
The following simple and mathematical correct workaround was found. 

\begin{align}
    LoD_{pertubated} &= h_s \, \dfrac{dLoD}{d\vec{p}} \label{eq_73}\\
    AoA_{pertubated} &=  h_s \, \dfrac{dAoA}{d\vec{p}}\\
    TSFC_{pertubated} &=  h_s \, \dfrac{dTSFC}{d\vec{p}} \label{eq_74}
\end{align}


With these background, two different workflows, for 
solving the gradients of the total fuel mass 
with respect to the shape parameters can be 
discussed. The difference in both occurs in 
using the auxiliary cruise fuel fraction term 
$fr_{cr}$ given in the following equation 
or inserting it directly into the remaining 
equations as presented in section \eqref{sec_Fuel_Mass_iter}

\begin{equation}
    fr_{cr} =  \frac{m_s - f_{cr}}{m_s} \tag{\ref{eq_20_Cruise_Fraction}}
\end{equation}

In case of direct insertion, the equation 
\eqref{eq_63} is obtained, as explained 
in section \ref{sec_Shape_Gradients} and 
is referred to as \emph{direct method}
\begin{equation}
    m_f = \dfrac{ff_2 \, m_{f, zero} + ff \, m_{cr} - ff \, ff_2 \, m_{f, zero}}
    {ff_2 \, (ff - f_{reser})} \tag{\ref{eq_63}}
\end{equation}

For defining the fuel cruise fraction 
$fr_{cr}$ explicitly the method is 
referred to as \emph{indirect method}. 
The indirect method's solution  
is given in equation \eqref{eq_75},
which derivation can be seen in section \ref{sec_Fuel_Mass_iter}
with the equations \eqref{eq_17} to \eqref{eq_20}.

\begin{equation}
    \label{eq_75}
    m_{f} = \frac{fa - f_{zero}}{f_{reser} -fr_{cr} 
    \,fr_1 \, fr_2\,fr_3 \, fr_4 \, 
    fr_5\, fr_6\, fr_7 }
\end{equation}

After analyzing two main differencing could be
observed. The direct method often claims 
more fix-point iterations. In contrast, 
it was found to be more accurate 
by satisfying lower convergence criteria. In case 
of low convergence parameter, the 
indirect method stucked inside a loop, where 
it jumped between two values.
Since the \emph{missioninformer} has a limit 
of maximal allowed iterations of 500, 
the iterative method would stop 
after 500 iterations and thus would 
require more iterations than the 
direct method. However, a relative 
and absolute tolerance \cite{noauthor_numpys_2021} value of 
$5 \mathrm{e}{-5}$ is assumed to be accurate enough.
With this tolerance value, no 
permanent stay for the indirect method inside 
a loop could be observed.
Also, in cases where the missions were 
defined reasonably, only 4 iterations 
were required for obtaining the 
gradient of the total fuel mass $\frac{dm_f*}{d\vec{p}}$ 
per gradient component and perturbation 
direction.
Therefore, the \emph{missioninformer} default method 
is chosen to be the indirect method.\newline 

The parameter which defines the quality of 
the gradients by applying CDS is the 
choice of the step-size $h_s$. In order 
to find an appropriate $h_s$ two different 
missions were defined. One, which 
raised the \emph{constraint violation check} 
to output warning statements and 
one without. Thus, these missions can be 
considered to be hard to be solved (mission 1)
and the opposite (mission 2), 
respectively. For both missions, 
applying both methods (direct and indirect)
a step-size study is performed. In case 
of the harder to solve mission 1, the number 
of maximal fix-point iterations was reached. The 
missions definitions can be 
found in table \ref{tab_5_Mission_Step_Size}.
The difference 
between current table \ref{tab_5_Mission_Step_Size}
and the table \ref{tab_2} from subsection 
\ref{subsec_INvestigate_Surro} is only 
in the definition of the range for mission 1.
In table 
\ref{tab_5_Mission_Step_Size} this has been 
reduced by $1000$ meters.

\begin{table}[!h]
        \centering
        \begin{tabular}{l| c c c c c c }
            \textbf{Mass} & \multicolumn{1}{c}{\textbf{Mission 1}} &&& 
            \multicolumn{1}{c}{\textbf{Mission 2}}\\
            \hline
            maximal takeoff & $245 \mathrm{e}{3}$ &&&      $245\mathrm{e}{3}$         &&\\
            maximal landing & $192.2 \mathrm{e}{3}$ &&&      $192.2\mathrm{e}{3}$         &&\\
            operating empty & $132.5\mathrm{e}{3}$ &&&      $132.5\mathrm{e}{3} $         &&\\
            manufactures empty & $119.2\mathrm{e}{3} $ &&&      $119.2\mathrm{e}{3} $         &&\\
            maximum zero fuel  & $180.5\mathrm{e}{3} $ &&&      $180.5\mathrm{e}{3} $         &&\\
            maximum fuel & $107.6\mathrm{e}{3}$ &&&      $107.6\mathrm{e}{3}$         &&\\
            design payload & $38.52\mathrm{e}{3}$ &&&      $33.60\mathrm{e}{3} $         &&\\
            maximum payload & $48.00\mathrm{e}{3}$ &&&      $48.00\mathrm{e}{3}  $         &&\\
            \hline
            &  &&&               &&\\
            \textbf{Flying parameters} &  &&&               && \\
            \hline 
            $Ma$ & $0.83$ &&&      $0.82$         &&\\
            $h$ & $10.668\mathrm{e}{3}$ &&&      $11.000\mathrm{e}{3}  $         &&\\
            $R_{cr}$ & $10186.0\mathrm{e}{3} $ &&&      $5185.6\mathrm{e}{3} $         &&\\
            \hline
            &  &&&               &&\\
            \textbf{Others} &  &&&               && \\
            \hline 
            step-size & $1\mathrm{e}{-5} $&&&      $1\mathrm{e}{-5}$         &&\\
            weight-factor & $0.7$ &&&      $0.3$         &&\\
        \end{tabular}
        \caption{Missions definitions for investigation on step-size}
        \label{tab_5_Mission_Step_Size}
\end{table} 

\FloatBarrier
The convergence behavior of both missions and 
both methods is presented in section \ref{sec_Step_Size_Study}.
However, the final result is that missionsinformers
default step-size is chosen to be $h_s = 1 \mathrm{e}{-5}$.