
\section{Mission fuel mass iteration}
\label{sec_Fuel_Mass_iter}

For solving the cruise fuel burn equation \eqref{eq_10} 
the present total mass of the airplane $m_s$, which 
is the starting mass for the cruise segment, is required.
However, the cruise starting mass $m_s$ is not known beforehand.
It depends on the fuel mass of the remaining segments fuel masses. 
These are the fuel 
weights of descent, landing and taxi to parking. As explained 
in section \ref{sec_Cruise_Mission_Fuel}, only the 
cruise fuel weight $m_{fe}$ for
the given mission or missions is calculated by solving the ODE 
\eqref{eq_10}. In this section $m_{fe}$ is denoted as $m_{cr}$
in order to highlight that it is only the fuel mass for 
the cruise segment. The required fuel masses for the remaining
flight segments are obtained through empirical fuel fractions 
from table \ref{tab_1}. The process for computing the total fuel
mission mass 
$m_f$, the cruise fuel $m_{cr}$ and the resulting 
cruise starting mass $m_s$  can be viewed 
in the figure \ref{fig_4_Mass_Workflow}.\newline 
%
%
\begin{figure} [!h]
    \resizebox{\textwidth}{!}{
    
%%% Preamble Requirements %%%
% \usepackage{geometry}
% \usepackage{amsfonts}
% \usepackage{amsmath}
% \usepackage{amssymb}
% \usepackage{tikz}

% Optional packages such as sfmath set through python interface
% \usepackage{sfmath}

% \usetikzlibrary{arrows,chains,positioning,scopes,shapes.geometric,shapes.misc,shadows}

%%% End Preamble Requirements %%%

\input{/home/jav/Progs/Virt_Env/writing/lib/python3.12/site-packages/pyxdsm/diagram_styles.tex}
\begin{tikzpicture}

\matrix[MatrixSetup]{
%Row 0
\node [DataIO] (output_ode_Solv) {$\begin{array}{c}\text{initial starting and mission fuel mass } ($$m_{s,0}, m_{f,0})$$\end{array}$};&
&
&
\\
%Row 1
\node [Function] (ode_Solv) {$\begin{array}{c}\text{ODE solver}\end{array}$};&
\node [DataInter] (ode_Solv-mass_Calc) {$\begin{array}{c}$$m_{cr}$$\end{array}$};&
&
\\
%Row 2
\node [DataInter] (mass_Calc-ode_Solv) {$\begin{array}{c}\text{new starting and mission fuel mass }($$m_{s,n}, m_{f,n})\end{array}$};&
\node [Function] (mass_Calc) {$\begin{array}{c}\text{mass recalculation}\end{array}$};&
&
\node [DataIO] (right_output_mass_Calc) {$m_{s}*, m_{cr}*, m_{f}*$};\\
%Row 3
&
&
&
\\
};

% XDSM process chains


\begin{pgfonlayer}{data}
\path
% Horizontal edges
(ode_Solv) edge [DataLine] (ode_Solv-mass_Calc)
(mass_Calc) edge [DataLine] (mass_Calc-ode_Solv)
(mass_Calc) edge [DataLine] (right_output_mass_Calc)
% Vertical edges
(ode_Solv-mass_Calc) edge [DataLine] (mass_Calc)
(mass_Calc-ode_Solv) edge [DataLine] (ode_Solv)
(ode_Solv) edge [DataLine] (output_ode_Solv);
\end{pgfonlayer}

\end{tikzpicture}

    }
    \caption{Initial mass and mass fix-point iteration workflow}
    \label{fig_4_Mass_Workflow}
\end{figure}
%
%

Since the cruise starting mass $m_s$ is not known 
before knowing the cruise and total fuel masses ($m_{cr}, m_{f}$), 
the initial cruise starting and total fuel mass ($m_{s,0}, m_{f,0}$)
are estimated based on the equations \eqref{eq_17} to 
\eqref{eq_19}. The variables for the equations 
$m_{takeoff,max},\, f_{zero},\, f_{eng,start}$
$m_{oe},\, m_{payload}$ are denoted 
as maximal takeoff, zero fuel, 
engine start, operating empty weight and design
payload, respectively. The fuel fractions 
$fr_1 \,, fr_2\,,fr_3 \,, fr_4$ are for the following 
flight segments, after engine start, taxi to runway, takeoff 
and climb and acceleration, respectively.

\begin{equation}
    \label{eq_17}
    f_{zero} = m_{oe} + m_{payload}
\end{equation}

\begin{equation}
    \label{eq_18}
    m_{f,0} = m_{takeoff,max} - f_{zero}
\end{equation}


\begin{equation}
    \label{eq_19}
    m_{s,0} = \left(f_{zero} + f_{eng,start}\right) \, fr_1 \, fr_2\,fr_3 \, fr_4
\end{equation}



By proving $m_{s,0}$ to the cruise fuel ODE 
\eqref{eq_10} the corresponding cruise fuel $m_{cr}$ is 
obtained, which is passed to the block \emph{mass recalculation}.
Here the new cruise starting and total fuel masses 
($m_{s,n}, m_{f,n}$) are calculated. For computing 
former ($m_{s,n}$) the equation \eqref{eq_19} is still 
valid, however, computing the new total fuel mass $m_{f,n}$
two different possibilities were found. One by using all 
known mass relationships straight forward and solving $m_{f,n}$ 
numerically, e.g. the Newtons method. The second 
method was obtained by inserting the 
mass equations into each and 
performing transformations. With that 
an analytical solution was obtained. Here, only 
the derivation of the analytical solution shall 
be explained.\newline 

The demand for  
$m_{f,n}$ is stated in equation \eqref{eq_20}, where 
$m_{after,cr}$ is the mass after
the cruise segment and $ fr_{cr}$ is introduced as auxiliary
fuel fraction 
for the cruise segment and is given 
in equation \eqref{eq_20_Cruise_Fraction}. In words, the mass after 
the cruise segment divided by the mass after acceleration 
and climb must be equal to the cruise fuel fraction.


\begin{equation}
    \label{eq_20_Cruise_Fraction}
    fr_{cr} =  \frac{m_s - f_{cr}}{m_s}
\end{equation}

\begin{equation}
    \label{eq_20}
    \frac{m_{after,cr}}{m_{s} }- fr_{cr} \overset{!}{=} 0
\end{equation}

The mass after the cruise segment $m_{after,cr}$
 can be written in 
the form of equation \eqref{eq_21}, where 
$fr_{reser}$ is the known fuel reserve fraction, 
provided in table 
\ref{tab_1}. The fuel fractions 
$fr_5\,, fr_6\, , fr_7$ stand for taxi to parking,
landing and descent, respectively and are also 
given in table \ref{tab_1}. The unknown
in this equation is the new total fuel mass 
$m_{f,n}$. Applying insertions and mathematical 
transformations with the equations \eqref{eq_19} to 
\eqref{eq_21}, $m_{f,n}$ can be written as 
equation \eqref{eq_23}, where $fa$ as an 
auxiliary term is given in equation \eqref{eq_22}.


\begin{equation}
    \label{eq_21}
    m_{after,cr} = \frac{f_{zero} + m_{f,n} \, fr_{reser}}
            {fr_5\, fr_6\, fr_7} = 0
\end{equation}

\begin{equation}
    \label{eq_22}
    fa = f_{zero} \, fr_{cr} \, fr_1 \, fr_2\,fr_3 \, fr_4 \, 
    fr_5\, fr_6\, fr_7
\end{equation}

\begin{equation}
    \label{eq_23}
    m_{f,n} = \frac{fa - f_{zero}}{f_{reser} -fr_{cr} 
    \,fr_1 \, fr_2\,fr_3 \, fr_4 \, 
    fr_5\, fr_6\, fr_7 }
\end{equation}


% ==================================================
% =========== RECAP ===============================
% ==================================================

To recap, after an initial guess for
starting cruise and total fuel mass 
$m_{s,0}, m_{f,0}$, they are passed 
to the ODE solver, which outputs 
its corresponding cruise fuel weight $m_{cr}$.
With the equation \eqref{eq_20_Cruise_Fraction} 
an auxiliary fuel fraction for the cruise 
segment is calculated and used in the 
\emph{mass recalculation} box, depicted
 in figure \ref{fig_4_Mass_Workflow}.
Here by employing the equations \eqref{eq_19}
and \eqref{eq_23} the new cruse starting 
and total fuel mass $m_{s,n}, m_{f,n}$ are 
obtained, respectively.
This process is repeated as long 
the convergence criteria 
$\Delta =  m_{f,n} -  m_{f,old} \leq \epsilon$ is met, where 
$\epsilon$ is a tolerance parameter. 
The relative and absolute tolerance parameter are 
set to be $5\mathrm{e}{-5}$ \cite{noauthor_numpys_2021}.
Such an iterative method is also called \emph{fix-point iteration}.
Since the described workflow is iterative, the resulting 
change and their interdependencies  (mass snowball effect)
are taken into account.
In case a convergence is achieved, the final 
values for the cruise starting, cruise fuel and 
total fuel mass \ref{fig_4_Mass_Workflow}
are found ($m_{s}*, m_{cr}*, m_f*$).\newline

As mentioned above, the problem \eqref{eq_20} can 
be solved numerically or analytically. In order 
to verify the analytical solution it 
has been compared successfully with the 
numerical Newtons' method. The presented 
workflow defining the 
auxiliary term $fr_{cr}$ 
\eqref{eq_20_Cruise_Fraction} can be 
further simplified by directly inserting 
$fr_{cr}$ into the equations. This 
results into equation \eqref{eq_63} 
and its derivation with respect to 
the shape parameter is explained 
in more details in section \ref{sec_Shape_Gradients}. 
The results of 
this direct (without auxiliary term) 
approach differed marginally 
from the previous auxiliary method.
Therefore, \emph{Matlab} and \emph{SymPy}
(open source 
Python library) were used in order 
to verify the undertaken 
mathematical transformations. The 
symbolic results confirmed the handcrafted 
analytical solution.\newline


One interesting finding could be observed. 
Consider the
figure \ref{fig_4_Mass_Workflow}, which shows 
the reoccurring or iterative process,  
thus the fix-point iteration.
Depending on applying
the auxiliary or the direct version, different 
numbers of iterations were needed to 
get to the final solutions for 
$m_{s}*, m_{cr}*, m_f*$. As mentioned, also the  
end values for $m_{s}*, m_{cr}*, m_f*$ 
would differ, however only around the 10th 
decimal place. By changing the missions 
definitions, it could be observed that in most cases 
the direct version required more iterations to 
converge. In some cases, more than two
to four times more iterations 
were required. Thus, \emph{missioninformer's} default 
method to solve equation \eqref{eq_20} is 
by employing the auxiliary term $fr_{cr}$.\newline





% ==================================================
% =========== Check violation ======================
% ==================================================
Finally, the \emph{missioninformer's} \emph{constraint 
violation check} shall be 
elaborated. Since the user is completely free 
in defining the desired mission or missions, it is 
possible for an unpracticed user to define an
unrealistic high range or payload. Also, 
possible is that, e.g. the predicted
$\tilde{LoD}$ with the preliminary aircraft synthesis
software turned out to be too optimistic and at the 
detailed Reynold-Averaged-Navier-Stokes-Equations-
based (RANS) aerodynamics analysis. In such cases 
the required fuel masses or other weights can exceed 
the mass constraints set by the top level aircraft 
requirements (TLAR) and the preliminary design.
 Therefore, verifying whether 
given constrains are violated is necessary. The 
workflow with the \emph{constraint violation check} is 
depicted in figure \ref{fig_5_Constraint_Violation}.
As it can be observed, it is performed after each mass 
recalculation. 
%
% ==============================================
% ========== Constraint Viol Workfflow =========
% ==============================================
\begin{figure} [!h]
    \hspace*{-4cm} 
    \resizebox{1.2\textwidth}{!}{
    
%%% Preamble Requirements %%%
% \usepackage{geometry}
% \usepackage{amsfonts}
% \usepackage{amsmath}
% \usepackage{amssymb}
% \usepackage{tikz}

% Optional packages such as sfmath set through python interface
% \usepackage{sfmath}

% \usetikzlibrary{arrows,chains,positioning,scopes,shapes.geometric,shapes.misc,shadows}

%%% End Preamble Requirements %%%

\input{/home/jav/Progs/Virt_Env/writing/lib/python3.12/site-packages/pyxdsm/diagram_styles.tex}
\begin{tikzpicture}

\matrix[MatrixSetup]{
%Row 0
\node [DataIO] (output_ode_Solv) {$\begin{array}{c}\text{initial starting} \\ \text{and mission fuel } \\ \text{mass } ($$m_{s,0}, m_{f,0})$$\end{array}$};&
&
\node [DataIO] (output_constr_Viol) {$\begin{array}{c}\text{mission input} \\ \text{file}\end{array}$};&
&
\\
%Row 1
\node [Function] (ode_Solv) {$\begin{array}{c}\text{ODE solver}\end{array}$};&
\node [DataInter] (ode_Solv-mass_Calc) {$\begin{array}{c}$$m_{cr}$$\end{array}$};&
&
&
\\
%Row 2
\node [DataInter] (mass_Calc-ode_Solv) {$\begin{array}{c}\text{new starting and} \\ \text{mission fuel mass } \\ (m_{s,n}, m_{f,n})\end{array}$};&
\node [Function] (mass_Calc) {$\begin{array}{c}\text{mass recalculation}\end{array}$};&
\node [DataInter] (mass_Calc-constr_Viol) {$\begin{array}{c}m_{s,n}, m_{f,n}\end{array}$};&
&
\node [DataIO] (right_output_mass_Calc) {$\begin{array}{c}m_{s}*, m_{cr}*, \\  m_{f}*\end{array}$};\\
%Row 3
&
&
\node [Function] (constr_Viol) {$\begin{array}{c}\text{check constraint} \\  \text{violation}\end{array}$};&
&
\\
%Row 4
&
&
&
&
\\
};

% XDSM process chains


\begin{pgfonlayer}{data}
\path
% Horizontal edges
(ode_Solv) edge [DataLine] (ode_Solv-mass_Calc)
(mass_Calc) edge [DataLine] (mass_Calc-ode_Solv)
(mass_Calc) edge [DataLine] (mass_Calc-constr_Viol)
(mass_Calc) edge [DataLine] (right_output_mass_Calc)
% Vertical edges
(ode_Solv-mass_Calc) edge [DataLine] (mass_Calc)
(mass_Calc-ode_Solv) edge [DataLine] (ode_Solv)
(mass_Calc-constr_Viol) edge [DataLine] (constr_Viol)
(ode_Solv) edge [DataLine] (output_ode_Solv)
(constr_Viol) edge [DataLine] (output_constr_Viol);
\end{pgfonlayer}

\end{tikzpicture}

    }
    \caption{Second workflow with some more details}
    \label{fig_5_Constraint_Violation}
\end{figure}
%
%
In case of a found constraint 
violation, a colored warning statement will be given 
to highlight an eventual inconsistency with 
the possible incorrectness of 
the mission inputs definition. The following 
checks are made, where 
$m_{takeoff,max}, \,m_{eng,start},$ $m_{land,max}, \, m_{aft,desc}, $
$m_{f,max}, \, m_{f}$
are denoted as the weights of 
maximal takeoff, engine start, maximal landing, after 
descent, maximal fuel and actual required total fuel, respectively.

\begin{equation}
    \label{eq_21_Check_viloation}
        \begin{aligned}
            m_{takeoff,max} <  m_{eng,start}\\
            m_{land,max} < m_{aft,desc}\\
            m_{f,max} < m_{f}
        \end{aligned}
\end{equation}