\section{Evaluating different surrogate models}
\label{sec_Diff_Surrogates}
In this section the different surrogate models 
Kriging (Gaussian), TPS and RBF are 
explored, which are all described 
in section \ref{sec_Surrogates}. Since most 
explanations for the upcoming explorations 
were given in subsection \ref{subsec_INvestigate_Surro}, 
only necessary details will be repeated.
The two missions which are going 
to be used for the following investigations are 
shown in table \ref{tab_2}.
Both missions passed 
the constraints check 
violation successfully and thus 
the number of the fix point iterations 
for calculating the gradients is low.
The table \ref{tab_7} shows the 
options for Kriging and TPS and 
\emph{SciPy}s RBF. For all \emph{SciPy}s RBF 
kernels a \emph{smooth} value of 
$0.1$ \cite{noauthor_rbf_2021} was used.  More
information about the settings and kernel 
is given in section \ref{sec_Surrogates}. \newline 

%
\begin{table}[!h]
        \centering
        \begin{tabular}{l c c |l }
            \multicolumn{3}{c|}{\textbf{\emph{SMARTy} 
            Kriging and TPS}} & 
            \multicolumn{1}{l}{\textbf{\emph{SciPy}}} \\
            \hline
%           ------------ Second Title -----------------------
            \multicolumn{1}{l}{\textbf{Name}} & 
            \multicolumn{1}{c}{\textbf{Augmentation}} &
            \multicolumn{1}{c}{\textbf{Regularization}} &
            \multicolumn{1}{|c}{\textbf{RBF Kernel}}\\
            \hline
            A & $-1$ &False &  multiquadric       \\
            B & $0$ &False  & inverse      \\
            C & $+1$ &False &  gaussian        \\
            D & $+2$ &False &  linear       \\
            E & $-1$ &True  &  cubic   \\
            F & $0$ &True   &  thin plate   \\
            G & $+1$ &True  &   -  \\
            H & $+2$ &True  &  -   \\
        \end{tabular}
        \caption{Kriging, TPS and RBF parameter used for investigation 
        surrogate quality}
        \label{tab_7}
\end{table} 
%
\FloatBarrier
Similarly to the results sections before, much 
research has been done. However, not 
all results can be shown, rather 
results which are supposed to sum the 
found results up will be shown and elaborated.
Recall, 35 effective or fully trimmed sample points 
were given as the input data for training the surrogate 
models. The investigations were performed 
for state and gradients as well. In 
subsection \ref{subsec_INvestigate_Surro} 
three different versions were introduced, \emph{V38,V28}
and \emph{V18}, where each version only effectively
have three less sample points than its declarations name.
According to subsection 
\ref{subsec_INvestigate_Surro} for V38 the
\emph{Leave-one-out cross-validation} is applied. Whereas 
for V28 and V18 the \emph{Root Mean Square Error} (RMSE)
is calculated by having 10 and 20 additional 
sample points, respectively. For
this additional sample points the true value of output variables 
$AoA, LoD, TSFC$ and their gradients is known.\newline


The first exploration, which will be considered is 
by using the 38 sample data (effective 35 trimmed samples).
For this V38, the total missions fuel mass $m_f$, 
the cruise fuel weight $m_{cr}$ and the cruise 
starting mass $m_s$ were monitored. However, 
since only for the cruise segment 
a physical based equation is applied, it is sufficient 
to only portray the curve of the total 
fuel mass $m_f$ with the different 
surrogate models.
As concluded in section \ref{sec_Step_Size_Study}, the indirect 
method is preferred to the direct method. Therefore, 
the indirect method is used in order to perform this 
surrogate investigation.
Figure \ref{fig_88} and \ref{fig_89} exhibit the total 
fuel mass $m_f$ of mission 1, by 
employing Kriging and TPS, respectively. 
Figures \ref{fig_90} and \ref{fig_91} show
the same for mission 2.
The 
horizontal axis gives information about the chosen 
model intern parameter, which can be read from table \ref{tab_7}.
In both cases it can be observed, 
changing the model intern parameter does have an impact 
of the outcome ($m_{f}$). For both surrogate models
with their different options the deviation is less than 
$260$ kilogram. In case of Kriging and TPS in figures \ref{fig_88}
and \ref{fig_91}, respectively it is only $25$ kilogram. Such a deviation
is considered to be acceptable.

\begin{figure}[!h]
    %\vspace{0.5cm}
    \begin{minipage}[h]{0.46\textwidth}
        \centering
        \includegraphics[width =\textwidth]{2_Figures/3_Task/3_Surrogate/1_Fleiss/1_Krig/1_Miss_1_Final.pdf}
        \caption{Final mission fuel weight $m_f$, Mission 1, Kriging variants}
        \label{fig_88}    
    \end{minipage}
    \hfill
    \begin{minipage}{0.46\textwidth}
        \centering
        \includegraphics[width =\textwidth]{2_Figures/3_Task/3_Surrogate/1_Fleiss/2_TPS/1_Miss_1_Final.pdf}
        \caption{Final mission fuel weight $m_f$, Mission 1, TPS variants}
        \label{fig_89}    
    \end{minipage}
\end{figure} 


\begin{figure}[!h]
    %\vspace{0.5cm}
    \begin{minipage}[h]{0.46\textwidth}
        \centering
        \includegraphics[width =\textwidth]{2_Figures/3_Task/3_Surrogate/1_Fleiss/1_Krig/1_Miss_2_Final.pdf}
        \caption{Final mission fuel weight $m_f$, Mission 2, Kriging variants}
        \label{fig_90}    
    \end{minipage}
    \hfill
    \begin{minipage}{0.46\textwidth}
        \centering
        \includegraphics[width =\textwidth]{2_Figures/3_Task/3_Surrogate/1_Fleiss/2_TPS/1_Miss_2_Final.pdf}
        \caption{Final mission fuel weight $m_f$, Mission 2, TPS variants}
        \label{fig_91}    
    \end{minipage}
\end{figure} 

 \FloatBarrier
Figures \ref{fig_92} to \ref{fig_95} exhibits the number of 
the fix point iterations, which were required for 
only mass calculation for the states, as explained 
in section \ref{sec_Fuel_Mass_iter}. It can be 
seen that in no event more than 4 state fix point 
iterations are required. The difference between the 
gradient and the state fix point iterations can 
be simplified as follows. The gradient fix point iterations number 
is approximately twice as high as the state version. This is so, because 
the CDS evaluates the functions 
value at two points, thus the state fix point iteration 
is run twice.\newline

Figures \ref{fig_96} to \ref{fig_99} depict the same 
kind of study for RBF and its options.
It can be observed, that RBF predictions highly depend on the chosen 
kernel. Also, the number of the state fix point iterations are much 
higher.
Figures \ref{fig_100} and \ref{fig_101} show a comparison 
with all surrogate models and their options for 
mission 1 and 2, respectively. It can be seen, that 
only in case of mission 1, depending 
on RBF's kernel a match with Kriging and TPS 
occurs. In general, it can be stated that the
\emph{SciPy} RBF's output is noticeably different to 
\emph{SMARTy}'s Krigings and 
TPS output. However, Krigings and TPS output 
exhibit a comparatively low deviation. Furthermore, 
it can be said that RBF has a much higher demand of 
state fix point iterations, which makes is 
unfeasible for application purposes.

\FloatBarrier
 \begin{figure}[!h]
    %\vspace{0.5cm}
    \begin{minipage}[h]{0.46\textwidth}
        \centering
        \includegraphics[width =\textwidth]{2_Figures/3_Task/3_Surrogate/1_Fleiss/1_Krig/4_Miss_1_Nr_Iter.pdf}
        \caption{Number of the state fix-point iterations, Mission 1, Kriging variants}
        \label{fig_92}    
    \end{minipage}
    \hfill
    \begin{minipage}{0.46\textwidth}
        \centering
        \includegraphics[width =\textwidth]{2_Figures/3_Task/3_Surrogate/1_Fleiss/2_TPS/4_Miss_1_Nr_Iter.pdf}
        \caption{Number of the state fix-point, Mission 1, TPS variants}
        \label{fig_93}    
    \end{minipage}
\end{figure} 

\begin{figure}[!h]
    %\vspace{0.5cm}
    \begin{minipage}[h]{0.46\textwidth}
        \centering
        \includegraphics[width =\textwidth]{2_Figures/3_Task/3_Surrogate/1_Fleiss/1_Krig/4_Miss_2_Nr_Iter.pdf}
        \caption{Number of the state fix-point iterations, Mission 2, Kriging variants}
        \label{fig_94}    
    \end{minipage}
    \hfill
    \begin{minipage}{0.46\textwidth}
        \centering
        \includegraphics[width =\textwidth]{2_Figures/3_Task/3_Surrogate/1_Fleiss/2_TPS/4_Miss_2_Nr_Iter.pdf}
        \caption{Number of the state fix-point, Mission 2, TPS variants}
        \label{fig_95}    
    \end{minipage}
\end{figure} 

% -------------------- RBF STATE -------------------------------



\begin{figure}[!h]
    %\vspace{0.5cm}
    \begin{minipage}[h]{0.46\textwidth}
        \centering
        \includegraphics[width =\textwidth]{2_Figures/3_Task/3_Surrogate/1_Fleiss/3_RBF/1_Miss_1_Final.pdf}
        \caption{Final mission fuel weight $m_f$, Mission 1, RBF variants}
        \label{fig_96}    
    \end{minipage}
    \hfill
    \begin{minipage}{0.46\textwidth}
        \centering
        \includegraphics[width =\textwidth]{2_Figures/3_Task/3_Surrogate/1_Fleiss/3_RBF/1_Miss_2_Final.pdf}
        \caption{Final mission fuel weight $m_f$, Mission 2, RBF variants}
        \label{fig_97}    
    \end{minipage}
\end{figure} 


\begin{figure}[!h]
    %\vspace{0.5cm}
    \begin{minipage}[h]{0.46\textwidth}
        \centering
        \includegraphics[width =\textwidth]{2_Figures/3_Task/3_Surrogate/1_Fleiss/3_RBF/4_Miss_1_Nr_Iter.pdf}
        \caption{Number of state fix point iterations, Mission 1, Kriging variants}
        \label{fig_98}    
    \end{minipage}
    \hfill
    \begin{minipage}{0.46\textwidth}
        \centering
        \includegraphics[width =\textwidth]{2_Figures/3_Task/3_Surrogate/1_Fleiss/3_RBF/4_Miss_2_Nr_Iter.pdf}
        \caption{Number of state fix point iterations, Mission 2, TPS variants}
        \label{fig_99}    
    \end{minipage}
\end{figure} 

\begin{figure}[!h]
    %\vspace{0.5cm}
    \begin{minipage}[h]{0.46\textwidth}
        \centering
        \includegraphics[width =\textwidth]{2_Figures/3_Task/3_Surrogate/1_Fleiss/4_Compare/2_All/1_Miss_1_Final.pdf}
        \caption{Total fuel mass $m_{f}$ for all surrogate models, Mission 1}
        \label{fig_100}    
    \end{minipage}
    \hfill
    \begin{minipage}{0.46\textwidth}
        \centering
        \includegraphics[width =\textwidth]{2_Figures/3_Task/3_Surrogate/1_Fleiss/4_Compare/2_All/1_Miss_2_Final.pdf}
        \caption{Total fuel mass $m_{f}$ for all surrogate models, Mission 2}
        \label{fig_101}    
    \end{minipage}
\end{figure} 

\FloatBarrier
The next step is to consider the effect of a decreasing number 
of training samples and its effect on the surrogate models.
This is done only for Kriging and TPS, since 
\emph{SciPy}'s RBF showed 
a too high sensitivity on the chosen kernel and also 
required too many fix point iterations. 
Kriging and TPS have the same 8 options. 
The total fuel mass $m_{f}$ is plotted 
in the figures \ref{fig_102} to \ref{fig_105}. 
Comparing  $m_{f}$ for mission 1 
with Kriging 
and TPS via figures \ref{fig_102} and \ref{fig_103}, respectively, 
the following can be observed. The vertical scaling axis 
of Kriging is finer, which means, a high change on the left 
figure results in a low change on the right figure (TPS).
The maximum difference in the version which could occur 
for Kriging is $350$ kg and for TPS is $1600$ kg. Therefore, 
the absolute deviation needs to be calculated, which are 
shown in figures \ref{fig_104} and \ref{fig_105} for 
Kriging and TPS, respectively. In the mentioned figures, 
the green and red curves are more interesting to us, since 
it is assumed that V38 is the most accurate version and 
thus is considered as the basis. For mission 1, Kriging 
V18 and V28 are clearly closer to V38. In other words, 
Kriging tends to deliver with fewer data 
results closer to the fine sampling outputs than TPS. In case 
Kriging would also be the most accurate method
(see subsection \ref{subsec_Krig_TPS_Accu}), it 
could be preferred for training surrogate models 
with a fewer number of training data.

\begin{figure}[!h]
    %\vspace{0.5cm}
    \begin{minipage}[h]{0.46\textwidth}
        \centering
        \includegraphics[width =\textwidth]{2_Figures/3_Task/3_Surrogate/5_Fleiss_V_All/1_Regular/1_Mission_1_Krig_Fuel_Weight.pdf}
        \caption{Total fuel mass $m_{f}$ for all Versions, Mission 1, Kriging}
        \label{fig_102}    
    \end{minipage}
    \hfill
    \begin{minipage}{0.46\textwidth}
        \centering
        \includegraphics[width =\textwidth]{2_Figures/3_Task/3_Surrogate/5_Fleiss_V_All/1_Regular/3_Mission_1_TPS_Fuel_Weight.pdf}
        \caption{Total fuel mass $m_{f}$ for all Versions, Mission 1, TPS}
        \label{fig_103}    
    \end{minipage}
\end{figure} 

\begin{figure}[!h]
    %\vspace{0.5cm}
    \begin{minipage}[h]{0.46\textwidth}
        \centering
        \includegraphics[width =\textwidth]{2_Figures/3_Task/3_Surrogate/5_Fleiss_V_All/1_Regular/2_Mission_1_Delta_Krig_Fuel_Weight.pdf}
        \caption{ Deviations of the different versions of $m_{f}$, Mission 1, Kriging}
        \label{fig_104}    
    \end{minipage}
    \hfill
    \begin{minipage}{0.46\textwidth}
        \centering
        \includegraphics[width =\textwidth]{2_Figures/3_Task/3_Surrogate/5_Fleiss_V_All/1_Regular/4_Mission_1_Delta_TPS_Fuel_Weight.pdf}
        \caption{ Deviations of the different versions of $m_{f}$, Mission 1, TPS}
        \label{fig_105}    
    \end{minipage}
\end{figure} 



\FloatBarrier
The same also must be tested for mission 2. The results are 
presented in figures \ref{fig_106} to \ref{fig_109}. 
The observations made for mission 1 are confirmed by having 
investigated on mission 2.

\begin{figure}[!h]
    %\vspace{0.5cm}
    \begin{minipage}[h]{0.46\textwidth}
        \centering
        \includegraphics[width =\textwidth]{2_Figures/3_Task/3_Surrogate/5_Fleiss_V_All/1_Regular/1_Mission_2_Krig_Fuel_Weight.pdf}
        \caption{Total fuel mass $m_{f}$ for all Versions, Mission 2, Kriging}
        \label{fig_106}    
    \end{minipage}
    \hfill
    \begin{minipage}{0.46\textwidth}
        \centering
        \includegraphics[width =\textwidth]{2_Figures/3_Task/3_Surrogate/5_Fleiss_V_All/1_Regular/3_Mission_2_TPS_Fuel_Weight.pdf}
        \caption{Total fuel mass $m_{f}$ for all Versions, Mission 2, TPS}
        \label{fig_107}    
    \end{minipage}
\end{figure} 

\begin{figure}[!h]
    %\vspace{0.5cm}
    \begin{minipage}[h]{0.46\textwidth}
        \centering
        \includegraphics[width =\textwidth]{2_Figures/3_Task/3_Surrogate/5_Fleiss_V_All/1_Regular/2_Mission_2_Delta_Krig_Fuel_Weight.pdf}
        \caption{Deviations of the different versions of $m_{f}$, Mission 2, Kriging}
        \label{fig_108}    
    \end{minipage}
    \hfill
    \begin{minipage}{0.46\textwidth}
        \centering
        \includegraphics[width =\textwidth]{2_Figures/3_Task/3_Surrogate/5_Fleiss_V_All/1_Regular/4_Mission_2_Delta_TPS_Fuel_Weight.pdf}
        \caption{Deviations of the different versions of $m_{f}$, Mission 2, TPS}
        \label{fig_109}    
    \end{minipage}
\end{figure} 

\FloatBarrier
Up to now, only state solutions were discussed. However, the whole 
investigation process was also done for the gradients.
Since the gradient investigation does not lead to
further findings, only 
the most important results will be shown. Furthermore,
only mission 1 is displayed as a representation of both missions
The figures \ref{fig_110} and \ref{fig_111} 
depict the mean curve of the gradient 
of the total fuel masses w.r.t. all shape parameter.
Mean curve means, that the results of the 8 options 
were used for calculating their respective mean values.\newline

\begin{figure}[!h]
    %\vspace{0.5cm}
    \begin{minipage}[h]{0.46\textwidth}
        \centering
        \includegraphics[width =\textwidth]{2_Figures/3_Task/3_Surrogate/5_Fleiss_V_All/2_Grad/1_Mean_Miss_1_Grad_Krig.pdf}
        \caption{Total fuel mass gradients w.r.t. shape parameters, All versions, Mission 1, Kriging}
        \label{fig_110}    
    \end{minipage}
    \hfill
    \begin{minipage}{0.46\textwidth}
        \centering
        \includegraphics[width =\textwidth]{2_Figures/3_Task/3_Surrogate/5_Fleiss_V_All/2_Grad/2_Mean_Miss_1_Grad_TPS.pdf}
        \caption{Total fuel mass gradients w.r.t. shape parameters, All versions, Mission 1, TPS}
        \label{fig_111}    
    \end{minipage}
\end{figure} 

\FloatBarrier
Next, the required execution time by having chosen 
the different surrogate models is going to be 
displayed. It is measured 
right after invoking \emph{missioninformer} till
the state and gradients for both missions were computed entirely
and stored to the hard disk. In other words, it is 
the execution time of \emph{missioninformer} itself, which 
is measured with different surrogate models.
The results are depicted in figure \ref{fig_112}.
It can be observed that the Kriging versions 
are also more stable in the run time 
than TPS. The impact of different Kriging 
options does not have high impact of 
its execution time. TPS on the other hand, 
can be a little faster than Kriging with some options, however
also can be much slower with other options.
Also, depending on the sample data size, 
a high deviation can be observed, when 
regularization is activated (E to H).


\begin{figure}[!h]
        \centering
        \includegraphics[width =0.8\textwidth]{2_Figures/3_Task/3_Surrogate/5_Fleiss_V_All/2_Grad/7_Exec_Time.pdf}
        \caption{Missioninformers execution time with different surrogate models}
        \label{fig_112}    
\end{figure} 

\FloatBarrier
\subsection{Kriging and TPS accuracy}
\label{subsec_Krig_TPS_Accu}
The objective of this subchapter is to 
present quantifiable information about the accuracy
of Kriging and TPS models 
depending on their user-defined options. The question is, 
which model and which option leads to the 
lowest RMSE. The answer 
to this question is found using 
\emph{SMARTy}'s automated model 
selection capability. The workflow for this 
purpose was repeated shortly in the introduction 
of this section and is given in more detail 
in subsection \ref{subsec_INvestigate_Surro}.
Therefore, the results shall be 
presented straight forward. On the left 
side the best option set for the state 
is given. Here also the declaration 
of the 16 tested models is provided, which 
is valid for the gradient investigation, depicted
on the right side, as well.
For the state calculation, only three 
surrogate models $\tilde{LoD}, \tilde{AoA}, \tilde{TSFC}$
are generated. Therefore, out of the total 16 
options for each of the three different 
surrogate models, only one offers the least 
RMSE. Therefore, for the states, each option 
only once can be found to be the best.
In contrast, the gradient has 
multiple shape parameters and each shape parameter 
has his own best option. Due to this, one 
option can be found to have the least value of 
RMSE multiple times per associated output 
parameter $LoD, AoA, TSFC$.\newline

The figure \ref{fig_113}, \ref{fig_115} and 
\ref{fig_117} depict the state version for 
V18, V28 and V38 respectively. The figures 
\ref{fig_114}, \ref{fig_116} and \ref{fig_118} 
their respective gradient version. Each figure 
on the left side contains a legend, which 
is valid for both sides. A clear distriction 
between Kriging and TPS models can be
made with the letters
\emph{A - H} and \emph{I-P}, respectively.
 Unfortunately,
no absolutely clear winner can be filtered out of these 
results. The distributions are too different for this purpose.
However, a suggestion can still be  made by 
starting with  options, which should not be 
taken into consideration.
Viewing all the gradient results (right side), 
options \emph{C, I, J, K} and \emph{L} show 
the least number of best RMSE values.
Also, the same options were not found 
to be the best choice for the states. 
The option \emph{A} is also not to be found 
as the best choice for states nor 
does it have a good result for the 
gradients for V18 and V28.  
Option \emph{A} is clearly no suggestion, but 
does not perform as bad as the other 
options mentioned before. The remaining 
possible options are: 
\emph{B, D, E, F, G, H, L, M, N, O} and 
\emph{P}. From these options 
\emph{F, H, M} and \emph{N} exhibit the 
best state results. These are 2 Kriging and 2 TPS models.
3 out of these 4 models include regularization and 
have the augmentation value of $1$. From here on, the most 
appealing option by including the gradients, 
is option \emph{N}.  
Furthermore, 
figures \ref{fig_105} 
and \ref{fig_109} should be considered 
in this examination. They  
depict the deviation for the 
different versions (V18, V28, V38) 
for \emph{SMARTy}'s TPS with 
an augmentation of $-1$. For both 
missions the deviation is high. This  
option 
is declared as \emph{J} in 
the figures \ref{fig_113} and 
\ref{fig_118} and also here, 
unsatisfactory results are found. \newline

In conclusion, it can be said, no real winner 
could be found out for the options. However, 
using regularization and an augmentation 
of value $1$ can be recommended. If a final 
decision had to be made, option \emph{N},
the TPS model with regularization
and an augmentation of $1$ should be chosen.
Note that performing this model selection study 
took more than 3 days on the workstation described 
in section \ref{sec_Motivation}.
This means performing
model selection for each new set of training samples 
within an aerodynamic optimization 
is not feasible. Therefore, TPS with 
active regularization and an augmentation 
of $1$ is set as the default model 
in \emph{missioninformer}.

\begin{figure}[!h]
    %\vspace{0.5cm}
    \begin{minipage}[h]{0.46\textwidth}
        \centering
        \includegraphics[width =\textwidth]{2_Figures/3_Task/3_Surrogate/6_Best_Models/state_Model_18.pdf}
        \caption{Lowest RMSE by having compared Krigings and TPS 8 options, state V18}
        \label{fig_113}    
    \end{minipage}
    \hfill
    \begin{minipage}{0.46\textwidth}
        \centering
        \includegraphics[width =\textwidth]{2_Figures/3_Task/3_Surrogate/6_Best_Models/grad_Model_18.pdf}
        \caption{Lowest RMSE by having compared Krigings and TPS 8 options, gradient V18}
        \label{fig_114}    
    \end{minipage}
    \vspace{0.5cm}
    % ------------------------------------ V28 ----------------
    \begin{minipage}[h]{0.46\textwidth}
        \centering
        \includegraphics[width =\textwidth]{2_Figures/3_Task/3_Surrogate/6_Best_Models/state_Model_28.pdf}
        \caption{Lowest RMSE by having compared Krigings and TPS 8 options, state V28}
        \label{fig_115}    
    \end{minipage}
    \hfill
    \begin{minipage}{0.46\textwidth}
        \centering
        \includegraphics[width =\textwidth]{2_Figures/3_Task/3_Surrogate/6_Best_Models/grad_Model_28.pdf}
        \caption{Lowest RMSE by having compared Krigings and TPS 8 options, gradient V28}
        \label{fig_116}    
    \end{minipage}
    \vspace{0.5cm}
    % ------------------------------------ V38 ----------------
    \begin{minipage}[h]{0.46\textwidth}
        \centering
        \includegraphics[width =\textwidth]{2_Figures/3_Task/3_Surrogate/6_Best_Models/state_Model_38.pdf}
        \caption{Lowest RMSE by having compared Krigings and TPS 8 options, state V38}
        \label{fig_117}    
    \end{minipage}
    \hfill
    \begin{minipage}{0.46\textwidth}
        \centering
        \includegraphics[width =\textwidth]{2_Figures/3_Task/3_Surrogate/6_Best_Models/grad_Model_38.pdf}
        \caption{Lowest RMSE by having compared Krigings and TPS 8 options, gradient V38}
        \label{fig_118}    
    \end{minipage}
\end{figure} 
