\section{Surrogate models}
In this section different 
surrogate models, which 
are listed in section \ref{sec_Surrogates},
will be explored with regard to 
their respective options.
For generating the RBF surrogate model \emph{SciPy} was used 
with its default kernel multiquadtratic, which is given
in section \ref{subsec_Kriging_Surrogate} 
in equation \eqref{eq_34}. Therefore, a \emph{smooth}
value \cite{noauthor_rbf_2021} of $0.1$ was employed, which 
means regression is performed. Note for 
further reading, when the word RBF is
mentioned, it always means \emph{Scipy'}s RBF.
Kriging surrogate models 
were generated by invoking \emph{SMARTy}. Here the chosen default Kriging
kernel was 
Gaussian, Augmentation and regularization were set 
to $1$ and \emph{True}, respectively. 
The training samples 
for generating the surrogate models 
were provided  by the \emph{DLR}'s analysis and 
optimization workflow \emph{FSAerOpt} 
\cite{merle_high-fidelity_2019}. It is 
able to calculate fully trimmed states 
and consistent adjoint based gradients involving the 
flow solver \emph{TAU}. A total of $38$
training samples distributed in the 
$Ma,h,m$ space using the Halton Design Of 
Experiments (DOE) method, which 
are re depicted 
in the figures \ref{fig_56_Mass_Mach}
to \ref{fig_57_Mach_H},  were provided 
for this and the upcoming surrogate 
models studies.
\newline 

\begin{figure}[!h]
    %\vspace{0.5cm}
    \begin{minipage}[h]{0.46\textwidth}
        \centering
        \includegraphics[width =\textwidth]{2_Figures/3_Task/2_Interpol_Model/00.Sampling.MassHalfModel[kg].MachNumber[-].png}
        \caption{Halton DOE for half mass and Mach number ($m/2, \, Ma$)}
        \label{fig_56_Mass_Mach}    
    \end{minipage}
    \hfill
    \begin{minipage}{0.46\textwidth}
        \centering
        \includegraphics[width =\textwidth]{2_Figures/3_Task/2_Interpol_Model/01.Sampling.MassHalfModel[kg].Altitude[m].png}
        \caption{Halton DOE for half mass and altitude ($m/2, \, h$)}
        \label{fig_57_Mass_H}    
    \end{minipage}
\end{figure} 

\begin{figure}
    \centering
    \includegraphics[width =0.46\textwidth]{2_Figures/3_Task/2_Interpol_Model/02.Sampling.MachNumber[-].Altitude[m].png}
    \caption{Halton DOE for Mach number and altitude ($Ma, \, h$)}
    \label{fig_57_Mach_H}    
\end{figure}

\FloatBarrier
The input for the surrogate model is 3 dimensional, 
meaning Mach numbers, altitudes and masses 
$Ma, h, m$ are provided. With this 
set of input variables the desired predicted 
outcome for the output variables $LoD, AoA, TSFC$
are generated. Note that the output
variables are called $LoD, AoA, TSFC$, but 
their respective interpolated 
values are denoted as $\tilde{LoD}, \tilde{AoA}, \tilde{TSFC}$.
This means 3 different surrogate 
models 
are trained. Because of the 3 dimensional input 
and the 1 dimensional output, plotting the 
surrogate models would require 4 axes. Therefore, for 
showing  results, plots at constant
Mach numbers, altitudes and masses ($Ma, h, m$)
are going to be presented. Also note that the mass, when 
solving the ODE is equivalent to the cruise 
starting mass $m_s$. \newline

Since more research has been done than can 
be showed here explicitly with figures, only 
some results will be shown. This issue is understood 
better by highlighting the fact that the 
explorations are performed for state as  
well as the gradients. In case of the 
state surrogate models only 3 surrogate models are 
constructed. In case of the gradients,
$126$ shape parameter were given, thus 
$126*3 = 378$ surrogate models could be presented.
Additionally, to recall, only sections where 
of $Ma, h, m$ is constant can be visualized 
properly. Therefore, some 
results, which 
are supposed to summarize the findings, are chosen
to be displayed. The structure of presenting the results 
is as follows. On the left and right side the RBF and 
Kriging surrogate interpolation models are depicted 
for the same constant parameter of of $Ma, h, m$, respectively.
Therefore, interpolation models for $LoD, AoA, TSFC$
will be 
given at two different respective constant 
parameter values for $Ma, h, m$. In other words, 
for each input parameter $Ma, h, m$, 
one parameter is chosen to be constant 
at a specific value. This value is changed 
two times and isoplanes through 
the entire space of the 
interpolation models (RBF, Kriging) are sliced and shown 
next to each other. In total, 6 different constant 
values for each input parameter $Ma, h, m$ were
chosen. They are uniformly distributed between 
the lowest and highest values of the respective 
input parameter ($Ma, h, m$).\newline

% ------------------------------- 01 ----------------------------------------
For interpreting the figures, be reminded that the angle 
of attack is also often referred to as $AOA = \alpha$
and note that the y-axis is equally color-coded.
Figures \ref{fig_56} to \ref{fig_59} depict 
the interpolation models for the constant
altitude $10058.4$ m  and $11887.2$ m, for 
the output variables $LoD, AoA, TSFC$.
In figures \ref{fig_56} and \ref{fig_57} 
noticeable differences in the interpolation 
models for $AoA$ can be observed. The Kriging 
model on the right side of depicts a 
smoother transition than RBF. This observation 
can bee seen in all figures 
with a constant altitude from \ref{fig_56}
to \ref{fig_67}. This holds also, when varying 
the altitude or even the output variable.
In other words, each interpolation 
model for the different desired 
output variables $LoD, AoA, TSFC$ 
confirms this observation. A possible 
reason for this occurrence could be that the
Kriging value for 
regularization is automatically tuned in \emph{SMARTy} and can 
be higher 
than the value for \emph{SciPy}'s RBF.
The answer to the 
question, which models predict the underlying 
physical phenomenons accurately requires a deep understanding 
of the relationship of all the input and output 
parameters and can be partly answered in 
subsection \ref{subsec_Krig_TPS_Accu}\newline

% % ====================================================================
% % ================== Const H =============================================
% % =======================================================================
% % ------------------- AoA---------------------
\begin{figure}[!h]
    %\vspace{0.5cm}
    \begin{minipage}[h]{0.46\textwidth}
        \centering
        \includegraphics[width =\textwidth]{2_Figures/3_Task/2_Interpol_Model/Regular/RBF/Const_H/Mach_Mass_AoA_1.png}
        \caption{Surrogate interpolation model for $AoA$ at constant altitude $h = 10058.4$ [m], RBF}
        \label{fig_56}    
    \end{minipage}
    \hfill
    \begin{minipage}{0.46\textwidth}
        \centering
        \includegraphics[width =\textwidth]{2_Figures/3_Task/2_Interpol_Model/Regular/Kriging/Const_H/Mach_Mass_AoA_1.png}
        \caption{Surrogate interpolation model for $AoA$ at constant altitude $h = 10058.4$ [m], Kriging}
        \label{fig_57}    
    \end{minipage}
\end{figure} 

\begin{figure}[!h]
    %\vspace{0.5cm}
    \begin{minipage}[h]{0.46\textwidth}
        \centering
        \includegraphics[width =\textwidth]{2_Figures/3_Task/2_Interpol_Model/Regular/RBF/Const_H/Mach_Mass_AoA_6.png}
        \caption{Surrogate interpolation model for $AoA$ at constant altitude $h = 11887.2$ [m], RBF}
        \label{fig_58}    
    \end{minipage}
    \hfill
    \begin{minipage}{0.46\textwidth}
        \centering
        \includegraphics[width =\textwidth]{2_Figures/3_Task/2_Interpol_Model/Regular/Kriging/Const_H/Mach_Mass_AoA_6.png}
        \caption{Surrogate interpolation model for $AoA$ at constant altitude $h = 11887.2$ [m], Kriging}
        \label{fig_59}    
    \end{minipage}
\end{figure} 

% ------------------- TSFC---------------------
\begin{figure}[!h]
    %\vspace{0.5cm}
    \begin{minipage}[h]{0.46\textwidth}
        \centering
        \includegraphics[width =\textwidth]{2_Figures/3_Task/2_Interpol_Model/Regular/RBF/Const_H/Mach_Mass_TSFC_1.png}
        \caption{Surrogate interpolation model for $TSFC$ at constant altitude $h = 10058.4$ [m], RBF}
        \label{fig_60}    
    \end{minipage}
    \hfill
    \begin{minipage}{0.46\textwidth}
        \centering
        \includegraphics[width =\textwidth]{2_Figures/3_Task/2_Interpol_Model/Regular/Kriging/Const_H/Mach_Mass_TSFC_1.png}
        \caption{Surrogate interpolation model for $TSFC$ at constant altitude $h = 10058.4$ [m], Kriging}
        \label{fig_61}    
    \end{minipage}
\end{figure} 

\begin{figure}[!h]
    %\vspace{0.5cm}
    \begin{minipage}[h]{0.46\textwidth}
        \centering
        \includegraphics[width =\textwidth]{2_Figures/3_Task/2_Interpol_Model/Regular/RBF/Const_H/Mach_Mass_TSFC_6.png}
        \caption{Surrogate interpolation model for $TSFC$ at constant altitude $h = 11887.2$ [m], RBF}
        \label{fig_62}    
    \end{minipage}
    \hfill
    \begin{minipage}{0.46\textwidth}
        \centering
        \includegraphics[width =\textwidth]{2_Figures/3_Task/2_Interpol_Model/Regular/Kriging/Const_H/Mach_Mass_TSFC_6.png}
        \caption{Surrogate interpolation model for $TSFC$ at constant altitude $h = 11887.2$ [m], Kriging}
        \label{fig_63}    
    \end{minipage}
\end{figure} 

% ------------------- LoD---------------------

\begin{figure}[!h]
    %\vspace{0.5cm}
    \begin{minipage}[h]{0.46\textwidth}
        \centering
        \includegraphics[width =\textwidth]{2_Figures/3_Task/2_Interpol_Model/Regular/RBF/Const_H/Mach_Mass_LoD_1.png}
        \caption{Surrogate interpolation model for $LoD$ at constant altitude $h = 10058.4$ [m], RBF}
        \label{fig_64}    
    \end{minipage}
    \hfill
    \begin{minipage}{0.46\textwidth}
        \centering
        \includegraphics[width =\textwidth]{2_Figures/3_Task/2_Interpol_Model/Regular/Kriging/Const_H/Mach_Mass_LoD_1.png}
        \caption{Surrogate interpolation model for $LoD$ at constant altitude $h = 10058.4$ [m], Kriging}
        \label{fig_65}    
    \end{minipage}
\end{figure} 

\begin{figure}[!h]
    %\vspace{0.5cm}
    \begin{minipage}[h]{0.46\textwidth}
        \centering
        \includegraphics[width =\textwidth]{2_Figures/3_Task/2_Interpol_Model/Regular/RBF/Const_H/Mach_Mass_LoD_6.png}
        \caption{Surrogate interpolation model for $LoD$ at constant altitude $h = 11887.2$ [m], RBF}
        \label{fig_66}    
    \end{minipage}
    \hfill
    \begin{minipage}{0.46\textwidth}
        \centering
        \includegraphics[width =\textwidth]{2_Figures/3_Task/2_Interpol_Model/Regular/Kriging/Const_H/Mach_Mass_LoD_6.png}
        \caption{Surrogate interpolation model for $LoD$ at constant altitude $h = 11887.24$ [m], Kriging}
        \label{fig_67}    
    \end{minipage}
\end{figure} 




% % ====================================================================
% % ================== Ccnst Mach =============================================
% % =======================================================================

\FloatBarrier
The figures from \ref{fig_68} to \ref{fig_75} show the interpolations 
models, when a cut at constant Mach numbers is performed. 
Here the same effect as described above for 
the constant altitude case can be observed. Kriging 
is smoother in the curve of its predicted values 
than RBF is. However, in the case of constant Mach numbers,
one more phenomena can be observed. The actual 
values of for the desired output variables 
$LoD, AoA, TSFC$ are different when 
comparing RBF (left side) with Kriging (right side).
This effect is clearly visible, e.g. in figures
\ref{fig_72} and \ref{fig_73}.\newline

% % ------------------- AoA---------------------
\begin{figure}[!h]
    %\vspace{0.5cm}
    \begin{minipage}[h]{0.46\textwidth}
        \centering
        \includegraphics[width =\textwidth]{2_Figures/3_Task/2_Interpol_Model/Regular/RBF/Const_Mach/H_Mass_AoA_1.png}
        \caption{Surrogate interpolation model for $AoA$ at constant $Ma = 0.815$ [-], RBF}
        \label{fig_68}    
    \end{minipage}
    \hfill
    \begin{minipage}{0.46\textwidth}
        \centering
        \includegraphics[width =\textwidth]{2_Figures/3_Task/2_Interpol_Model/Regular/Kriging/Const_Mach/H_Mass_AoA_1.png}
        \caption{Surrogate interpolation model for $AoA$ at constant $Ma = 0.815$ [-], Kriging}
        \label{fig_69}    
    \end{minipage}
\end{figure} 

\begin{figure}[!h]
    %\vspace{0.5cm}
    \begin{minipage}[h]{0.46\textwidth}
        \centering
        \includegraphics[width =\textwidth]{2_Figures/3_Task/2_Interpol_Model/Regular/RBF/Const_Mach/H_Mass_AoA_6.png}
        \caption{Surrogate interpolation model for $AoA$ at constant $Ma = 0.845$ [-], RBF}
        \label{fig_70}    
    \end{minipage}
    \hfill
    \begin{minipage}{0.46\textwidth}
        \centering
        \includegraphics[width =\textwidth]{2_Figures/3_Task/2_Interpol_Model/Regular/Kriging/Const_Mach/H_Mass_AoA_6.png}
        \caption{Surrogate interpolation model for $AoA$ at constant $Ma = 0.845$ [-], Kriging}
        \label{fig_71}    
    \end{minipage}
\end{figure} 


% % ------------------- LoD---------------------

\begin{figure}[!h]
    %\vspace{0.5cm}
    \begin{minipage}[h]{0.46\textwidth}
        \centering
        \includegraphics[width =\textwidth]{2_Figures/3_Task/2_Interpol_Model/Regular/RBF/Const_Mach/H_Mass_LoD_1.png}
        \caption{Surrogate interpolation model for $LoD$ at constant $Ma = 0.815$ [-], RBF}
        \label{fig_72}    
    \end{minipage}
    \hfill
    \begin{minipage}{0.46\textwidth}
        \centering
        \includegraphics[width =\textwidth]{2_Figures/3_Task/2_Interpol_Model/Regular/Kriging/Const_Mach/H_Mass_LoD_1.png}
        \caption{Surrogate interpolation model for $LoD$ at constant $Ma = 0.815$ [-], Kriging}
        \label{fig_73}    
    \end{minipage}
\end{figure} 

\begin{figure}[!h]
    %\vspace{0.5cm}
    \begin{minipage}[h]{0.46\textwidth}
        \centering
        \includegraphics[width =\textwidth]{2_Figures/3_Task/2_Interpol_Model/Regular/RBF/Const_Mach/H_Mass_LoD_6.png}
        \caption{Surrogate interpolation model for $LoD$ at constant $Ma = 0.845$ [-], RBF}
        \label{fig_74}    
    \end{minipage}
    \hfill
    \begin{minipage}{0.46\textwidth}
        \centering
        \includegraphics[width =\textwidth]{2_Figures/3_Task/2_Interpol_Model/Regular/Kriging/Const_Mach/H_Mass_LoD_6.png}
        \caption{Surrogate interpolation model for $LoD$ at constant $Ma = 11887.24$ [m], Kriging}
        \label{fig_75}    
    \end{minipage}
\end{figure} 


% % ====================================================================
% % ================== Const Mass =============================================
% % =======================================================================
\FloatBarrier
The figures from \ref{fig_76} to \ref{fig_87} show the interpolations 
models, when a cut at constant mass is made. 
These are slices, where the RBF plots (left side)
for the output variables $LoD,AoA, TSFC$
are also smooth. 
However, Kriging (right side) remains 
smooth, which suggests, that RBF and Kriging trends
must not necessarily be contradicting.
The second difference 
with regard to the constant Mach number and altitude slices is, 
that comparing
RBF (left side) and Kriging (right side) small 
noticeable differences in the values for the output variables 
$LoD, AoA, TSFC$ in the overall space can be observed.
 This can be seen, e.g. by 
comparing figure \ref{fig_78} and \ref{fig_79}. 
Consider the upper left corners and the whole 
upper edges. Here the clear contrast 
in the colors and thus in the values 
for $AoA$ can bee seen. These  differences, can 
also become extreme when comparing figures \ref{fig_86}
and \ref{fig_87}. 
Employing RBF the values for $LoD$ are all 
greater or equal to $LoD \geq 21.046$. In contrast, 
the Kriging model (right side) also exhibits 
$LoD$ values which are around $19$. This highlights, 
the interpolation for $LoD$ highly 
depends on the underlying surrogate interpolation model.\newline

% % ------------------- AoA---------------------
\begin{figure}[!h]
    %\vspace{0.5cm}
    \begin{minipage}[h]{0.46\textwidth}
        \centering
        \includegraphics[width =\textwidth]{2_Figures/3_Task/2_Interpol_Model/Regular/RBF/Const_Mass/Mach_H_AoA_1.png}
        \caption{Surrogate interpolation model for $AoA$ at constant mass $m = 171000$ [kg], RBF}
        \label{fig_76}    
    \end{minipage}
    \hfill
    \begin{minipage}{0.46\textwidth}
        \centering
        \includegraphics[width =\textwidth]{2_Figures/3_Task/2_Interpol_Model/Regular/Kriging/Const_Mass/Mach_H_AoA_1.png}
        \caption{Surrogate interpolation model for $AoA$ at constant mass $m = 171000$ [kg], Kriging}
        \label{fig_77}    
    \end{minipage}
\end{figure} 

\begin{figure}[!h]
    %\vspace{0.5cm}
    \begin{minipage}[h]{0.46\textwidth}
        \centering
        \includegraphics[width =\textwidth]{2_Figures/3_Task/2_Interpol_Model/Regular/RBF/Const_Mass/Mach_H_AoA_6.png}
        \caption{Surrogate interpolation model for $AoA$ at constant mass $m = 245000$ [kg], RBF}
        \label{fig_78}    
    \end{minipage}
    \hfill
    \begin{minipage}{0.46\textwidth}
        \centering
        \includegraphics[width =\textwidth]{2_Figures/3_Task/2_Interpol_Model/Regular/Kriging/Const_Mass/Mach_H_AoA_6.png}
        \caption{Surrogate interpolation model for $AoA$ at constant mass $m = 245000$ [kg], Kriging}
        \label{fig_79}    
    \end{minipage}
\end{figure} 

% % ------------------- TSFC---------------------
\begin{figure}[!h]
    %\vspace{0.5cm}
    \begin{minipage}[h]{0.46\textwidth}
        \centering
        \includegraphics[width =\textwidth]{2_Figures/3_Task/2_Interpol_Model/Regular/RBF/Const_Mass/Mach_H_TSFC_1.png}
        \caption{Surrogate interpolation model for $TSFC$ at constant mass $m = 171000$ [kg], RBF}
        \label{fig_80}    
    \end{minipage}
    \hfill
    \begin{minipage}{0.46\textwidth}
        \centering
        \includegraphics[width =\textwidth]{2_Figures/3_Task/2_Interpol_Model/Regular/Kriging/Const_Mass/Mach_H_TSFC_1.png}
        \caption{Surrogate interpolation model for $TSFC$ at constant mass $m = 171000$ [kg], Kriging}
        \label{fig_81}    
    \end{minipage}
\end{figure} 

\begin{figure}[!h]
    %\vspace{0.5cm}
    \begin{minipage}[h]{0.46\textwidth}
        \centering
        \includegraphics[width =\textwidth]{2_Figures/3_Task/2_Interpol_Model/Regular/RBF/Const_Mass/Mach_H_TSFC_6.png}
        \caption{Surrogate interpolation model for $TSFC$ at constant mass $m = 245000$ [kg], RBF}
        \label{fig_82}    
    \end{minipage}
    \hfill
    \begin{minipage}{0.46\textwidth}
        \centering
        \includegraphics[width =\textwidth]{2_Figures/3_Task/2_Interpol_Model/Regular/Kriging/Const_Mass/Mach_H_TSFC_6.png}
        \caption{Surrogate interpolation model for $TSFC$ at constant mass $m = 245000$ [kg], Kriging}
        \label{fig_83}    
    \end{minipage}
\end{figure} 

% ------------------- LoD---------------------

\begin{figure}[!h]
    %\vspace{0.5cm}
    \begin{minipage}[h]{0.46\textwidth}
        \centering
        \includegraphics[width =\textwidth]{2_Figures/3_Task/2_Interpol_Model/Regular/RBF/Const_Mass/Mach_H_LoD_1.png}
        \caption{Surrogate interpolation model for $LoD$ at constant mass $m = 171000$ [kg], RBF}
        \label{fig_84}    
    \end{minipage}
    \hfill
    \begin{minipage}{0.46\textwidth}
        \centering
        \includegraphics[width =\textwidth]{2_Figures/3_Task/2_Interpol_Model/Regular/Kriging/Const_Mass/Mach_H_LoD_1.png}
        \caption{Surrogate interpolation model for $LoD$ at constant mass $m = 171000$ [kg], Kriging}
        \label{fig_85}    
    \end{minipage}
\end{figure} 

\begin{figure}[!h]
    %\vspace{0.5cm}
    \begin{minipage}[h]{0.46\textwidth}
        \centering
        \includegraphics[width =\textwidth]{2_Figures/3_Task/2_Interpol_Model/Regular/RBF/Const_Mass/Mach_H_LoD_6.png}
        \caption{Surrogate interpolation model for $LoD$ at constant mass $m = 245000$ [kg], RBF}
        \label{fig_86}    
    \end{minipage}
    \hfill
    \begin{minipage}{0.46\textwidth}
        \centering
        \includegraphics[width =\textwidth]{2_Figures/3_Task/2_Interpol_Model/Regular/Kriging/Const_Mass/Mach_H_LoD_6.png}
        \caption{Surrogate interpolation model for $LoD$ at constant mass $Ma = 11887.24$ [m], Kriging}
        \label{fig_87}    
    \end{minipage}
\end{figure} 


\FloatBarrier
This section can be concluded with the following.
Clearly, there are differences in the outcome 
of the predicted values and trends for the desired 
output variables $LoD, TSFC, AOA$. As a
consequence, once more it can be said that the 
choice of the surrogate model heavily determines 
the results of any further calculation.