
\chapter{Results}
\label{chap_Results}

In this chapter the results of the methods described 
in previous chapter shall be presented. The first 
topic for this purpose is the solution of 
the state ODE from equation \eqref{eq_10}.


\section{Solving the ODE}
\label{sec_Solve_ODE_Methods}

The main equation upon which's solution 
the remaining computations are based on 
is the ODE for the cruise 
fuel burnt mass $m_{cr}$.

\begin{equation}
    \frac{dm_{cr}}{ds} = \frac{g}{a} \, \frac{1}{Ma}\,TSFC \,(m_s + m_{cr})
    \, \frac{1}{
    LoD \, cos(AoA) + sin(AoA)}   \tag{\ref{eq_10}}
\end{equation}

The used technique therefore is by calling 
SciPy's function \emph{solve\_ivp} \cite{noauthor_scipys_2021}.
The library comes along with different solvers and the solution 
of all were investigated. However 
only a selection of the most interesting 
results  
shall be presented here. Having compared all 
available methods inside \emph{solve\_ivp} 
it could be found out that \emph{LSODA} and 
\emph{RK45} offered the least difference 
to the analytical solution obtained by equation 
\eqref{eq_16}. Its 
derivation and explanations can be 
read in sections \ref{sec_Solve_ODE}.
\begin{equation}
    m_{fe}(s = R_{cr}) =m_s \left(1-e^{\frac{-g}{a} \, 
    \frac{1}{Ma} \, TSFC \, 
    \frac{1}{LoD \,cos(AoA) + sin(AoA)} \, s} \right)
    \tag{\ref{eq_16}}
\end{equation}

The analytical solution is only valid, when
assuming that $LoD, AoA, TSFC$
are constant. Comparing the analytical solution 
with \emph{SciPy}'s numerical solution by employing 
all available  methods, it could be observed, that 
$LSODA$ exhibits best results. The figures from 
\ref{fig_9_LSODA} to \ref{fig_12_RK45_Standard} depicts 
the analytical and the numerical solution on the left 
side and their deviation on the right side. One of the 
options within \emph{LSODA} and \emph{RK45}
is to set points where an integration step shall be 
performed. The higher the number of the given 
calculation points is, the higher the accuracy of 
the outcome is. Figures \ref{fig_9_LSODA} and 
\ref{fig_10_RK45} depict the results when 
the user exerted calculation points. From range 
$R = [0;100]$ meter and for the last 100 meter of the 
cruise flight range, 500 uniformly 
calculation points were set. In between, 
$\frac{R_{max}}{8} = \frac{7297989}{8} = 912248 $
calculation points were set. Even though 
a high number of evaluation points 
leads to a more accurate result, recall that for 
each additional evaluation point, the 
surrogate models needs to be invoked. Therefore, 
it is desired to keep the number 
of the control points low, if possible.\newline

%%
\begin{figure}[!h]
    \centering
    \includegraphics[width =0.85\textwidth]{2_Figures/3_Task/1_LSODA.pdf}
    \caption{Solution obtained having used \emph{LSODA} with user defined 
    calculation points}
    \label{fig_9_LSODA}
\end{figure}
%
%
\begin{figure}[!h]
    \centering
    \includegraphics[width =0.85\textwidth]{2_Figures/3_Task/1_RK45.pdf}
    \caption{Solution obtained having used \emph{RK45} with user defined 
    calculation points}
    \label{fig_10_RK45}
\end{figure}


In figures \ref{fig_11_LSODA_Standard} and \ref{fig_12_RK45_Standard},
the same calculations are performed with \emph{LSODA} and \emph{RK45}, 
respectively, but with their methods default number 
of calculation points. The default number of calculation 
points is much lower, e.g. for \emph{RK45} less than 20 sample 
points were required. Comparing all the 4 presented
results it can be stated that the deviation or \emph{delta} 
between the analytical and numerical solution is 
in the order of grams, which is
sufficient for 
the accuracy requirement of the \emph{missioninformer}.
Even though \emph{LSODA} delivers the best performance, 
since \emph{RK45} requires less integrations steps 
points with barley less accuracy, \emph{RK45} with 
its default method for setting calculation 
points is set as default in the \emph{missioninformer}.


\begin{figure}[!h]
    \centering
    \includegraphics[width =0.85\textwidth]{2_Figures/3_Task/2_StandardLSODA.pdf}
    \caption{Solution obtained having used \emph{LSODA} without user defined 
    calculation points}
    \label{fig_11_LSODA_Standard}
\end{figure}


\begin{figure}[!h]
    \centering
    \includegraphics[width =0.85\textwidth]{2_Figures/3_Task/2_StandardRK45.pdf}
    \caption{Solution obtained having used \emph{RK45} without user defined 
    calculation points}
    \label{fig_12_RK45_Standard}
\end{figure}



\FloatBarrier
The results above were presented by assuming 
$LoD, AoA, TSFC$ to be constant. However, 
for \emph{missioninformer} they are not constant, but 
dependent on $Ma, h, m$. In order 
to see the effect of these variables not to 
be constant, all available methods 
within \emph{SciPy}s ODE solver were tested again.
For testing purposes no real aerodynamic 
data output was used, but rather some 
reasonable values, which were only dependent 
on $m$, were generated. This data 
set was the input for a \emph{1d} interpolation, which 
is also available in \emph{SciPy}. Coming to 
the workflow, every time the ODE is solved, the 
values for the variables $LoD, AoA, TSFC$ are 
obtained by the mentioned 1d interpolation.
This process can also be done with a lower 
or higher number of calculation points.
However, no deviation between the lower 
and higher number of calculation points 
version, could 
be observed.\newline 

The figures \ref{fig_13_LSODA_Inter} and 
\ref{fig_14_RK45_Inter} show the results, when the mentioned
1d interpolation and no interpolation is applied.
On the left side, the default mechanism 
for choosing calculation points is plotted.
On the right side, the above explained 
user defined calculation points were 
used for deriving the solution. Both 
sides show a deviation between 
the interpolation active and inactive versions, which 
is called \emph{delta} in the mentioned figures.
Figure \eqref{fig_13_LSODA_Inter} shows 
a strange and unexpected behavior in 
its delta. This is explained as follows.
The number of calculation 
points for the active and inactive 
with the defaults' 
method to define calculation points is 
not equal. 
In the inactive interpolation version, the 
default option for defining 
calculation points 
requires less calculation points.
However, for calculating delta, 
both number of calculation points 
mus be equal, to be valid.
Therefore,
the delta on the left side can 
not be considered to be correct. \newline 


This observation tells us, that \emph{LSODA} is more 
flexible in increasing the number of 
calculation points than RK45. This 
can bee seen by viewing the delta on the 
left side of the figure \ref{fig_14_RK45_Inter}.
Here, a monotonically increasing delta  can be identified. 
The default version of
\emph{RK45} took less than 20 calculation points, 
whereas \emph{LSODA} required around 60 
calculation points.


\begin{figure}[!h]
    \centering
    \includegraphics[width =0.9\textwidth]{2_Figures/3_Task/1_Effect_LSODA_High_Number_Ranges.pdf}
    \caption{Solution obtained having used \emph{LSODA} with and without user defined 
    calculation points and active and inactive interpolation}
    \label{fig_13_LSODA_Inter}
\end{figure}


\begin{figure}[!h]
    \centering
    \includegraphics[width =0.9\textwidth]{2_Figures/3_Task/1_Effect_RK45_High_Number_Ranges.pdf}
    \caption{Solution obtained having used \emph{RK45} with and without user defined 
    calculation points and active and inactive interpolation}
    \label{fig_14_RK45_Inter}
\end{figure}


\FloatBarrier
Comparing \emph{LSODA} with \emph{RK45} directly can be done 
via figure \ref{fig_15_RK45_LSODA}. Important 
for present concern is the right side, where 
the blue and orange graphs are the objects of 
our focus. However, the blue graph is not visible.
The reason for that is, that is has been 
draw before the orange graph has been drawn.
Therefore, the blue graph is beneath the 
covering orange graph. With this it can be 
said, no noticeable deviation in the 
results for active interpolation by having 
defined a ridiculously high number 
of calculation points can be seen. This leads to 
the following conclusion. No real difference 
in the accuracy of \emph{LSODA} and \emph{RK45} is 
exhibited, however, \emph{LSODA} requires more 
calculation points. As a consequence 
of this, \emph{RK45} can be chosen as the 
default method to solve the ODE 
in the \emph{missioninformer}.


\begin{figure}[!h]
    \centering
    \includegraphics[width =\textwidth]{2_Figures/3_Task/6_Compare_User_Default_RK45_LSODA_High_Number_Ranges.pdf}
    \caption{Direct compare \emph{RK45} with \emph{LSODA} for in/active and default/user versions}
    \label{fig_15_RK45_LSODA}
\end{figure}

