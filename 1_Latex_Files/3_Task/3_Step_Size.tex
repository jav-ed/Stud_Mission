\newpage
\FloatBarrier
\section{Conducting a step-size-study}
\label{sec_Step_Size_Study}
In section \ref{subsec_CDS} the reasons for 
the necessity of a step-size investigation 
has been described. In short it 
can be said, a wrong chosen step-size 
leads to not useable gradient values 
for the optimization. In this section the 
results of the performed step-size 
study shall be shown and elaborated.        
In order to calculate the gradients 
using different step-sizes, the 
procedure given in 
table \ref{tab_6} was employed.
The left side shows the interval 
from which the step-sizes were taken,  
on the right side, the number of evenly distributed 
points, which shall be used as the step size, 
are given. The investigation was conducted for 
2 different missions as explained in section 
\ref{subsec_CDS}. Both missions can be 
seen in table \ref{tab_5_Mission_Step_Size}.
Mission 1 encountered 
or failed the \emph{constraint violation check}, which 
is explained in section \ref{sec_Fuel_Mass_iter}.
In cases, where 
the \emph{constraint violation check} displayed warning 
messages, the respective mission required 
a high number of fix-point iteration. Mostly, the limit 
of maximal 500 allowed iterations was reached.
In contrast, mission 2 successfully passed 
the \emph{constraint violation check} and thus no 
warning messages were encountered. Also, 
the number of iterations were much lower.
\begin{table}[!h]
    \centering
        \begin{tabular}{l c}
            \multicolumn{1}{p{2cm}}{\textbf{Intervall}} &
            \multicolumn{1}{p{2cm}}{\textbf{\# points}}\\
            \hline
            $[1\mathrm{e}{-1} \; ; 1\mathrm{e}{-3} ]$ & $20$  \\
            $[1\mathrm{e}{-3} \; ; 1\mathrm{e}{-4} ]$ & $10$ \\
            $[1\mathrm{e}{-4} \; ; 1\mathrm{e}{-5} ]$ & $10$\\
            $[1\mathrm{e}{-5} \; ; 1\mathrm{e}{-6} ]$ & $15$\\
            $[1\mathrm{e}{-6} \; ; 1\mathrm{e}{-7} ]$ & $15$\\
            $[1\mathrm{e}{-7} \; ; 1\mathrm{e}{-8} ]$ & $10$\\
        \end{tabular}
        \caption{Chosen step-sizes }
        \label{tab_6}
\end{table}


Furthermore, two different values for the tolerance 
parameter where used for the above-mentioned investigations.
For the first examination the absolute 
and relative tolerance parameter \cite{noauthor_numpys_2021} for 
the direct and indirect gradient calculation 
workflows, which are explained in section \ref{subsec_CDS}, 
are set equal to $5\mathrm{e}{-5}$. For the 
second study, for the direct method, the 
tolerance parameter were set to $5\mathrm{e}{-6}$, 
which can be done since the direct method 
can reach finer tolerances as 
explained in section \ref{subsec_CDS}.
The CDS is computed in application for each
shape parameter. In this study, only five 
selected parameters were investigated.
 All the upcoming 
results are achieved by employing \emph{SciPy}'s RBF
with
multiquadric as kernel, which is given 
in equation \eqref{eq_34}, as the underlying 
surrogate model generator. \newline 

Note that not each 
analyzed result can be shown here, since 
the number of these would claim too many pages.
The outcomes, which describes the 
gained finding will be highlighted instead.
For this propose, the result will be
reviewed for the direct and indirect method, where 
the direct method has a tolerance parameter of $5\mathrm{e}{-6}$.
The results are only shown for mission 2
and for one shape parameter.\newline


The first interesting finding is that both 
methods, direct and indirect, 
retain a so-called plateau with 
respect to step size variations.
In the region of the plateau, 
the types of error, mentioned 
earlier, are in balance. In other words, 
within the plateau, no type of 
error dominates and the total value is small.
 In case of 
a too small step-size the round-off 
error becomes such high, that numerical 
instability is the consequence. In contrast,  
a too large step-size results in very 
high truncation error. In both cases, the 
result by using CDS cannot be considered 
to approximate the real gradient value.
This plateau  
was searched for, in order to be able to  
make statements and recommendations about the 
value for the step-size. This observation 
was seen in all tested cases for mission 2.
The figure \ref{fig_16} and \ref{fig_17} 
show the plateau for the direct and 
indirect method for mission 2, respectively.
It can be observed, that for both 
methods, direct and indirect, the  
plateau can bee seen around the 
step-size of $1\mathrm{e}{-5}$. This 
result was found also for 
remaining step-size studies  
the different shape parameters.\newline 

% =======================================================================
% ========================== SHAPE 0  =============================================
% =======================================================================
% ------------------- SHape -0 Miss 1----------------------
\begin{figure}[!h]
    \centering
    \includegraphics[width =0.9\textwidth]{2_Figures/3_Task/1_Steps/Shape_0/Dir_Miss_2_RBF.pdf}
    \caption{Step-Size-Investigation, Mission 2, Shape parameter $0$, Applied method: Direct, Unequal tolerance parameter}
    \label{fig_16}
\end{figure}


\begin{figure}[!h]
    \centering
    \includegraphics[width =0.9\textwidth]{2_Figures/3_Task/1_Steps/Shape_0/Indirect_Miss_2_RBF.pdf}
    \caption{Step-Size-Investigation, Mission 2, Shape parameter $0$, Applied method: Indirect, Unequal tolerance parameter}
    \label{fig_17}
\end{figure}

\FloatBarrier
However, a clear plateau could only 
be detected for mission 2. The deviation 
of the gradient with changing step-size
for mission 1 
was in the orders of $10^7 \,  kg/m$ and  $10^9\, kg/m$. 
This leads to the conclusion, when 
the \emph{constraint check violation} is 
passed successfully, the step-size 
of $1\mathrm{e}{-5}$ can be used for 
the CDS gradient calculation. Otherwise, when 
receiving warning statements by the 
\emph{constraint check violation}, 
not only can the step-size of $1\mathrm{e}{-5}$ 
not be used, rather CDS as the method 
for obtaining the gradients should not 
be used. This because, when no plateau can 
bee seen, there is no possibility to find 
a reasonable required step-size.\newline


Another conclusion which can be made is 
by comparing the direct and indirect 
results. The 
indirect method seems to be more 
stable within a broader step-size interval.
Also, this occurrence could be seen in 
for mission 2 for all tested shape parameters.
Note the scaling of the y-axis, which 
defined as 3 times the standard deviation. With this 
in mind the stability 
of the gradient by changing the step-size 
within the plateau becomes more tangible.\newline 

As summary, it can be stated that
the iterative indirect method was found 
to be more stable and choosing 
the step-size of $1\mathrm{e}{-5}$
could be proven to be optimal. This value
neither leads to numerical 
instabilities nor does it not comply 
with the accuracy requirements. Therefore,
the default method for solving 
the gradients with respect to the shape 
parameters is the indirect 
method with its default step-size of 
$1\mathrm{e}{-5}$.
