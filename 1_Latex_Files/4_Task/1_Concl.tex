\chapter{Discussion}
The \emph{missioninformer}, a tool, was 
proposed to  help to fulfill the  
environmental demands for reducing air traffic pollution. Clearly, the aircraft 
sector contributed and contributes 
to the current environmental crisis which we 
are facing. Since an aircraft has become 
indispensable, not only for the individual, 
but also for the global economy, the solution 
for a healthier environment cannot be 
banning aircraft, but rather improving it 
with regard to CO2, NOx and SO2 emissions. 
The \emph{missioninformer}, a lightweight tool was 
written from scratch. 
Its major task is 
to calculate the consumed fuel for
one or multiple individual 
missions for one aircraft. For the most 
important part of the mission, the cruise segment, 
physics-based equations are applied. In section \ref{sec_Solve_ODE}
they are introduced, the precision 
with which they are solved is satisfactory and 
is presented in section \ref{sec_Solve_ODE_Methods}.
They are 
fed with interpolated aerodynamic data, where 
the interpolation models were generated 
by using high fidelity RANS-based training samples.\newline


The next step \emph{missioninformer} 
went, was also to include the remaining 
flight segments by employing fuel fractions.
All the used fuel fractions 
were given in section \ref{sec_Cruise_Mission_Fuel}
in table \ref{tab_1}. 
At this stage, the \emph{missioninformer} 
can predict the fuel required for a whole 
flight mission. For real world applications, 
the user must be able to define the flown mission 
himself. The \emph{missioninformer} meets that need 
and in a structured way, so less effort is required, 
which is described in section \ref{sec_introduc_Miss} \newline


Since the \emph{missioninformer} is a leightweight code and 
because it can be used as a black box, it already
can be implemented easily into a global gradient
free optimization workflow. Since gradient-based 
optimization is known to be faster, especially
with a high number of design variables, 
\emph{missioninformer} was enhanced with the ability 
to calculate gradients. These are gradients of 
the fuel mass of the whole flight mission 
with respect to all arbitrarily given shape parameters. 
To guarantee a high accuracy of  \emph{missioninformer}'s 
state and gradient solutions many different studies 
were conducted.\newline

In order to solve the ODE for the cruise flight segment, two tested solvers exhibited 
particular high accuracies in their results. One of them (\emph{RK45}) also 
demanded few integration steps and thus excels 
in terms of execution time. Therefore, 
(\emph{RK45}) is well suited for research and industrial applications.
Based on this reasoning it is used as a standard ODE solver.\newline


For the gradient calculation, by employing central differencing, 
it can be stated that the iterative indirect method was found to be more
stable. Also, choosing the step-size of $1\mathrm{e}{-5}$ 
could be proven to be optimal. 
This value neither
leads to numerical instabilities nor does it 
not comply with the accuracy requirements.
Therefore, the default method for solving the gradients
with respect to the shape parameters
is the indirect method with its default step-size of $1\mathrm{e}{-5}$.\newline 

Having compared \emph{SciPy}'s RBF and \emph{SMARTy}'s Kriging (Gaussian)
and TPS
interpolation models for different model options, the following can be stated.
Clearly, there are differences in the
outcome of the predicted values and trends for the
desired output variables $LoD, TSFC , AOA$.
As a consequence, once more it can be said that 
the choice of the surrogate model heavily
determines the results of any further calculation. \newline

After having performed a computational intensive 
model selection investigation, the following 
conclusion is made.
No real winner 
could be found out for the options. However, 
using regularization and an augmentation 
of value $1$ can be recommended. If a final 
decision had to be made,
the TPS model with regularization
and an augmentation of $1$ should be chosen.
Note that performing this model selection study 
took more than 3 days on the workstation described 
in section \ref{sec_Motivation}.
This means performing
model selection for each new set of training samples 
within an aerodynamic optimization 
is not feasible. Therefore, TPS with 
active regularization and an augmentation 
of $1$ is set as the default model 
in \emph{missioninformer}.\newline

For the sake 
of applicational feasibility, execution time for the \emph{missioninformer}
with different surrogate models and their respective 
different intern parameters was measured. 
With that it can be stated, the \emph{missioninformer}
suits well for a fast and easy implementation 
into a gradient-based and gradient-free 
optimization workflow, where it completes 
its calculation within few minutes. 
Having introduced the \emph{missioninformer} broadly, 
it clearly can be considered as a possible tool 
at hand to be integrated into new digital design methods 
to reduce the 
negative impact of aviation on the environment, which 
can not be neglected nor overlooked.
 

% -------------------Conclusions and outlook--------------------------------------

\chapter{Conclusions and outlook}
A tool for calculating the total fuel mass and 
its gradient w.r.t. arbitrary shape parameter and for any 
user defined mission or multiple missions is presented.
It was generated from scratch and can be used 
as black box. 
It could be observed, that in case of two 
missions, the runtime was between $50$ to $800$ seconds 
(figure \ref{fig_112}) depending on the employed 
interpolation models and including the gradient computation.
 The following surrogate models 
were intensively investigated with different options:
RBF from the \emph{SciPy} library, Kriging (Gaussian) and TPS
from \emph{DLR}'s toolbox \emph{SMARTy}. RBF 
could clearly be filtered out from the list of the 
attractive surrogate model options, mainly 
due to its long execution times and 
highly options dependent results. In order to solve 
the ODE, which is required for 
computing the fuel fraction of the
cruise segment, different numerical solvers were used. 
By having 
applied simplifications an analytical solution of 
the ODE was introduced. The analytical solution 
was compared with different numerical solvers solutions. 
With these comparisons, 
statements about the accuracy of the numerical 
ODE solver could be made. The solution 
of the total fuel mass incorporates
mass snowball effects though the explained 
fix-point iteration. For the gradients, also 
an analytical approach was desired.
%
However, after comparing the analytically 
determined gradients with central finite 
differences results, the assumption was made for the derivation
were found to be not tenable.
%
A detailed analysis 
for finding an appropriate value for the step-size 
of the central differencing scheme was undertaken. Its 
results could be shown to be unambiguous. \newline 

As an outlook, the analytical solution for the gradients 
should be revisited. It failed due to the elaboration of the mass
term. It is assumed that 
there is a way to reformulate this term and thus obtain 
the gradients analytically. An alternative could be
employing automatic differentiation.
Up to now, only the most important flight segment, the cruise 
segment is described by equations which 
are based on physical observations. The currently 
used fuel fractions for the other  
flight segments could be replaced
by physics-based equations as well. Also in practice,
step-climbs are performed 
in order to fly with an optimal $LoD$. This means, 
the assumption of constant altitude,
which  is made in \emph{missioninformer} 
needs to be adapted. Next, state and gradient calculations are
not consistent due to the chosen approach with multiple
independent surrogate models. \emph{Missioninformer} should be 
tested in an aerodynamic gradient-based optimization 
to find out how many 
training points are necessary to achieve significant
reduction in the missions fuel consumption. Finally, 
the gradients of the missions fuel w.r.t. 
the structural mass needs to be calculated in order 
to use \emph{missioninformer} within 
a multi-disciplinary-optimization workflow.