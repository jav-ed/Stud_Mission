
% ===================================================
% ============Introduce Missioninformer =============
% ===================================================
\section{Introducing missioninformer}
\label{sec_introduc_Miss}
In this section the overall processes and the functionalities 
of the proposed tool \emph{missioninformer} shall be presented 
in an abstract way, without intern solved equations at this stage.
It is a code written entirely in \emph{Python} and makes 
use of the following libraries:
\begin{multicols}{2}
    \begin{enumerate}
        \item NumPy \item SciPy  \item matplotlib
        \item SMARTy (DLR's own Python library)
        \item re (regex)
        \item[\vspace{\fill}]
    \end{enumerate}
\end{multicols}

Its main workflow can be described with the figure \ref{fig_1_Missinfo}. It 
obtains two input files and outputs the mass of total fuel which is required 
for the given mission and the gradients of the mission fuel with 
respect to shape parameters. 
The mission input file is completely written in \emph{Python} and uses a regular dictionary, 
which is similar to a json file for storing purposes. 
In order to define one mission in the mission input file, the 
following 3 groups of parameters need to be set. The first 
group, named masses, contains weights:
\begin{multicols}{2}
    \begin{enumerate}
        \item maximal takeoff weight
        \item maximal landing weight  
        \item operating empty weight
        \item manufactures empty weights
        \item maximum zero fuel weight
        \item maximum fuel weight
        \item design payload
        \item maximum payload
    \end{enumerate}
\end{multicols}

\begin{figure}[!h]
    \def\svgwidth{\linewidth}
    \input{2_Figures/1_Task/1_Missioninformer.pdf_tex}
    \caption{Broad overview: Workflow \emph{missioninformer}}
    \label{fig_1_Missinfo}
\end{figure}

\FloatBarrier
The second group, named flying parameters, consists of the 
cruise Mach number (\emph{Ma}), 
cruise altitude (\emph{h}) and cruise range. The third group can be 
adjusted optionally by the user. It allows to change the step size 
for the central differencing scheme, which will be discussed 
in section \ref{sec_Step_Size_Study} and define 
a weighting factor for each mission. The step size steers the accuracy 
and reliability of the calculated gradients. In case, multiple missions 
shall be used, for each mission a weighting factor must be 
given. Using multiple missions or adding missions is done easily by  
copying the section of the parameters and
redefining these with desired values. Each new mission is 
summed to as one overall variable and needs to be passed 
as an argument to the mission collection 
method. \newline

The second input for the \emph{missioninformer} is a database. It is required 
to generate surrogate, i.e. interpolation models for the state variables 
$LoD, AoA \,(\alpha), TSFC$ and their gradients with respect to all 
shape parameters. $LoD$ is called glide ratio, sometimes referred to 
performance and defined as: $LoD = \frac{C_L}{C_D} = \frac{L}{D}$, where 
the prefix \emph{C} stands for coefficient and the letter \emph{L} and \emph{D} 
are denoted as lift and drag, respectively. $AoA$ is the \textbf{a}ngle 
\emph{o}f \textbf{a}ttack 
and is commonly denoted with the greek letter $\alpha$. $TSFC$ is 
called the \textbf{T}hrust \textbf{S}pecific \textbf{F}uel \textbf{C}onsumption.
In order to get an interpolation model for the state and gradient variables, 
the database must contain values for those and their associated inputs
$Ma, h, mass$. The $mass$ is the total mass at the related condition
for which the aircraft is trimmed.
In simpler terms, considering figure \ref{fig_2_Lod_Interpol},
it can be observed that for combinations of $Ma, h, mass$ 
their associated known values for $LoD$ are passed to an interpolation generator. 
Having obtained an interpolation model, for any given set of 
$Ma, h, mass$ the corresponding $\tilde{LoD}$ can be calculated. It should 
clearly stated that, the closer the input set ($Ma, h, mass$) 
values are to those which were used to generate the interpolation model,
the better the prediction for the outcome $\tilde{LoD},
\tilde{AoA}, \tilde{TSFC}$ is. This means, if 
the given input set is highly out of the range of the training data, 
then an absurd extrapolation may occur. Figure \ref{fig_2_Lod_Interpol} 
only depicts the workflow for obtaining an interpolation model for 
$LoD$. However, analogously to the explained proceeding by only 
replacing $LoD$ with the desired output variable ($AoA, TSFC$ and their 
gradients), 3 interpolation models for the state variables and the
remaining for the gradients can be achieved. \newline 

Since for each state variable ($LoD, AoA, TSFC$) the gradient with 
respect to each shape parameter is required, the number of the 
interpolation models for the gradients is directly connected
to the number of 
the shape parameters. For example, having 126 shape parameters, 
results in having 126 gradients for each state variable. Thus,
126 interpolation models for the gradients of each state variable
($LoD, AoA, TSFC$) 
is necessary. In total, $126*3 = 378$ interpolation models only 
for the gradients are required.\newline 
% ============================================
% ================Interpolation =====================
% ============================================

\begin{figure}[!h]
    \def\svgwidth{\linewidth}
    \input{2_Figures/1_Task/2_LoD.pdf_tex}
    \caption{Exemplary interpolation model generation}
    \label{fig_2_Lod_Interpol}
\end{figure}

Even though it might seem to be a heavy computation it 
is much faster compared to the 
CFD calculation performed for each sample point 
to provide the state variables 
and their gradients as training data.
By using the interpolation approach the integration of 
the Breguet's range equation and the iteration 
of the mission fuel consumption 
becomes feasible although using 
accurate but time-expensive CFD-based data. Furthermore,
the tool 
scales nicely with an increasing number of 
missions.
However, 
the disadvantage is that the quality of the calculations 
completely depends on the quality of the interpolation models. The 
interpolation model itself depends on the accuracy of 
the provided input or training data as well as the 
underlying method. Furthermore, 
the final goal is to use the \emph{missioninformer} within a 
gradient based optimization. The optimizing algorithm then 
heavily relies on the correctness of the gradients. 
Additionally, gradient computations itself is a sensitive 
calculation to do. With the importance of the correct chosen 
interpolation model in mind, in section \ref{sec_Surrogates}
two commonly used interpolation methods and several options are investigated.
\newline


The database containing the training samples 
is given by the output of \emph{DLR}'s aerodynamic analysis and 
optimization workflow \emph{FSAerOpt} 
\cite{merle_high-fidelity_2019} involving the 
flow solver \emph{TAU}.
It is calculated at trimmed states, however, trimming is not always
feasible. 
In this case, \emph{missioninformer} automatically identifies the not 
trimmed values and does not include them into the intern \emph{Python} 
database. This feature is part of the modular skills of the 
missioninformer. It thus enables an easy extension or insertion 
to an already existing workflow. This is reinforced by the type of output, 
i.e. the total fuel burn and their gradients are stored as simple ASCII text files 
and as NumPy arrays as well. In case of a \emph{Python} based 
workflow, the NumPy 
arrays can be loaded into the RAM (Random Access Memory)
with one additional line of code. 

\subsection{Workflow}
A more detailed workflow is depicted in figure \ref{fig_3_Workflow}. 
Before discussing the figure, note that, this is still a broad workflow, and 
it is supposed to provide only a general and not a deep understanding. Among 
other, the cruise segment fuel equation, its derivative, its solution, the used 
interpolation models and outer loops for example for the mass recalculations 
are skipped for now. The
detailed elaboration happens in chapter \ref{chap_Methodlogy}. 
The workflow \ref{fig_3_Workflow} can be read as follows: After 
having generated the 
database containing values for $LoD, AoA, TSFC$ and their 
gradients for a given set of inputs ($Ma,h, mass$), the mission input file 
can be adjusted in order to define the flight mission. In case of 
a multi-mission definition one more step regarding weighting is done, which 
is not depicted in the current workflow overview.
% ============================================
% ================ 2nd Workflow ==============
% ============================================
\begin{sidewaysfigure} [!h]
    \resizebox{\textwidth}{!}{
    
%%% Preamble Requirements %%%
% \usepackage{geometry}
% \usepackage{amsfonts}
% \usepackage{amsmath}
% \usepackage{amssymb}
% \usepackage{tikz}

% Optional packages such as sfmath set through python interface
% \usepackage{sfmath}

% \usetikzlibrary{arrows,chains,positioning,scopes,shapes.geometric,shapes.misc,shadows}

%%% End Preamble Requirements %%%

\input{/home/jav/Progs/Virt_Env/writing/lib/python3.12/site-packages/pyxdsm/diagram_styles.tex}
\begin{tikzpicture}

\matrix[MatrixSetup]{
%Row 0
\node [DataIO] (output_mission_Inf) {$\begin{array}{c}mission input, database\end{array}$};&
&
&
&
&
&
\\
%Row 1
\node [Function] (mission_Inf) {$\begin{array}{c}\text{missioninformer}\end{array}$};&
\node [DataInter] (mission_Inf-st_interpol) {$\begin{array}{c}$$LoD, AoA, TSFC$$\end{array}$};&
\node [DataInter] (mission_Inf-grad_interpol) {$\begin{array}{c}\frac{LoD}{dp}, \frac{d\alpha}{dp}, \frac{dTSFC}{dp}\end{array}$};&
&
&
&
\\
%Row 2
&
\node [Function] (st_interpol) {$\begin{array}{c}\text{state intp. model:} = \tilde{S}\end{array}$};&
&
\node [DataInter] (st_interpol-st_fuel_burn) {$\begin{array}{c} \tilde{S}\end{array}$};&
\node [DataInter] (st_interpol-grad_fuel_burn) {$\begin{array}{c} \tilde{S}\end{array}$};&
&
\\
%Row 3
&
&
\node [Function] (grad_interpol) {$\begin{array}{c}\text{grad intp. model:} = \tilde{G}\end{array}$};&
&
\node [DataInter] (grad_interpol-grad_fuel_burn) {$\begin{array}{c}\tilde{G}\end{array}$};&
&
\\
%Row 4
&
&
&
\node [ImplicitFunction] (st_fuel_burn) {$\begin{array}{c}\text{fuel burn state} \\ \text{equation}\end{array}$};&
&
&
\node [DataIO] (right_output_st_fuel_burn) {$\begin{array}{c}\text{state fuel burnt}\end{array}$};\\
%Row 5
&
&
&
&
\node [ImplicitFunction] (grad_fuel_burn) {$\begin{array}{c}\text{fuel burn gradient} \\ \text{equation}\end{array}$};&
&
\node [DataIO] (right_output_grad_fuel_burn) {$\begin{array}{c}\text{gradients fuel burnt}\end{array}$};\\
%Row 6
&
&
&
&
&
&
\\
};

% XDSM process chains


\begin{pgfonlayer}{data}
\path
% Horizontal edges
(mission_Inf) edge [DataLine] (mission_Inf-st_interpol)
(mission_Inf) edge [DataLine] (mission_Inf-grad_interpol)
(st_interpol) edge [DataLine] (st_interpol-st_fuel_burn)
(st_interpol) edge [DataLine] (st_interpol-grad_fuel_burn)
(grad_interpol) edge [DataLine] (grad_interpol-grad_fuel_burn)
(st_fuel_burn) edge [DataLine] (right_output_st_fuel_burn)
(grad_fuel_burn) edge [DataLine] (right_output_grad_fuel_burn)
% Vertical edges
(mission_Inf-st_interpol) edge [DataLine] (st_interpol)
(mission_Inf-grad_interpol) edge [DataLine] (grad_interpol)
(st_interpol-st_fuel_burn) edge [DataLine] (st_fuel_burn)
(st_interpol-grad_fuel_burn) edge [DataLine] (grad_fuel_burn)
(grad_interpol-grad_fuel_burn) edge [DataLine] (grad_fuel_burn)
(mission_Inf) edge [DataLine] (output_mission_Inf);
\end{pgfonlayer}

\end{tikzpicture}

    }
    \caption{Second workflow overview with some more details}
    \label{fig_3_Workflow}
\end{sidewaysfigure}
With these 
inputs \emph{missioninformer} generates interpolation models for states and 
for gradients. In order to get the mass of cruise segment fuel,  
its equation is solved. This equation requires the state 
surrogate models to be called. In fact, the solving of 
state cruise segment fuel equation is numerical and iterative and thus requires 
multiple function calls. In other words,  
solving the cruise segment fuel equation once, invokes 
the surrogate models multiple times. The gradients of 
the cruise segment fuel mass needs the state and the gradient 
interpolation models and also invokes both models 
multiple times for one solution. Contrary to the 
generation of the interpolation models, the invocation of 
the already calculated interpolation model is not 
computationally costly. As mentioned, a deeper insight into 
the whole process is provided in 
chapter \ref{chap_Methodlogy}. Once the state and gradients 
of the cruise segment fuel mass are calculated they are written to 
hard disk as output.


    
