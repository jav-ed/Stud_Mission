

\chapter{Introduction}
\label{chap_1_Intro}
This work is written for the \emph{Technische Universiät Braunschweig}
in a cooperation with the German Aerospace Center (\emph{DLR}).
It is the result of research topic defined by my supervisor 
Andrei Merle and Prof. Stefan Görtz, both working for the DLR. The overall goal, 
in very brief terms, is to calculate the fuel consumption 
and its gradients with respect to later introduced shape parameters 
for one or multiple individual flight missions. The developed tool for that purpose 
is called \emph{missioninformer}.
The workflow of the \emph{missioninformer} can be explained  
with figure \ref{fig_1_Missinfo}. \emph{Missioninformer}
can be treated as a black box, which makes its implementation into 
existing analysis and optimization processes
easier. It receives an input, the so-called 
\emph{mission input}, which contains descriptive information about 
a flight mission, e.g. payload, cruise range, flight Mach number (a 
more detailed explanation will be given in section \ref{fig_1_Missinfo})
and aerodynamical data.
With these inputs \emph{missioninformer} provides the user with 2 outputs.
The first one is a state value, it is the amount of fuel in kg which is 
required for the mission. The second output are the gradients of 
the fuel for the mission with respect to arbitrary aircraft 
shape parameters. \newline


% ===================================================
% ==================== Motivation ===================
% ===================================================
\section{Motivation}
\label{sec_Motivation}
Aviation has become an 
indispensable part of global economy. Its importance is 
evident in the private sector, where it is connecting
destinations worldwide
and in the business sector, where goods are transported and 
business trips are undertaken. Since an aircraft burns fuel and thus 
emits carbon dioxide (CO2), nitrogen oxide (NOx) and 
sulphur dioxide (SO2) its impact on 
the environment is significant 
and cannot be neglected. Emissions are so high that the European commission has 
formulated a document called Flightpath 2050, where it defines two 
of its goals to be reduction of CO2 and NOx emission by 75\% 
per passenger kilometer
and 90\%, respectively. 
Currently, commercial aviation is responsible
 for around 
3.5 \% of the global carbon dioxide and nitrogen 
oxide 
emission \cite{lee_contribution_2021}
and it can be assumed that in an increasingly globalized world the demand 
for fast and convenient mobility will not diminish. 
Industry and governments 
around the world realized that taking active steps towards a cleaner 
flying are essential for saving the planet earth. In 2010, the 
international Civil Aviation Organization (\emph{ICAO}) formulated 
industry-wide goals for reducing carbon emissions. 
These goals can be summarized by 2 main 
objectives, to establish carbon-neutral growth beyond  
year 2020 and to further reduce carbon emissions to half of 
the current level by the year 2050.\newline

One simple way of reducing the overall carbon emissions  
already set to practice is the application of single-engine taxiing. 
With a single operating engine
the fuel emissions are reduced only 
for the time in which the taxiing and waiting are carried out.
Considering a whole flight mission, taxiing and waiting
represents only a small fraction of the mission. Thus, this is 
not enough. Other techniques to reduce fuel consumption 
already in use are flying at an optimal cruise speed or 
using more environmental-friendly handbook trajectories. Methods 
which still acquire much exploration, however, 
can be assumed to play a main part in reducing fuel consumption.
These are overall new 
aircraft configurations, e.g. blended wing body and applying
technologies like laminar flow control, boundary layer 
ingestion, ultra-high bypass ratio engines, more and all electric 
aircraft. The possibility 
to combine multiple technologies is given and 
is expected to lead to best results. The proposed \emph{missioninformer} 
is not restricted to any aircraft configuration nor 
to any new applied technological advancements.\newline

The demand for more environmental-friendly 
aircraft is 
becoming steadily louder. Furthermore, fuel price 
has increased and is 
forecasted to increase in the future. Among others, one 
of the main objectives of
the \emph{DLR}'s institute 
\emph{Aerodynamics and Flow Technology} is to fulfill the described demands. 
Therefore, several 
optimization process were developed here
to find  structural, aerodynamical and 
engine optimized aircraft designs, 
details can be found in \cite{gortz_collaborative_2016, goertz_overview_2020}. 
The fuel consumption 
is targeted as the objective function inside an optimization process. The big 
advantage of such an optimization would result in reducing the CO2 emissions and 
thus pleasing the environmental demands. The \emph{missioninformer} is 
written in a modular manner as mentioned earlier in this chapter.
Therefore, it can be easily implemented into the \emph{DLR}'s 
multi-disciplinary, -point and -objective optimization frameworks. 
Also, the user is enabled to define his own mission or multiple missions. 
Therefore, this feature could be used to enhance the already existing 
multi-point capabilities for the optimization. The importance 
of a multi-mission-optimization becomes visible, when thinking about 
an aircraft, which is used for a wide variety of missions, e.g. 
for different kinds of ranges, flight altitudes, speeds and payloads. 
In case of a single mission optimization, the given 
flight parameter will be suited optimally. However, a small 
change in these conditions could result into unbearable consequences of
the fuel consumption. 
In practice, airlines want to buy aircrafts, which can be used 
for more than just one mission. A multi-mission optimized 
aircraft performs not as good for each condition, however it 
will cover a variety of missions and conditions within reason. The 
missioninformer, however, is not only suited for multi-point 
optimization, rather the enhanced multi-mission optimization.
For instance in the previously mentioned processes masses for which 
the aircraft is trimmed are defined.
Masses are defined through the method of multi-points, i.e.
these are 
only frozen and discontinuous discrete aerodynamic states for which 
an optimization is carried out. In case of the multi-mission
capable \emph{missioninformer}, 
the masses are defined continuously, which comes closer 
to the reality of flying and thus consuming 
fuel. Furthermore, the \emph{missioninformer} takes 
the snowball mass effect, which is described in 
section \ref{sec_Fuel_Mass_iter} into account. Also, 
since the input data is given by the aerodynamic output 
of \emph{DLR}' analysis and 
optimization workflow \emph{FSAerOpt} 
\cite{merle_high-fidelity_2019} involving the 
flow solver \emph{TAU}, each mission fuel mass 
computation gets 
aerodynamical data for a fully trimmed 
state.
\newline


With this introduction the main 8 tasks for 
developing the \emph{missionionformer} from scratch shall be mentioned:
\begin{enumerate}
    \item Literature review on mission analysis and review of Breguet's range formula.
    \item Familiarization with the gradient-based optimization process
    \item Conception of an interpolation module for mission analysis
    \item Implementation of the Python module
    \item Derivation of the module according to the optimization parameters
    \item Verification of the derivation in case the derivation is done by complex-step or analytically
    \item Interface for a gradient-based optimization process
    \item Identification of a best practice
  \end{enumerate}
As far it is possible, each task will be 
elaborated in this article. Since the tasks required much coding, 
it is important to 
mention the used programming language and the dependencies. 
As for the programming language,
\emph{Python 2} and \emph{Python 3} were chosen. For the libraries  
\emph{Matplotlib, Scipy, NumPy} and \emph{DLR}'s own 
Python library \emph{SMARTy} are deployed. For local coding a 
Linux workstation provided 
by the \emph{DLR} with the following hardware configuration 
was used: \emph{CPU:} Intel(R) Xeon(R)
CPU E3-1270 v3 @3.50GHz 
and 32GB of RAM. Calculations were performed on the  following high 
performance computer \emph{CARA}: \emph{CPU:} 2x AMD EPYC 7601 
(32 cores; 2,2 GHz) per node 
and 2168 nodes with 128 GB DDR4 (2666 MHz) RAM. 
All calculations performed in 
this report used the \emph{64} cores option. Some 
parts of \emph{missioninformer} where NumPy is used, can 
be assumed to benefit from this option. However, the 
missioninformer itself is not parallized.

% ===================================================
% ==================== STATE OF THE ART =============
% ===================================================
\section{State of the art}
\label{sec_1_State}
The idea to reduce the entire fuel consumption is not
new and a summary
of the available methods is given in \cite{betts_survey_1998} 
and a more recent one in \cite{kao_modular_2015}. However,
some works still shall be recited. Flight Optimization System 
(\emph{FLOPS}) is a modular approach allowing the user to select 
appropriate modules for a given mission \cite{mccullers_aircraft_1984}.
The aerodynamical data are generated by using the \emph{Delta Method}, 
which is an empirical drag prediction method \cite{feagin_delta_1978}. 
\emph{FLOPS} enables some numerical optimization schemes, e.g. Davidon-Fletcher-Powell,
Broyden-Fletcher-Goldfarb-Shano and a Quadratic Extended Interior Penalty method 
\cite{kao_modular_2015}. The gradients are calculated using finite 
differing, which is easy to apply, however an adequate 
derivation step size is required. The accuracy of gradients may 
not be sufficient and in case of a high number of design variables 
parallelization should be considered in order to get gradients 
in an acceptable time limit. Transport Aircraft System Optimization 
(\emph{TASOPT}) is another well known tool for mission analysis and 
optimization \cite{greitzer_design_2010}. The benefit of the latter 
is that physics-based models, rather than empirical data-tables are 
applied. Using 
data-tables results in interpolation could result into not 
realizable designs. The physics for \emph{TASOPT} is obtained mostly by low-order 
models.\newline 

\emph{pyACDT} is an aircraft conceptual design tool \cite{perez_pyacdt_2008} 
containing a mission analysis component. It uses the Breguet 
range equation and empirically obtained fuel fractions for each 
flight mission segment. Liem et al. \cite{liem_aerostructural_2013} 
take a similar 
approach, however with the advancement, that fuel fractions are only 
going to be used for the startup, taxi and landing segments of the mission.
The fuel burn for climb, cruise and descent segments are calculated 
using a range equation and the flight equilibrium equations. The 
different segments are linked together such that the fuel weights at 
the end points of two neighboring segments must be equal. 
Another well known tool for aircraft conceptual design 
is developed by Lissys Ltd and is called \emph{PIANO}. It 
is freely available, however, only for Windows and thus 
not compatible for HPC-based optimization processes.
\newline 

Probably the two most well known mission analysis tools 
are \emph{pyMission} and \emph{SUAVE}. 
\emph{pyMission} is developed by the authors of 
\cite{kao_modular_2015}
and is now part of the openMDAO framework \cite{gray_openmdao_2019}. It 
uses the flight equilibrium equations and a fuel 
burn rate equation. In order to solve the fuel
burn rate equation, which is an ODE, the amount 
of fuel carried at the end of the mission 
is provided as the initial condition. The explicit Euler 
scheme starting from the end of the mission is applied. 
Thus, marching backwards in distance to the start 
of the mission is performed. For the rest of the mission 
fuel fractions are used. The gradients 
(total derivatives) for 
the optimization performed are obtained 
through the adjoint method \cite{martins_review_2013}. 
\emph{SUAVE} is a conceptual 
aircraft design framework developed at  
Stanford University and can be used for conventional 
and unconventional configurations. Tutorials as well 
as documentation for \emph{SAVE}'s mission analysis tool 
can be found on their 
homepage \cite{noauthor_homepage_2021}. 
In \cite{liem_multimission_2015, liem_aerostructural_2013,
liem_surrogate_2015} the mission analysis 
is applied by using surrogate models and mixture 
of experts. Different kind 
of Kriging methods and RBFs (Radial Basis Function) are used 
to get interpolation models for lift, drag and pitching 
moment coefficients from a four-dimensional input space with 
variables, Mach number, angle of attack, flight 
altitude and tail rotation angle. For startup, taxi, takeoff 
and landing fuel fractions are used. The fuel burn for 
climb, cruise and descent segments are derived through 
numerical integration of a range equation. 
The results of the paper can be summarized as follows: 
the adaptive sampling improves the accuracy of surrogate models, 
the convergence was to be found slow in some cases, particularly when modeling 
complex profiles. Traditional 
surrogate models (RBF and Kriging without mixture of experts)
performed well for the simpler 
lift and pitching moment coefficient ($C_L, C_M$)
profiles. Using a mixture of gradient enhanced Kriging (GEK)
models to approximate drag
coefficients gave approximation errors of less than 
5\% with less than 150 samples, whereas the adaptive sampling
failed to converge when training a global model. Since 
the traditional surrogate models approximate 
less complex behavior well, the authors of 
the papers recommend applying traditional 
surrogate modelling instead of mixture 
of experts. Latter requires much effort 
for its implementation.
Finally, it is worth mentioning Dabas et al., who optimize
the pylon shape by coupling an aerodynamic
optimization with a mission
analysis tool in order to propagate aerodynamic 
shape modifications to mission level \cite{dabas_error-based_2019}. The objective 
function, however is not directly the 
amount of fuel burnt, but rather the Cash Operating Costs (COC),
which contain the burnt fuel as part of its weighted components.
